\chapter{Methods of Data Collection \& Analysis}

\section{Introduction} 

\emph{This chapter describes in detail the range of empirical research methods used to carry out this project. This includes the design of a case study, as well as the various methods used for data collection and analysis}

\emph{I begin by describing the case study design, including the overall structure of the case and its background. I then describe the methods of data collection used in this work, including a set of ethnographic studies carried out over a three year period, three informetric studies that provide a quantitative account of ICOADS use and acknowledgment, and a set of semi-structured interviews conducted in year three of the study. I conclude by describing the use of two existing frameworks to organize and analyze these different data sources, and the validity constructs that have guided this analysis.} 

\section{Case Study Design}

The marked features of a case study are that it:\\

\begin{itemize}
\itemsep1pt\parskip0pt\parsep0pt
\item
  Investigates a contemporary phenomenon within its real-life context
\item
  Is appropriate when the boundaries between phenomenon and context are
  not clearly evident
\item
  Copes with the technically distinctive situation in which there will
  be many more variables of interest than data points
\item
  Relies on multiple sources of evidence, with data needing to converge
  in a triangulating fashion
\item
  Benefits from the prior development of theoretical propositions to
  guide data collection and analysis \citep[Quoted from][]{yin2014case}.
\end{itemize}

In social science research that takes things like organizations, institutions or systems as a unit of analysis, the case study approach can be used to test, extend, or further develop existing theories, or empirical frameworks \citep{stake1995art}. This case study will adapt and extent existing socioecological frameworks for  studying sustainability issues in sociotechnical systems settings.  

The research design of a case study requires the following elements:

\begin{enumerate}
\item 
 Clear definition of the case type
\item
  Research Questions
\item
  Propositions
\item
  Unit(s) of Analysis
\item
  Logic linking the data to the propositions
\item
  Criteria for interpreting the findings (Yin, 2014). 
\end{enumerate}

Below, I describe the design choices made for the ICOADS case study. 

\section*{Research Design for Case Study of ICOADS}

\subsection{Type of Case Study}

Distinctions in case study types are generally made along categorical lines such as  exploratory, descriptive, explanatory, or critical \citep{flyvbjerg2006five}.  The following case study of ICOADS is most comparable to the descritptive approach, as it attempts to describe how and why a certain institutional arrangement has been successful at sustaining a shared resource set over a period of time.\\

\subsection{Research Questions}

The specific research questions that this case study will answer are:
\begin{itemize}
\item\textbf{CS RQ 1} : What are the different governance models that ICOADS has used to manage shared resources over its thirty year existence (1983-present)?

To reiterate from Chapter 1, I rely on the definition of a governance system as``a set of institutional arrangements (such as rules, policies, and governance activities) that are used by one or more actor groups to interact with and govern interactions.''(Cox et al., 2014). 

\item\textbf{CS RQ 2} What causes a governance system to shift from one regime to another?

A regime shift is, following Smith, Stirling, and Berkhout (2005), understood to be a function of two processes:

\begin{enumerate}
\item Shifting selection pressures (either external or internal pressure to change) bearing on the regime.
\item The coordination of resources available inside and outside the regime to adapt to these pressures. 
\end{enumerate}

\item\textbf{CS RQ 3} What types of disturbances are ICOADS resilient or vulnerable to?
\end{itemize}

In order to identify selection pressures, governance shifts, and disturbance types, I use a systematic approach to code system state variables related to ICOADS different governance models. This approach is described in the ``Criteria for Interpreting Results of Case Study'' section below.  

The broader research questions being pursued in this dissertation are: 

\begin{itemize}
\item What are the effective institutional arrangements (governance) for sustainable community based science infrastructures? 

\item How and why do they differ between domains of knowledge production?
\end{itemize}

These two research questions are addressed by synthesizing the results of this case study with those from previous cases using the same analytical framework. I describe this synthesis work in detail at the end of this chapter. 

\subsection{Unit(s) of Analysis}

In short, a unit of analysis is the subject of a case study, and units of observation are individual measures of the subject (Long, 2004. 

The unit of analysis in this case study is ICOADS as a socio-technical system. This is a single case study, but inter-case comparison will be conducted through exploration of different governance regimes adopted over the history of ICOADS. Smith, Stirling, and Berkhout provide a nice description of the need to expand a unit of analysis in this type of work, 
\begin{quote}
``Conventional economic analysis of technical change tends to focus on pressures that operate visibly at the level of the firm (such as pricing, competition, contracts, taxes and charges, regulations, standards, liability, profitability, skills and knowledge). Analysis at the level of the socio-technical regime, on the other hand, includes such factors, but goes beyond them to consider less economically visible pressures emanating from institutional structures and conventions, including changes in broad political economic ‘landscapes’, or wider socio-cultural attitudes and trend''(Smith, Stirling & Berkhout, 2005). 
\end{quote}
In this case study, there are multiple units of observation, including, users, producers, provisioners, and stakeholders of the ICOADS project, as well as the digital resources themselves ( e.g. ICOADS software and data)\\

\section{Data Collection Methods}

A number of different methods were used to collect data for this case study. With the exception of the semi-structured interviews completed in year three each of these studies have been published in peer-reviewed outlets. Where this is the case, I provide bibliographic references for that work. Below I describe these methods in detail. Each of the sections that follow are structured to include the following: 
\begin{itemize}
\item A description of the intention of research method used (what particular phenomenon it was useful for understanding), 
\item A brief review of the background literature that informed my operationalization of this approach, and
\item Description of the data (range, type, etc.) that were produced by an individual study.  
\end{itemize}

\subsection{Preparation for Fieldwork}

Collecting data of any kind requires a certain amount of familiarity with the topic, and subject that will be studied. This is an especially important and diverse topic for contemporary science studies where interdisciplinary research is conducted (Jirotka, Lee, and Olson, 2013). In preparation for doing fieldwork amongst experts in Marine Climatology, and for further enchaining my understanding of this field, I undertook following activities: 

\begin{itemize}
\item In the Fall of 2012, and 2014 I enrolled in ATMS 591 Atmospheric Sciences Seminar at the University of Illinois, Urbana-Champaign. This is an open discussion and lecture series featuring topics of importance to the field of meteorology, atmospheric science, and earth systems science more generally. 

\item I also completed four on-line (MOOC) courses with passing marks in the following related subjects: 
\begin{itemize}
\item``Marine Megafauna: An Introduction to Marine Science and Conservation'' (Duke University)
\item``Climate Literacy: Navigating Climate Change Conversations'' (University of British Columbia)
\item``Global Warming: The Science and Modeling of Climate Change'' (The University of Chicago)
\item``Climate Change'' (The University of Melbourne)
\end{itemize}

\item During both residencies at NCAR (described below) I attended weekly lectures at NCAR, including the Advanced Studies Program lecture series. This gave me the opportunity to both interact with visiting scientists, and describe the research agenda that I was involved in, further adding to my own competency in the science behind my participant's work.
\end{itemize}

\section{Ethnographic Study of ICOADS}

Two forms of ethnographic inquiry informed this project: In June of 2012 I began by using an interactional approach borrowed from the field of Science and Technology Studies (STS), and over time I also adopted a complimentary `historical ethnographic' approach as it is conducted in the field of Organizational Behavior. I describe each approach in detail below.  

\subsubsection{Participant Observation: Interactional Expertise}

\textbf{Methodological Intention}

Participant observation allows for ethnographic data to be collected in a natural context, adding a dimension of inter-personal authenticity to what are the otherwise sterile or staged interactions of a formal interview process. My intention in using this method of data collection was to capture the everyday lived experiences of developers, scientists, and engineers engaged in the ICOADS community. As these individuals are members of diverse academic, social, and cultural backgrounds they have a very different view of the phenomena of governance. In observing, participating in, and often arguing about climate science I was able to better understand and interpret behaviors within the community that I was engaged in studying.

The traditional focus of this type of ethnography in on ``unobtrusive observation'' - that is, entering, observing, and leaving a site with minimal impact on the everyday lives of the participants (Fine, 1993). Some of the earliest studies of laboratory cultures in STS contested the value of this strict form of participant observation; as Woolgar and Latour argued, the need to be inconspicuous and innocuous was detrimental to their participants own understanding and reflection on basic, routinized steps in a research process (Woolgar and Latour, 1979 p. 40). My intention in using an ethnographic approach to data collection is 

\textbf{Background Literature}

Over a three year period  (June 2012 – April 2015) I engaged the ICOADS community - including developers, funders, end-users, collaborators and previous project participants - at multiple sites of study, including a significant amount of time (15 months total) as a research fellow at the National Center for Atmospheric Research (NCAR) in Boulder, Colorado. My collection of data in this setting follows what Harry Collins calls “participant comprehension.” This mode of inquiry is informed by traditional ethnomethodology as found in cultural anthropology (e.g. Garfinkel, 196X) and science studies (e.g. Lynch, 1997). Harry Collins' work on participant comprehension began in 1984,  when studying the material culture of gravitational wave physics he described  ``an interpretation of participant observation under which the field-worker tries to acquire as high a degree of native competence as possible and interaction is maximized without worrying about disturbing the field site.'' (1998, p. 297).    
 
Collins' motivation for developing the extended form of description that participant comprehension generates is in pursuit of an interactional expertise, or a third kind of knowledge that sits between formal or propositional knowledge and informal or tacit knowledge (2004, p. 125-7). Collins' argument is that during his time studying gravitational wave physics he was able to achieve an expertise which was something different than the kind that comes with formally studying and being a part of a discipline as a practitioner. Through linguistics socialization, traditional ethnographic observation and extended field work, Collins could read, participate in discussions, and defend his arguments about physics, without having the ability to actually do the physics himself. 

My time working in this field, as well as studying and using ICOADS has allowed me to develop something like an interactional expert – I understand the limitations and uses of the data in ways that allow me to converse with its community of users. Six months into this project, I was elected to the American Meteorological Society's Board of Data Stewardship.After a year and a half of interaction, I was even been invited to give a talk at a biennial marine climatology meeting (Weber and Worley, 2014). Neither of these accolades mean that I am able to do marine climatology research, but they demonstrate that I can developed a level of mutual respect within this community, and was able to eventually access, manipulate, and contribute to data and curation in my performing an ethnographic mode of inquiry.   

 
\textbf{Operationalized Study}

Like Collins, my interaction is maximized by informally interviewing, talking with, and observing people who produce and use ICOADS; The operationalized study included the following periods of data collection: 

\begin{itemize}
\item From June - August of 2012 I was an Advanced Studies Program (ASP) fellow at the National Center for Atmospehric Research (NCAR) in Boulder, Colorado. While there, I observed and interacted with a team of four individuals that curate ICOADS at NCAR. I also informally interviewed scientists who used ICOADS, asking basic questions about their experience, and their interaction with the project over time. During this time, I also established a formal relationship with one of ICOADS original developers who was willing to become my ``key informant'' (Payne, 2004). This individual sponsored my project within the ICOADS community, and became a mentor of sorts to me in the field of marine climatology, and data science more generally.  

\item After leaving the site, I continued to work with this team of four curators, exchanging emails, and calling into  teleconferences about their work on ICOADS. 

\item In August, 2013 I returned to NCAR to collaborate with this team of four individuals, and a larger network of visiting scientists and ICOADS partners in the Boulder, CO area.

\item In August 2014 I left NCAR, but continued to interact with broader network, including regular participation in teleconference calls with my main informant and a steering committee established to govern ICOADS.

\item During this time I also attended the following meetings where I either presented my on-going informetric work, or observed my participants presenting their own work: 2013 American Geophysical Union (AGU) Winter Meeting; 2011-2014 American Meteorological Society (AMS) Annual Meetings; 2012-2015 Earth Science Informatics Partners (ESIP) Winter and Summer meetings; the 2014 Joint Technical Commission for Oceanography and Marine Meteorology (JCOMM) Workshop on Advances in Marine Climatology (CLIMAR-IV).  

\end{itemize}

\textbf{Data Collected}

Forms of data collected in the ethnographic work includes field and interview notes, memos (which summarized my observations, and were later refined and shared with my participants for validation (Emerson, 2011); I subscribed to a current email list and participated in conversations there; I gathered general documentation about the production of ICOADS itself- meaning user manuals, production notes for programmers working on transformation of different versions of ICOADS, as well as formally published proceedings of meetings, meeting minutes, and journal articles describing the project dating from as early as 1981.   

I made a conscious decision not to record interviews conducted during this stage of research. I came to this decision after initially attempting to record conversations and observing discomfort of my participants in having their conversations taped. I made every effort to announce, and clearly communicate that my participation was part of a dissertation focusing on ICOADS governance. I also sent an email to project partners in ICOADS outlining the major goals of my dissertation, which they approved and sent back their encouragement. 

\subsection{Document Analysis: Historical Ethnography}

\textbf{Methodological Intention}

In order to meaningfully answer research questions about the historical evolution of ICOADS' governance regimes, I also used an approach to gathering historical data from the field of Organizational Behavior. Historical Ethnography is a technique pioneered by Diane Vaughn in her work with NASA and a USA Presidential Panel in the aftermath of the \emph{Challenger disaster} (Vaughn, 2006; 2009). The approach requires balancing archival work with present-day validation of participants' memory of those past events. It can be described loosely a a process of ``digging into the past, deliberately reconstructing history in order to identify and then track the processes connecting past and present.'' (1999) What makes my work a historical ethnography, as opposed to pure history, is that it includes a sensitivity to present day events and a deep commitment to tying those events to the history out of which they were borne. This is the major source of data for the ``Background Environment'' section of Chapter 4.   

\textbf{Background Literature}

Vaughn's work with NASA was based on what she called ``documentary practices'' within NASA as an organization (Vaughn, 2006). Trying to better understand on-going Congressional testimony, she sought an explanation of how deviance - through preference for risk, and safety standards - diverged between engineering and management cultures. The breakdown of communication between the two groups was traced through reports filed by engineers in previous shuttle missions, as well as \emph{Challenger} flight simulations. By connecting these reports, their language, and the documentary trail of decision making processes that informed a NASA management decisions, she could ultimately provide a more thorough explanation of the disaster based on organizational breakdowns in safety reportage. 

Another germane historical ethnography can be found in John Middleton's study of early Modern Era Swahili merchants. Middleton drew upon Vaughn's method of historical ethnography in order to show how certain artifacts (coins, dock locations, merchant societies, etc.) have persisted in structural shape throughout Eastern African. This continuity created a porous boundary between present and past in divisions of class, race, and gender. Middleton's research, although based on historical documents, was grounded by `collective memory' of cultures he observed and lived within, and by `extrapolation from modern ethnography' in current mercantile society's of East Africa (Middleton, 2013).   

\textbf{Operationalized Study \& Data Collected}

I conducted archival work in two physical locations: the National Center for Atmospheric Research archives, as well as NOAA's Earth System Research Laboratory (ESRL) which has a small but useful set of papers related to the COADS project; both are located in Boulder, Colorado. I also asked each participant that I interacted with to provide me with email archives, past meeting minutes, and any documentation related to previous ICOADS development. NOAA has a number of digitized historical texts available online at (http://www.history.noaa.gov/index.html), as well as links out to historical material that covers marine climatology and its history. JCOMM and NOAA digitally archive the proceedings of two major conferences related to marine climatology (CLIMAR and MARCDAT) dating back to the first meeting in 1999. Through my key informant, I was also able to obtain an archived discussion forum that recorded early emails exchanged by ICOADS developers.

In total, I gathered over 10 GB of digital material - of which I read through, created memos, and used for discussion with my informants. This was also an iterative process of tracing citations, and trying to establish the provenance of certain ideas or even ideologies. This process eventually led me to Matthew Fontaine Muary, a USA Naval officer who played a key role in both naval history, and weather data standardization. Maury's personal papers, as well as valuable secondary sources were digitized and are available online through the Virginia Military Institutes's University Archives\footnote{http://www.vmi.edu/archives.aspx?id=19209#fulltext}. 

\section{Informetric Studies}

One of the social dilemmas that we will see ICOADS face is related to evaluation in that - like an infrastructure - the diversity of metrics defining success of the project is one of its chief hurdles in securing reliable funding. The informetric studies conducted in this dissertation are a heuristic attempt at gathering ``demand-side'' criteria for the provisioning of ICOADS. As opposed to supply side criteria, such as the amount of funding ICOADS has received, the number of records in its archive, the number of papers it has produced, etc, demand side informetric work tries to determine, who uses, what, when, and why? 

I stress that these are imperfect studies in the traditional Information Science sense of having robust statistical measures of significance, however they are another data source that provides a means for triangulation on what are a broader set of research questions about the sustainability of a shared resource. I return to the discussion of metrics in Chapter 5.   

\subsection{Citation Content Analysis}

\textbf{Methodological Intention}

The premise of this study is to use close reading techniques to better understand the context in which ICOADS has been referred to in the formal literature of marine climatology (1985-2014). As a result of failed experiments in calculating traditional forms of bibliometric impact for the ICOADS project, this study was designed to give a more thorough account of not only how frequently something is cited, but for which reasons, and with what sentiment. This approach may be especially salient for the earth systems science community, because of the tradition of writing data ``papers'' which announce or describe a new data-related product (Mayernik et al., 2014). These publications are seen as a way for data producers to announce an update (newly digitized material, newly processed or interpolated data points, new grid resolutions etc.), or to comprehensively describe the process of aggregating and developing a new archive of software and data. The context in which citations are made to papers about ICOADS can offer a way to deepen an understanding of how the archive has been used over time, and how users have judged its quality, reliability, utility and value.  

A ``data paper'' may be cited in a number of ways - to document the use of a certain data source (example 1 below), or it may be cited for the propositional content that the data paper contains - such as assertions it makes about the coverage or completeness of the data being released (example 2, below). All of this is to recognize that journal publications are complex documents that play a number of different and sometimes conflicting roles.\\ 

\begin{itemize}
\item \textbf{Citation 1}

\emph{The error properties of both input data in OA procedure are obtained using Comprehensive Ocean–Atmosphere Dataset (COADS) (Woodruff et al. 1998) ship observations as the base data. (Wheeler et al, 2008)}

\item\textbf{Citation 2}

\emph{Thus, the B150\% CFC supersaturation levels observed in the summer of 1988 are probably the result of seasonal warming combined with the substantially lower mean wind speeds in spring and summer (Woodruff et al., 1998) allowing only partial reequilibration of the upper water during the warming. (Lee et al. 2002)}
\end{itemize}

\textbf{Background Literature}

Similar informetric work has focused on developing a theory of citation (Cronin, 1984; and Leydersdorff, 1984), or typing citations in a formulaic way so as to describe their function in a scholarly communication system (White, 2004). Attempts to type citations based on their function within a formal publication dates back as far as the 1970's with Small's work in citation context (1982) and symbol (1978), Pertiz's study of citation roles (1983), and Moravcsik and Murugesan's typology of citation functions (1975). These early studies were conducted in the traditions of library and information science, systems development and analysis, and applied linguistics; as such the classifications of these citation elements were aimed at improving the indexing of a document or information retrieval system. In short, previous works that analyzed the context and content of citations were, as Zhang, Milosevic and Ding put it ``...constructed more from the perspective of users’ needs and perceptions rather than from those of the citing authors, especially in terms of authors’ citing motivations. ''(2013)

Recent work on extracting and coding the context of citations has focused on sentence level content analysis to improve abstracting services (Nakov, Schwartz, and Hearst, 2013), as well as the semantic and syntactic coding of “citation contexts” to create a comprehensive schema of citation behavior (Zhang, Milosavic and Ding, 2013; hereafter referred to as the CCA approach). Both of these studies use methods of qualitative inquiry to mark-up a text manually, and then use these annotations to improve machine-learning algorithms that automate the classification of citations.

Work at manually coding citation contexts had previously been popular in fields like organizational theory (Anderson, 2006; Golden-Biddle et al., 2006; Lounsbury and Carberry, 2005; Mizruchi and Fein, 1999) but there has yet to be a substantive use of this work to automate or improve manual processes.

There have also been a number of citation typing, and scholarly communications ontologies developed for semantically encoding digital publications. The most successful and fully developed of these are the SPAR (semantic publishing and referencing) suite of ontologies. Included in SPAR are the Citation Typing Ontology (CiTO) (Shotton, 2010) and the Scholarly Contributions and Roles Ontology (SCoRO) (Shotton and Peroni, 2013) which provide an exhaustive vocabulary for typing the content of a citation, and the roles played in producing a publication, respectively.


\textbf{Operationalized Study}

There are four major publications that have announced a new release, or a substantive update to ICOADS.
\begin{enumerate}
\item Woodruff, S.D., R.J. Slutz, R.L. Jenne, and P.M. Steurer, 1987: A comprehensive ocean-atmosphere data set. Bulletin of the American Meteorology Society 68, p. 1239-1250.

\item Woodruff, S.D., H.F. Diaz, J.D. Elms, and S.J. Worley, 1998: COADS Release 2 data and metadata enhancements for improvements of marine surface flux fields. Physical Chemistry of the Earth, 23, p. 517-526.

\item Worley, S.J., S.D. Woodruff, R.W. Reynolds, S.J. Lubker, and N. Lott, 2005: ICOADS Release 2.1 data and products. International Journal of Climatology (CLIMAR-II Special Issue), 25, 823-842. doi:10.1002/joc.1166

\item Woodruff, S.D., S.J. Worley, S.J. Lubker, Z. Ji, J.E. Freeman, D.I. Berry, P. Brohan, E.C. Kent, R.W. Reynolds, S.R. Smith, and C. Wilkinson, ()2011) ICOADS Release 2.5: Extensions and enhancements to the surface marine meteorological archive. International Journal Climatology (CLIMAR-III Special Issue), 31, 951-967. doi: 10.1002/joc.2106
\end{enumerate}

Using the bibliographic databases Scopus and Web-of-Knowledge (WoK), I retrieved all citations made to these publications as of August 1, 2013, and then removed duplicate or overlapping citations. The result was ``citing documents corpus'' that contained 1,195 documents. I then used a systematic sampling technique to select documents evenly across the 30 years span of ICOADS publications:

\begin{table}[h]
\begin{tabular}{lllll}
              & \textbf{Citations received} & \textbf{Lapse} & \textbf{Documents sampled per year} & \textbf{Study sample} \\
\textbf{2011} & 57                          & 2              & 7                                   & 15                    \\
\textbf{2005} & 190                         & 8              & 6                                   & 48                    \\
\textbf{1998} & 162                         & 15             & 3                                   & 41                    \\
\textbf{1988} & 786                         & 25             & 4                                   & 128                   \\
              & \textbf{}                   & \textbf{}      & \textbf{}                           & \textbf{252}         
\end{tabular}
\end{table}

\textbf{Data Collected}

Following the practice of Nabokov et al. (2004; 2013) a colleague and I then extracted the sentence in which each ICOADS document was formally cited, and instances where the project was mentioned by name (either formal or by acronym) but was not cited.

We then coded the sentences in which ICOADS was cited or mentioned using the following categories:
 
\begin{itemize} 
\item Number of authors
\item Relation to cited work: Is the citing publication related to the ICOADS project in some way?
\item Location of citations and mentions : Where in the publication is ICOADS cited or mentioned?
\item How ICOADS was referenced: Direct citation, endnote or mention
\item How ICOADS was referred to, materially : What do the authors call ICOADS? (i.e. a dataset, an archive, etc.)
\item Function of the citation: Meant to answer why was ICOADS being cited (Uses CiTO controlled vocabulary)
\item Sentiment of the citation (and mentions) : A citation or mention could be positive, negative, or neutral
\end{itemize}

We chose a small sample (n=10) to initially code together, and iterated on this process for three rounds (n=40 total) until we achieved a consistent inter-rater reliability. We coded all 252 documents in the sample. The result is a set of descriptive statistics for the categories that we coded. The full coding schema and access to bibliographic records can be found in the project repository\footnote{ http://git.io/vJnPX}.

\subsubsection{Data Usage Index}

\textbf{Methodological Intention}

Federally funded research and development centers typically measure the levels of service they provide to end users through descriptive statistics, such as the number of times a data file has been downloaded over a given period of time. To better understand the use of ICOADS in the climate science community I designed a study that drew upon user-log analysis techniques from Information Science \citep{jansen2006search, nicholas2005scholarly}, and in particular a Data Usage Index (hereafter referred to as DUI) from biodiversity informatics. 

\textbf{Background Literature}

The DUI was originally designed to measure the impact of institutional contributions to the Global Biodiversity Information Federation Database (GBIF) by using \emph{indicators} of how data were discovered and accessed (i.e. the number of downloads, page hits, files contributed by an institution, etc.) (Chavan and Ingwersen, 2011). In the DUI, indicators are extracted from user query logs and combined in simple, but unique ways to calculate impact factors\footnote{For more detail, see repository with tables and data  http://git.io/vJnPX}.\\

\textbf{Operationalized Study}

This intention of this study was to modify the  DUI's indicators from the context of a biodiversity database to the Research Data Archive (RDA) at NCAR, which provisions a number of different climate data and software resources (including
ICOADS). To do this we developed a set of use cases based on a researcher's discovery, selection, and mode of access to different data products from the RDA.\\

From the use cases we identified six key indicators that were captured
in the RDA's logs, as well as potentially informative relationships
between these indicators (see Table 1 below).\\
\\


\begin{table}[h]
\resizebox{\textwidth}{!}{%
\begin{tabular}{llll}
\textbf{Code}   & \textbf{Indicator}        & \textbf{Explanation}                                         &  \\
uu(ds)          & Unique Users              & Unique users that downloaded data during a time window       &  \\
n(ds)           & Number of Datasets        & Number of Datasets assigned DS number by RDA                 &  \\
f(ds)           & Files DS                  & Number of files in Dataset per time window                   &  \\
d(ds)           & Download Frequency        & Total number of files downloaded per time window             &  \\
hp(ds)          & Homepage Hits             & Home Page Hits of Data Set per time window                   &  \\
d(ds) /uu (ds)  & Download Density          & Average number of files downloaded per unique user           &  \\
d(ds) / f (ds)  & Usage Impact              & Total number of downloaded files over total files in dataset &  \\
d(ds) / hp(ds)  & Usage Balance             & Files downloaded by number of homepage hits per time window  &  \\
hp(ds) / f(ds)  & Interest impact           & Total homepage hits per number of files in dataset           &  \\
hp(ds) / uu(ds) & Secondary Interest Impact & Total Homepage hits over unique users                        &  \\
ss(ds) / d(ds)  & Subset Ratio              & Subset requests over total number files downloaded           & 
\end{tabular}
}
\end{table}\\

The completed use cases also demonstrated that two user types could be
identified based on how data were accessed:

\begin{itemize}
\item
  \textbf{Programmatic Users}: accessed or downloaded data through a
  command line tool (e.g. `-curl {[}\^{}curl{]} or ``wget''
  {[}\^{}wget{]}) or through a scripting language.
\item
  \textbf{Assisted Users:} access data via the graphical user interface,
  or by subset requests made through a separate tool developed by the
  RDA staff.
\end{itemize}

\textbf{Data Collected}

To test the effectiveness of the modified DUI three RDA data products were selected - including the most recent release of ICOADS (version 2.5). Indicators from the completed use-cases were then extracted from the user logs of each dataset over a sixteen month period.\\

From the index of each dataset we then further calculated two impact factors:\\

  \textbf{Usage Impact Factor}
  \begin{equation}
  (d(u) / f(u))/\sum_{1}^{n}{d}/\sum_{1}^{n}{f}
  \end{equation}
  where (u) is the given resource unit, (d)
  is the download frequency of users, (f) is the number of files
  downloaded per user session, and (n) is the total number of units (files
  available in the dataset) in the denominator.\\
  
  \textbf{Interest Impact Factor}\begin{equation}
  (hp(u) / f(u))/\sum_{1}^{n}{hp}/\sum_{1}^{n}{f}
  \end{equation} which is identical to UIF except
  download frequency (d) of users is replaced by the number of homepage
  hits (hp) a dataset receives.\\

This resulted in a composite index for each of the three datasets\footnote{ Found in http://git.io/vJnPX}.\\

\subsubsection{Phylomemetics}

\textbf{Methodological Intention}

In my ethnographic work participants often explained that ICOADS was reused in developing new climatology data products. However, operationalizing a tracking of the reuse of ICOADS through citation analysis proved limited by the inconsistency in how the project is acknowledged, and which research object is subsequently cited (see Weber, Mayernik, and Worley, 2014 for more details on this dilemma).

To better understand the history of how ICOADS has been reused, we conducted a study using techniques from the field of evolutionary biology (Page, 2009) and cultural analytics (Mace, 2005). This work produced a \emph{genealogy} of ICOADS. 

\textbf{Background Literature}

The inspiration for this work came from a set of studies that looked at the history of climate models by tracing their relatedness (See Figure 1.1); The figure on the left is from a study conducted by Edwards, who created a ``family tree'' of Atmospheric General Circulation Models (AGCM) from archival documents and oral histories \citeyearpar{edwards2010vast}. The figure on right is a dendrogram showing relatedness of climate models that participated in the 5th Climate Model Intercomparrison Project (CMIP), constructed using principal component analysis on documentation available for each model \citep{knutti2013climate}.\\

\begin{figure}
\includegraphics[width=4.5in, height=3in]{edknut}\\
\captionbelow{Edwards' genealogy and Knutti's principle components}
\end{figure}\\
\emph{Quantitative phylogeny of digital artifacts}

The approach is not without precedence: application of quantitative phylogenetic methods to linguistics and textual criticism is almost as old as phylogenetic methods themselves. In fact, Platnick and Cameron argue that similar methods were accepted as standard practice in both fields before biologists came to embrace them (1977). There has also  been a recent resurgence of interest in phylogenetic approaches to non-biological problems partly due to computational advances in bioinformatics, which not only allow for faster and easier computation, but also support the use of “molecular clocks” to root known speciation times (sometimes called divergence points) in ways that were previously difficult or impossible (Mace and Holden, 2005; Mace, Holden, and Shennan 2005).

To further emphasize the shift between biotic and abiotic studies of evolution, Howe and Windram coin the phrase “phylomemetics” in lieu of phylogenetics, “given the use of the word ‘meme’ to refer to a non-genetic principle that behaves in a genetic way” (2011). Though the differences between memetic and genetic evolution may have bearing on the models and algorithms used to study these processes, in this work, we use methods and software developed explicitly for phylogenetic work, and refer to our study as such. Previous work in linguistics and textual criticism also borrows heavily from biogeography in coupling an analysis of linguistic divergence – how dialects differ from one generation to the next – with analysis of human migration routes (Rexová, Frynta, & Zrzavý, 2003). Similarly, phylogenies of historical texts have been constructed for literary works such as Chaucer’s Canterbury Tales (Barbook, 1998) and Little Red Riding Hood (Tehrani, 2013). These approaches typically focus on finding divergence points to estimate when texts were altered, replicated or significantly changed by different authors or cultural groups.

Most immediately applicable to this study are phylogenetic applications by archaeologists and anthropologists who conceptualize artifacts as, ``complex systems comprising any number of parts that act in concert to produce a functional unit,'' in which the ``changes that occur over generations… are highly constrained, meaning that new structures and functions almost always arise through modification of existing structures and functions as opposed to arising de novo ''(O’Brien, Lymen & Darwent, 2002). This “system” view of artifacts is particularly applicable to digital objects, which may also be viewed as complex systems comprising any number of interactions between layers of information content and representation (Wickett et al., 2012). Bit sequences, encoded information content and information systems work together to produce a functional unit, and the changes that occur over generations of use are constrained by the practices and sociotechnical contexts of the groups using them.

\emph{Qualitative phylogeny \& the biography of artifacts}

Just as quantitative phylogenetics has a long history of application to the study of material and textual artifacts, so does the qualitative study of evolution as cultural diffusion. Anthropologists, economists and sociologists have each noted the importance of tracking the social ``markings'' of mundane objects that personalize, and make a given object individual to a period of time (Appadurai, 1986). In this vein, Igor Kopytoff proposed that tracking the movement of an artifact between different contexts of use required a biographical approach that could see ``…a culturally constructed entity, endowed with culturally specific meanings, and classified and reclassified into culturally constituted categories ''(1986, p. 68).

Similarly, Williams and Pollock describe a technique called the biography of artifacts, which takes a popular software platform as a unit of analysis (e.g. Microsoft Sharepoint), and attempts to trace the way it was modified, changed, and socially shaped by studying the different contexts in which it was used. The ambition of the biography of artifacts approach is to show the evolution of similar technical artifacts in different social contexts, including their adaptability (or evolutionary fitness) across diverse software ecosystems (Williams and Pollock, 2009). Dosi and Nelson similarly relate evolutionary concepts from biology to behavioral economics and organizational theory (2003). In doing so, they relate technological change within private firms to environmental pressures in an ecology, effectively equating these externalities as selection mechanisms for evolutionary processes. Dosi and Nelson attempt to study links between organizational economics and evolutionary biology through qualitative observations of the practices, policies and technological adaptations of a firm.

A quantitative phylogenetic approach can add another dimension to each of these types of analysis. Though it cannot answer the same types of questions about how context or culture has shaped technical artifacts as used in different social settings, it can more rigidly answer when and to what extent an artifact has changed between cultures, and visualize those changes over time.

\textbf{Operationalized Study}

Our work on an ICOADS genealogy somewhat diverges from these previous studies in that we aren't making an evolutionary metaphor or analogy, we are directly borrowing techniques and software from phylogenetics (Page, 2009) in trying to quantify and subsequently visualize the relatedness of climate datasets. Part of the ambition in taking a biological approach to studying material aspects of ICOADS was to leverage the explanations and theories that this field offers for cooperation (Novak, 2006). If ICOADS data products did have a traceable evolution then it might be possible to use concepts like kin selection, fitness, group selection, and reciprocity (direct, indirect and networked) to shed light on how data have been reused, shared, developed, or refined. We were not trying to create a direct mapping between biotic reproduction and abiotic data reuse, but we were taking seriously the ecological metaphor that is often invoked in discussing the complexity of software / data intensive enterprises (Weber, Thomer, and Twidale, 2013).\\

\textbf{Data Collected}

To trace the genealogy of ICOADS, we searched for and harvested metadata records from NASA's Global Change Master Directory (GCMD) using the ISO 19115 standard. We used queries related to ICOADS such as ``International Comprehensive Ocean and Atmosphere Dataset'' or ``ICOADS'' and ``COADS'' to locate as many related ICOADS records as possible. In total, we discovered 99 records, of which only 23 represented different versions or subsets of the ICOADS project. The remaining (n =76 ) records represented derivatives or offspring of ICOADS.\\

Next, we identified properties (or characteristics) of climate datasets which we thought would be unique and important to signifying change from one generation of ICOADS data to the next. This included things like  format, encoding characteristics, or even the parameters of the data set (e.g. Sea-Surface Temperature, Sea-Level Pressure, Wind Direction, Swell Height, etc.). We then extracted fields containing these properties from the metadata records. The fields harvested included: Entry Title, Entry ID, Summary, Geographic Coverage, Start Date, End Date, Geographic Resolution, Temporal Resolution, Scientific Keywords (often dataset parameters), Geographic Keywords, Sources (platform of data collection), and Instruments.\\

Next, we converted each field into binary codes for ``presence'' or ``absence'' of individual keywords (Figure 1.2). In some cases we coded additional ``presence'' or ``absence'' of characters based on the free text summaries of the records (for instance, in some cases, resolution was stated in the free text ``Summary'' field but not the ``Geographic Resolution'' field).\\

\begin{figure}
\includegraphics[width=4in, height=2.5in]{ICOADS_Coding}\\
\end{figure}
\textbf{Outcomes}\\ 

With the coded data, we then produced a maximum likelihood (ML) tree (Felsenstein, 1981), by utilizing a statistical model specifically designed for use with morphological, or presence/absence data \. Again, the assumption that we were making in this process is that significant properties or characteristics of the metadata records could be clustered - such that data products that shared similar traits could be grouped together in the same way that similar species are grouped together in a phylogenetic analysis (for a complete discussion of this work see (Thomer and Weber, 2014; Weber, Thomer, and Worley, 2014.) \\

Our preliminary efforts were successful in creating a tree that chronologically resembled the release schedule of ICOADS (e.g. the root node was the earliest release of ICOADS and furthest nodes were latest
releases). To evaluate the accuracy of this work, we presented a poster at the Fourth JCOMM Workshop on Advances in Marine Climatology, which is a major event in the ICOADS community. We asked participants to annotate our tree, and give us feedback on:\\

\begin{enumerate}
\item
  The accuracy of the clustering of related datasets, and
\item
  Datasets related to or derived by ICOADS that were missing from our
  tree.
\end{enumerate}

\begin{figure}
\includegraphics[width=4in, height=2.5in]{gen}\\
\captionbelow{Poster presented at CLIMAR IV meeting, with participants annotations.}
\end{figure}\\

The feedback from the ICOADS community verified the accuracy of our work. In particular, many scientists recognized the data products that were clustered based on related features, such as derivative sea-surface temperature (SST) or arctic sea ice data products. However, a number of curators from other data repositories were able to identify missing datasets, and logical relationships between clusters that were not well represented by the tree structure (See Figure 1.3 for annotations). We then manually added these datasets and their coded characteristics, and re-produced a maximum likelihood tree.\\

\subsection{Interpretive Interactionism}

The final method of data collection was a series of interviews with members of the ICOADS Steering Committee (ISC), a governance mechanism adopted by the community late in my research process (year three). In total, I interviewed six of the seven ISC members.  

\subsubsection{Semi-structured Interviews}

\textbf{Methodological Intention}

The approach taken for conducting ISC interviews differed from ethnographic work for two specific reasons: 
\begin{enumerate}
\item The ethnographic interviews were informal, and opportunistic. Using a semi-structured format would allow me to ask similar questions of different individuals and compare their perspectives. 

\item With the permission of my participants, these interactions were recorded, partially transcribed, and analyzed for themes - allowing me to conduct further analysis of key ICOADS members and their experiences in governance roles.  
\end{enumerate}

\textbf{Background Literature}

The interpretive interactionist approach to qualitative research has two main objectives : 
\begin{enumerate}
\item To capture the thoughts and ideas of an individual in their own words, recognizing that language is highly personal and that individuals place meaning on the words that they choose, the descriptive metaphors that they employ, and the narrative structures that they use in reference to the world around them.     
\item To abstract from individual level analysis in trying to understand how personal meanings are shaped by interactions with people, and technologies in real-world contexts. (e.g. Prust, 1996)
\end{enumerate}

\textbf{Operationalized Study}

I interviewed six of the seven ISC members via teleconference. I used a set of standard questions about ICOADS governance that I asked every participant, but modified these based on the number of years they had been involved with ICOADS, and their professional background. The conversations lasted anywhere from 38 to 80 minutes. I began each conversation with an overview of my research goals, and by obtaining permission to record the conversation. I had already presented research in front of these individuals, and was attending regular ISC teleconferences, so they were already quite familiar with me and my work. 

\textbf{Data Collected}

Each interview was recorded, and partially transcribed for analysis. I also created a profile of each individual which included their relevant ICOADS publications, a CV, and a biographical sketch based on their personal websites or my previous interaction with them as participants of my ethnographic study.\\ 

\section{Analysis of Case Study Data}

The data collected through various methods in this dissertation has been aggregated in a research repository which hosts the software, protocols, interview tapes, and field memos written during ethnographic data collection. With the exception of taped interviews and field notes, which may reveal personally identifiable information about participants, these materials are available at: 

After aggregating the data in a repository, I then used the higher level categories of the knowledge commons framework (Background Environment, Attributes, Governance, and Patterns \& Outcomes) to code the various data. This consisted of re-reading material, listening to transcripts and assigning pieces of data to particular categories from the framework. 

In coding this data, I attempted to answer questions posed by each category. For instance, using the \emph{Background Environment} category I tried to answer questions posed by Madisson, Frischman and Strandburg (2014, such as ``What is the background context (legal, cultural, etc.) of this particular commons?''; ``What is the “default” status of the resources involved in the commons (patented,copyrighted, open, or other)?''; as well as questions guiding my own case study, such as``How was this default established?''; ``How well do participants understand this default?''; and ``Has the default changed?'' I then iterated through each category of the framework until I felt that I could reliably answer each set of questions. In some cases I returned to participants to ask follow up questions and clarify their statements or interpretation of events that I observed. Below, I further describe these categories and their application for my analysis. 

I then generated a set of secondary categories, and third level categories and repeated this process until I had exhausted all of the data sources. A similar iterative and approach was taken by Strandburg et al. who gathered quantitative and qualitative data while studying resource sharing in biomedical clinics (2014, p. 159-160), as well as Contreras' study of the genome commons (2014a).

\subsection{Analytical Frameworks Used for Organizing Analysis}

\subsubsection{IAD}

The Institutional Analysis and Development (IAD) framework was designed as method-agnostic tool that could “simplify the analytical task confronting anyone trying to understand institutions in their full complexity.” (McGinnis, 2011). The settings for much of this work were institutions that operated with a mix of governance structures, and had a need for longevity or sustainability of a single shared resource. Various versions of this framework can be found in the work of Kiser and Ostrom (1982), Ostrom (1990), Ostrom, Gardner, and Walker (1994), McGinnis (2000), and Ostrom (1998, 2005, 2007b, 2010, 2011a). This was to be achieved by integrating data collected by sociologists, lawyers, politicians, economists, and political scientists in trying to understand ``how institutions affect the incentives confronting individuals and their resultant behavior'' (Ostrom, 2005 p. 8). Below, I sketch out the most basic features of the framework.

The goals of the IAD framework can be broken down into it's two parts: 1. Analysis, and 2. Development. 

The analytical goals of the framework are to guide a systematic study of formal and informal institutions, allowing for similar variables to be gathered and compared across cases. 

The development goals are both diagnostic and design based; through design the IAD can account for how institutions are established, maintained, and transformed; and through diagnosis can identify missing institutions - or components of an institution - that are the a source of dysfunctional performance.

The IAD framework is divided into three levels of analysis for studying “underlying factors” of an intuition's success or failure: exogenous variables, action arenas, and outcomes and evaluations.

\begin{enumerate}
\item Exogenous variables include biophysical attributes of the shared resources, community attributes, and structural governance policies (formal or informal).

Exogenous variables are also considered the “input” to a complex system. In an overview of the framework Frischmann gives a simple example of this group, from a lobster fishing grounds 

\begin{quote}
``...attributes might include the relevant biological characteristics of lobsters, such as the rates at which they age and reproduce; attributes of the community of fishermen, such as the proximity in which they live to others, the existence of familial relationships, and the skill sets needed for lobster fishing; and the rules — explicit or informal — that govern fishing.'' (Frischmann, 2013 p . 8). 
\end{quote}

The community attributes, using the same lobstering example above, could include the owners of boats and docks, the distance fishers live from one another, the 

The most crucial aspect of this level is that while individual attributes or variables are identifiable and observable, they need not be considered fully decomposable (Ostrom, 2005). This means that the salient features of a commons are often thought to be “emergent properties” where the mutual constitution of people and objects create phenomena which are indivisible. The use of multiple research methods for the same institutional setting are often justified on the grounds of needing to understand these emergent properties (Ostrom, 2005).

\item An Action Arena – space where exogenous variables and actors interact, cooperate, and conflicts emerge.

This level of the IAD studies the stakeholders and the resources of the commons – as inventoried by the Exogenous Variables – in a particular period of interaction such as the harvest of a shared common field, or the fishing season of a particular coastal community. Using the understanding gained from cataloging the Exogenous variables, the analyst observes this process and then identifies tangible outcomes, objective results, and the way that conflicts from the past, or the present are voiced, contested, settled or deferred. This analysis can take place over long periods of time or short site visits in a single time frame. These observations can also be supplemented with statistical analysis of quantitative data, such as the crop yields of a commons studied over a certain period of time, or even through agent-based models or multi-modal network analysis used to simulate these interactions.

\item Outcomes and Evaluation – the results and feedback of action situations

 In the final stage of the framework an institution can be evaluated based on outcomes of the action arena; that is, how well an institution did or did not solve a collective action problem. Criteria for evaluation will vary by institutional setting, but usually includes calculation of costs, such as (1) information costs (2) coordination costs; and, (3) strategic costs (Ostrom et al. 1993). At a higher level of analysis, such as that aimed at developing policy, the sustainability of the shared resource can be analyzed in terms of (1) efficiency in access and use (2) equity of distributed wealth or costs, (3) accountability for resource management, or (4) adaptability of the institution in light of these outcomes (Ostrom et al. 1993; Imperial and Yandle, 2007).

 Over time the use of the IAD to study socioecological systems has proven useful in comparing the outcomes of real world scenarios with those presumed to outcomes addressed through public policy. 
\end{enumerate}

\subsubsection{Knowledge Commons Framework}

Initially described in 2010 as the ``constructed cultural commons, ''(Madison, Frischman and Strandburg, 2010) the knowledge commons is both a concept, and a framework to describe institutional arrangements that govern the community provisioning and production of resource systems made up of ``cultural ''artifacts - ranging from complex sociotechnical systems to simple cassette tape trading communities. The ambition of adapting the IAD to cultural sphere is to systematize investigations, provide guidance for a more rigorous set of evaluation criteria - especially as it relates to the matching of observation with existing theories and models-  and to enable the integration of findings cases across different domains of knowledge production. (2014)   

The proposed framework achieves this through three major innovations with the IAD:

\begin{enumerate}
\item It emphasizes among distributed actors, and the constructed nature of the resources themselves (purposefully built objects as opposed to natural resources)
\item Where the IAD separates “outcomes” (level 2) and “patterns of interactions” (level 3) the constructed cultural commons framework treats these as iterative, mutually constitutive processes. 
\item As a result of the more complex relationships between resources, participants, and governance structures – the “relevant” attributes of each may not neatly divide into categories. In short, this is the acknowledgment of a “mutually constitutive” socio-technical perspective.
\end{enumerate}

The quintessential case for the Knowledge Commons are free and open source software projects such as Apache or Linux where there is a mix of proprietary rights, resource producers, users, and on-going tensions amongst these participants in the governing both the project as well as versions or distributions (Schweick and English, 2012). However, a number of diverse case studies beyond open source software have emerged using this framework, including the study of patent pools amongst technology firms (Madison, 2012), the Associated Press (Muarry, 2014), Wikipedia (Madison, Frischmann, and Strandburg, 2014), and notably, scientific infrastructures supporting large-scale collaboration in biology (Contreras, 2014, astronomy (Madisson, 2014) and biomedicine (Strandburg, Cui and Frischmann, 2014). These latter studies of knowledge commons in scientific collaboration further collapse a distinction made in the original IAD framework between the ``patterns of interaction'' and outcomes following an Action Arena. The argument for this innovation is that patterns of interaction generated by the formal and informal rule systmes are often inseperable from the outcomes it produces, ``How people interact with rules, resources, and each other, in other words, is itself an outcome that is inextricably linked with and determinative of the form and content of the knowledge or informational output of the commons. ''(Frischmann, Madison, and Strandburg, 2014)

In using the Knowledge Commons framework to organize my data analysis, I make two important modifications to previous works: 

\begin{enumerate}
\item The first section titled `'Background Environment' is expanded in scope to include historic antecedents to the ICOADS project. Doing so makes the overall narrative more intelligible for a reader unfamiliar with climate science, marine climatology, or marine surface data. It also allows my analysis to demonstrate ways that historical events, and international policies governing meteorology impact the arrangement of contemporary institutions for data production and exchange. 

\item I describe regimes that govern an ``Action Arenas'' rather than focusing on specific sets of events. Through the different releases of ICOADS data products, I demonstrate a co-evolution of norms and rules rather than a causal relationship between, for instance, behaviors and sanctions. This also allows for an investigation of governance issues over a longer period of time, and overall I believe this lends to a more thorough analysis of ICOADS governance. To achieve this, I draw upon a systematic approach to coding variables related to governance of a shared resource system. I describe this process in detail below.  
\end{enumerate}

\subsection{Criteria for Interpreting Results \& Answering Research Questions}

To review - the Knowledge Commons Framework (KCF) provides four high level categories - Background Environment, Attributes, Action Arenas, and Patterns \& Outcomes - for analyzing data gathered through a case study design. The ambition of this work is provide a way to systematically gather data, and compare findings from various different domains of knowledge production, looking for similarities and differences in the ways that institutional arrangements are able to overcome social dilemmas in sustaining a commons. As the designers themselves note, ``...there is much work to refine and systematize this approach'' (Frischman, Madison, and Strandburg, 2014). One way that the IAD has been systematized in the domain of socioecoligcal systems is through the design of project specific databases, where case study data can be uploaded and compared by a large number of collaborators. Out of the necessity of creating a database structure, the process is systematized, creating variables, properties, and resulting definitions for how to properly code case study data. 

I draw on this work to create a set of definitions, variables and their properties for the governance of ICOADS. In particular, I draw on the Social-Ecological Systems Meta-Analysis Database (SESMAD) project's definition of governance variables (Cox et al, 2014). Below I offer two tables: 

\begin{enumerate}
\item Lists the variables to be coded for ICOADS governance.
\item Defines the properties of each variable. 
\end{enumerate}

\begin{figure}
\includegraphics[width=3.5in, height=2.5in]{vars}\\
\captionbelow{Lists the variables to be coded for ICOADS governance.}
\end{figure}

\begin{figure}
\includegraphics[width=3.5in, height=2.5in]{props}\\
\captionbelow{Defines the properties of each variable.}
\end{figure}

Each variable is further defined by the properties for that particular variable\footnote{This information can be found at http://git.io/vJnPX}.  

Properties consistent with each variable are defined below. Not all properties are required for a variable to be coded correctly. For instance, domain and range categories may be unnecessary for some variables. 

\subsection{Validity: External and Internal}

\textbf{External}

External validity was achieved through my presentation and sharing of findings from this work with members of the ICOADS community, and in particular two key informants who have regularly read drafts of this work, made suggestions for improvement, and clarified my writing for historical accuracy. Additionally, my key informant worked with me in coding of variables related to governance. This included exchanges about the proper definition of a variable, as well as what values the variables should contain. 

\textbf{Internal}

Internal validity is addressed through triangulation of data collection, and by comparing the results of my work to case studies that have used the same framework in different settings - including the genome commons in biology (Contreras, 2014), the Urea Cycle Research Network in biomedicine (Strandburg, Cui, and Frischmann, 2014), and the Galaxy Zoo project in Astronomy (Madison, 2014). Synthesizing results across these diverse domains of knowledge production is what allows me to address the stated research questions of the dissertation: 

\begin{enumerate}
\item What are the effective institutional arrangements (governance) for sustainable community based science infrastructures? and, 
\item How and why do they differ between domains of knowledge production? 
\end{enumerate}

Answers to these questions are presented in Chapter 5. 

\subsection{Limitations}
The case study design described above has a number of limitations, including;
\begin{itemize}
\item This is a single case and the results will be difficult to generalize from, especially for large concepts like sociotechnical systems, or knowledge commons.
\item The informetric studies are heuristic, and imperfect from traditional Informaiton Science standards. Instead of judging their signicance with p-values, I would argue that they were useful tools for collaboration with my participants.
\item A focus on governance is only one component of a sustainability study. This will limit the ability of this work to provide comrephenisve guidance on sociotechnical systems sustainability. 
\end{itemize}

\section{Summary}

In this Chapter, I have described the range of empirical research methods used to collect and analyze data; including the case study design, process of data collection, as well as the analysis and organization of the results into a modified knowledge commons framework, and the coding of variables related to evolution of ICOADS governance which draws upon the SESMAD framework. Results from this analysis will be presented in Chapter 4 (following). In Chapter 5, I use these findings to answer the stated research questions of the case study. I then compare these results to previously completed case studies, and answer the stated research questions of the dissertation overall. 

