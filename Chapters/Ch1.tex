\chapter{}

\section{Between Markets and States: The Commons}

Contemporary research and deveopment activities in basic science are increasingly cooperative () and collaborative ().  Increases in collaborative aspects of science can be seen in growing number of authorship on peer-reviewed articles (), organizational diversity in research funding applications (), and in the increased funding of research insitutes which tackle grand challenge science problems by combining expertise in two or more research areas (). Cooperative, aspects of contemporary research and development can be seen in the co-producitoin of infrastructures to support collaborative work, as well as the  scientific collaborations being managed as common property.  

These collaborartive and cooperative features of contemporary research and development activities also come at a time of intense competition for research funding (), and overall decrease in the allocation of federal monies to support basic sciene (). 

Combined, this environment poses a serious threat to cooperative institutions necessary to tackle grand challenge science problems. 

How does the collaboration and cooperation of contemproary science get sustained within a fiercly competetive environment for funding?

One way to answer this question is to look at the models of governance that are used in collaboratively produced, and cooperatively managed shared resources. 

Two tradtional models of organizing research and development are the market and the state: 

\begin{enumerate}
\item The market, where intellectual property rights and patents create exclusivity for private ownership, or 
\item The state, which regulate and intervenes in markets by subsiding innovation, and consequently creating open-access resource systems.
\end{enumerate}

In science and technology policy the two approaches are often referred to as a tension between the commercialization of science on the one hand, and the ``normative'' regulation of science on the other (Merton, 1963; Eisenberg 1989; Rai 1999; Reichman & Uhlir 2003; Mirowski, 2012; Madison, 2014). A third approach, which combines aspects of both states and marktets, is the commons - a term informally used to refer to any area of economic activity where there is ``“freedom-to-operate under symmetric constraints, available to an open, or undefined, class of users'' (Benkler, 2014). In practice, the commons resemble institutional arrangements predicated on self-organization and community governance at local levels; and broad cooperation, shared norms, and collective rule-making procedures at larger-scales (Ostrom, 1990). The commons offer an institutional alternative to markets and states, but are not fully separate from government subsidy, nor without answer to contemporary capital-based marketplace models. 

The commons have proven to be an institutional arrangement especially effective at sustaining socioecological systems that couple people, technologies, and environmental issues of a complex nature (Dietz et al., 2010). Examples of these success can be found at a number of different scales, ranging from effective international treaties negotiated to curb harmful atmospheric emission, such as the Montreal Protocol (Epstein et al., 2014), to the preservation of irrigation systems throughout local municipalities, such as those found in Nepal (Lam, 1998).  

Like governments and marketplaces, the commons has its own set of costs and benefits. The work of political economists like Elinor Ostrom shows us that governance models that work in one socioecological setting, may prove ineffective in another (Ostrom, 2010). The overaching goal of this dissertation is to understand which governance models work best, in which sociotechnical settings. In particular, I focus on research and develop settings in contemporary science. 

This research project will contribute to a better understanding of the institutional arrangements that enable long-term scientific cooperation thorugh the development of a systematic approach to sociotechnical systems sustainability. This will be achieved in two ways: 
\begin{enumerate}
\item The use of an emerging Knowledge Commons Framework (KCF) to conduct a case study of the International Comprehensive Ocean and Atmosphere Dataset (ICOADS), a project in marine climatology that has successfully sustained a cooperative model of knowledge production for over thirty years. I compare the results of this work with three previously completed case studies of genomics, astronomy, and biomedicine.

\item By adapting and modifying a protocol from the Social-Ecological Systems Meta-Analysis Database (SESMAD) for systematically coding variables related to different components of the Knowledge Commons Framework. My adaptation of this protocol focuses specifically on governance. I show how standard coding of relevant system state variables\footnote{in other words, the characteristics of system at a particular period of time (Walker et al, 2005)} allows for meaningful inter-case comparison with data collected about ICOADS, and has the potential to be used in diverse sociotechnical settings. 
\end{enumerate}

\subsection{The Struggle to Govern the Commons}

The commons is a general term that can refer to a resource management approach, as well as the the shared resource system itself (Hess and Ostrom, 2005). This dual-distinction is evident in the etymology of the word ``commons'';  the Latin root being ``\emph{communis}, which signifies something held in common by a group, but also a user community bound by responsibilities as well as rights ''(Disco and Kranarkis 2013, p. 14). Until very recently the study of how commons function and evolve has focused almost exclusively on natural systems; examples include grazing pastures in Switzerland, aquifers in Mexico, and forests in Western Africa. Previous scholarship in this domain, including the ``tragedy of the commons'' (Hardin, 1968), had created an artificial distinction between private ownership and government regulation. The refutation of this commonly accepted wisdom was pioneered by Elinor Ostrom, who focused on how institutions for collective action develop rules to govern shared resource systems, and what role these institutional arrangements have in creating cooperative, sustainable commons (1990). Ostrom and colleagues used a variety of methodological approaches, including laboratory work, field studies, and surveys, to show that state or market models rarely correspond with the way that successful shared resource systems are governed in the real word. Instead, they often found that users, producers, and provisioners of a natural resource system created self-governing systems which outperformed state and marketplace models. 

A salient example of the unintuitive results of this work comes from the fisheries literature where landings regulations (total amounts of a catch measured in millions of kilograms) are managed under different governance regimes. In the same geographic locations facing the same biophysical conditions, certain fisheries collapse, like the \emph{gadoids} managed under State regulation, while other fishiered thrive, such as \emph{lobsters} managed under a self-organized commons regime.\\  

\begin{figure}
\includegraphics[width=4in, height=2.5in]{Fisheries}

\captionbelow{Two landings in millions of kilograms over time. Dotted lines show lobster populations thriving, and Gadoids collapse, both in the same location, but under different governance regimes.}\\
\end{figure}

To put it plainly, governance matters. A large body of research on the commons shows us that the effectiveness of rules, sanctions, and policy instruments depends greatly on the context in which they are deployed; gadoids and lobsters may have different reproductive cycles which require different harvesting rules; technologies for trawling gadoids grounds may have advanced more rapidly than the caging techniques used in the lobster industry; or, it may be that a chemical change in the feeding grounds impacted the two species differently. Without a systematic and controlled way to study these different variables, the evaluation of policy effectiveness is thin, and likely incomplete. In developing a comprehensive framework for studying these types of related socioecological systems, commons scholars like Ostrom have been able to generate a deep understanding of which types of social dilemmas require which kinds of governance systems (Acheson, 2012). An empirical, systematic approach greatly increases the chances that a particular policy or a particular design intervention will be effective for sustaining cooperative arrangements.

\section{A Systematic Approach to Governing Knowledge Commons}

The NIH BD2K program may see the commons as a solution to all of the social dilemmas faced by producing and provisioning a research infrastructure, but it's unclear how effective this model will be, for whom will it be effective, and for how long. What's needed is a systematic approach to science commons in sociotechnical systems, just as a systematic approach to natural commons has been taken for socioecological systems. Just as what works in one socioecological context may not work in another, a one-to-one mapping will not be perfect between the natural resource systems of Ostrom's study, and the purposefully designed and engineered systems that are the subject of this dissertation. Sociotechnical systems are subject to a variety of different social dilemmas than the natural commons; including the coupling of social and technical phenomena which are not well suited for the types of frameworks that have been developed to study ``biophysical characteristics'' of the natural commons. 

An emerging concept in the commons literature is the notion of ``culturally constructed'' resources managed as ``knowledge commons.''(Madison, Frischmann, and Strandburg, 2010). Where natural commons are the institutions, resource sets, and interactions governed in a socioecological system, the knowledge commons are the cooperative arrangements that share, curate, produce, provision, and sustain informational resources within a sociotechnical system. Governance in the knowledge commons is, similarly, critical to sustainable knowledge production. As Frischmann, Madisson and Standburg, explain, ``The nested, multi-tiered character of productive and sustainable knowledge and information systems and the diversity of attributes that contribute to successful governance regimes are key to understanding knowledge commons as mechanisms for knowledge production, collection, curation, and distribution in the context of modern information and IP law regimes.'' (2014)

Recognizing that a binary distinction between markets and states is an oversimplification of the organizational models of governance in contemporary science, the commons begin to appear ubiquitous in contemporary research and development settings: knowledge commons are the shared libraries of open-source software that run high-performance computing centers; knowledge commons are the research data archives which enable whole earth climate simulations on a petabyte scale; and, knowledge commons are the instrumentation at the center of astronomical observatories, such as the aptly named ``Very Large Telescope'' in Chile. 

But, although knowledge commons appear to be a pervasive phenomena in science and technology settings there is little empirical understanding of how these types of informal institutional arrangements are successful over time. There is a lack of understanding about which governance arrangements enable sustainable knowledge production in which contexts, And, there is yet to emerge a systematic way of collecting data to compare across these different contexts. This dissertation therefore pursues two broad research questions:\\

\textbf{RQ 1}. What are the effective institutional arrangements (governance) for sustainable scientific knowledge commons?\\

\textbf{RQ 2}. How do these arrangements differ between domains of knowledge production?\\

These questions are answered by conducting a case study of a knowledge commons in the domain of climate science. By adapting two frameworks used to study socioecological systems to analyze data collected in the case study, I attempt to show the benefit of this approach as applied to the sustainability of knowledge commons in science. I achieve this by comparing these case study results to three previously completed case studies, and synthesizing the findings across all four cases. 

Before introducing the case study subject, the International Comprehensive Ocean and Atmosphere Dataset (ICOADS), I describe a set of basic concepts that will be used throughout this document. 
 

\section{Research Context and Rationale}

I turn next to an introduction of the subject of my case study, and the major subject of data collected for this dissertation. A thorough discussion of the background environment for ICOADS appears in Chapter 4. 

\subsubsection{Context}

The International Comprehensive Ocean and Atmosphere Dataset (ICOADS) is a cooperative project that curates, develops and distributes quality controlled data, metadata, historical documentation, and software to the climate science community. The project, originally named COADS, was initiated in 1981 by researchers at the Earth System Research Laboratory (ESRL), the National Climatic Data Center (NCDC), and the National Center for Atmospheric Research (NCAR). Over time the project grew to include international collaborators and the name was changed to the ``International COADS'' in order to reflect the contributions of organizations like the World Meteorological Organisation (WMO), the Intergovernmental Oceanographic Commission (IOC), and the Technical Commission for Oceanography and Marine Meteorology (JCOMM).\\

Contemporary data curated by ICOADS come from a variety of sources, including the Global Telecommunications System (GTS), in-situ measurements taken by sea-faring vessels, earth observing satellites, and both drifting and moored buoys. Historical data come from an effort in the early 1980s to aggregate existing marine data records from maritime archives around the world, as well as a continual stream of historical records that have been rediscovered and newly digitized; this now includes the use of crowd-sourcing efforts to transcribe weather recordings taken by military, shipping, and whaling voyages from the 17th, 18th, and 19th century. These early records are significant culturally and historically, as ``Sailors were among the first to systematically record the weather because the states of ocean and atmosphere controlled their progress and survival (Woodruff et al., 1986 citing Quayle, 1977)''.\\

The labor-intensive process of taking heterogeneous records from different observing platforms, and uniformly processing and integrating the data into a larger set of historical observations is the major value of ICOADS ongoing work. This includes the preservation of provenance metadata that is recorded for each individual record, allowing for researchers to trace backwards in time to verify whether the source of a climate anomaly is genuine or the result of a data processing
error.

Free and open access to ICOADS has helped it become recognized by the climate community as the ``most complete and heterogeneous collection of surface marine data in existence'' (Woodruff et al., 2011). ICOADS data have been used extensively in international climate assessments such as the IPCC AR4 and AR5 reports, as well as reanalysis projects that combine historical data with contemporary weather observations to create authoritative datasets for the climate modeling community (Kalnay et al., 1996). 

The success and widespread use of ICOADS has not resulted in greater stability for the funding of the project. This is due in part to the difficulty in calculating the research impact of many different ICAODS products (Weber et al., 2014), as well as an overall decrease in federal funding for the maintenance of research infrastructures (Berman and Cerf, 2013). The politicization of climate related research has also impacted ICOADS funding in recent years as congressional pressure to defund ``climate research'' continues to mount. In the winter of 2012, this became a major issue for the sustainability of the project, as NOAA announced that:

\begin{quote}
``For budgetary reasons, stemming from pending large cuts at the NOAA Climate Program Office (CPO), ESRL Directors have determined that it is no longer feasible for its Physical Science Division (PSD) to continue supporting any further ICOADS work effective immediately'' (Lawrimore, 2012)
\end{quote}

In response, project partners at the National Center for Atmospheric Research (NCAR), the UK Meteorological Office, and the Deutscher Wetterdienst (German Meteorological Service) signed a memorandum of understanding to continue contributing to the maintenance of the project, and its various resources. The research of this dissertation was conducted beginning at the point that signatures for a memorandum of understanding were signed. 

\subsubsection{Rationale}

For many of the reasons described above, ICOADS offers a unique case study of sustainability in a knowledge commons. These include, but are not limited to:

\begin{itemize}
\item 
Project partners are diverse in terms of their organizational affiliations, and expertise. As such, ICOADS is nested within a number of overlapping governance structures.
\item
Climate science is a unique domain of knowledge production in that it requires a broad scheme of cooperation in order to generate verifiable research results. And further, these results are some of the most scrutinized, and politically charged forms of knowledge production. 
\item
The resource sets provisioned by ICOADS have persisted over a thirty year period that has seen a number of fluctuations in funding, the politicization of the subject matter, and rapid technological change. ICOADS therefore presents an opportunity to better understand how diverse resource sets, consisting of software, data, human expertise, and computational infrastructures are sustained over time. 
\end{itemize}

\textbf{Structure of Document}

In Chapter 2, literature relevant to the emergence of the commons, and scientific cooperation is reviewed. Design choices related to the case study of ICOADS are described, including how sustainability, and governance will be empirically approached. 

Chapter 3 describes data collection methods, and the approach to organizing and analyzing the results of the ICOADS case study. 

Chapter 4 presents findings from the case study research, which is organized by the Knowledge Commons Framework (Frischmann, Madison, and Strandburg, 2014). 

Chapter 5 further analyzes the ICOADS case study by adapting portions of the Social-Ecological Systems Meta-Analysis Database (SESMAD) schema (Cox et al., 2014).  

In doing so, I  compare results from the ICOADS case study with previous case studies completed using the Knowledge Commons Framework. 

Chapter 6 states the implications of these results for policy, practice, and theory of the commons; and for the governance of sociotechnical systems more generally.



