\chapter{ICOADS as a Knowledge Commons}

\section{Introduction}

\emph{This chapter presents an analysis of ICOADS governance using the knowledge commons framework. I begin with an overview of concepts from previous chapters. I then use data collected from the case study of ICOADS to fill in the major framework categories - Background Environment, Attributes, Governance \& Rules-in-Use, and Outcomes. Further synthesis of these findings appear in Chapter Five.}

\section{Knowledge Commons Framework: Overview}

To briefly summarize the discussion from previous chapters: 

\begin{enumerate}
\item In socioecological systems sustainability, the commons has proven to be an effective institutional arrangement for sharing existing resource sets amongst diverse peoples. In sociotechnical systems, commons governance can also be used in creating, modifying, and remixing new resources; the latter is often described as a knowledge commons.

\item Resource sets making up a knowledge commons tend to resemble public goods - meaning they are non-rivalrus and non-exclusive. Governance in knowledge commons often rely upon loose interpretations of jurisdictional law (patents, copyright, etc) or unspoken norms and rules that are learned over time.

\item Both socioecological and sociotechnical systems overcome collective action dilemmas related to sustainability (e.g. free riders, tragedy of the commons, etc) through these rules and norms. However, as a result of differences in the way the two types of systems are produced, and provisioned they often face different types of social dilemmas. Therefore governance effective in one domain may, or may not be effective in another. Categorizing and identifying the unique social dilemmas of sociotechnical commons is a first step in systematizing the study of these institutional arrangements. The following case study is focused largely on understanding how governance impacts sustainability. 

\item Unlike socioecological systems, sociotechnical systems are intentionally created. Normative issues related to the planning, design, and deployment of sociotechnical systems is critical to understanding their current and past states. History, narrative, and shared culture are recognized as important factors for studying the evolution of knowledge commons, but are difficult to access via observation-based research methods. Therefore, archival resources, and historical approaches are emphasized for developing a comprehensive case study.

\end{enumerate} 

The Knowledge Commons Framework (KCF) is a recent attempt to use socioecological frameworks for studying sociotechnical systems. In part, the KCF is meant to help reduce the complexity of studying institutional arrangements in shared resource systems (Frischmann, Madisson, and Strandburg, 2014). KCF makes two major modifications to frameworks used to study sustainability in socioecological systems, notably the Institutional Analysis and Development (IAD) framework:

\begin{enumerate}
\item Interactions between the attributes of the community, it's rules, and the constructed nature of the resources being pooled are seen as mutually constitutive. This collapses a distnction made by the IAD between these three components. At the crux of this modification is a difference between systems which are designed and built versus those that naturally occur.\footnote{This point will be further emphasized in the following case study.}

\item The KCF also collapses a distinction between ``patterns of interactions'' and the outcomes that follow. This is to recognize that these interactions are not finite, but instead iterate through cycles of production, consumption, use, collapse, re-design, and eventually reproduction.
\end{enumerate}

\begin{figure}
\includegraphics[width=4in, height=2.5in]{Images/KnowledgeCommonsFramework.jpg}\\
\captionbelow{The Knowledge Commons Framework}
\end{figure}

In the following sections I use the Knowledge Commons Framework to analyze ICOADS sustainability across three governance regimes. I occasionally draw upon interview data from the ``interactionist'' studies described in Chapter 3. Where this is the case, I distinguish between participants with the following convention - ICOADS Steering Committee (ISC) and randomly assigned number (e.g. [ISC-06]). Participants are not identifiable through this coding nor the selection of transcribed interviews used in this chapter. 

\section{Background Environment}

\subsubsection{Knowledge Production in Climate Science}

To better understand the historical precedence of ICOADS it is important to first explore some of the unique aspects of the background environment from which this project emerged. This includes some preliminary words about differences between weather and climate data, and the types of research that each facilitate. 

An overly simplistic explanation might be that climate data are just weather data aggregated, and then averaged over long periods of time. This is partially true in that a basic understanding of the climate is necessarily dependent on past weather related observations, but the coverage and completeness of those weather records are, for reasons I describe below, highly variable, and the calculations of past events hardly average. 

Understanding and studying climate is based on inference, calculation and interpretation of past weather observations which are heterogeneous in type, of variable quality, and often subject to different, and sometimes conflicting property rights regimes. Joseph Fletcher, the first PI of COADS, put the problem of data access for the study of climate this way: 
\begin{quote}
``In laboratory science we can often formulate the question ourselves and design experiments to test possible answers. For geophysical systems nature defines the behavior we seek to understand and we learn what that behavior is by observing nature. We must, also, test our hypotheses against this observed behavior. This is not as neat as a laboratory experiment but is unavoidable. The problem of understanding climate change is further exacerbated by the long time scale. We have no choice but to look backward in time because we cannot afford to wait for the future to unfold. '' (Fletcher, 1992)
\end{quote}

To bound this overview, I'll first answer two basic questions related to the enterprise of climate science research: 

\begin{enumerate}
\item Where do historical weather records come from?, and
\item How is knowledge about the climate produced from these records? 
\end{enumerate}

Following this section, I then turn to the historical events leading up to ICOADS. \\

\textbf{Where do historical weather records come from?}\\

As Fletcher described above, the earth and environmental sciences depend on observational data to produce new knowledge. Contemporary sources of observational data include platforms such as networked censors, land-based observing stations and data loggers, radiosondes and aircrafts capable of sampling in the upper-atmosphere, satellites, drifting and moored buoys, and many manual forms of data collection such as a graduate student sitting in a field digging up fossils as evidence of a species occurrence event. 

Climate science, like many of the natural sciences, depends on historical records that have been preserved, archived, and made available for reuse in doing large-scale analysis. Because we cannot travel backwards in time to re-sample weather events of the past, the enterprise of climate science has developed a number of ingenious techniques and strategies to overcome limitations of spatial and temporal gaps in historical records. These innovations include finding new sources of data, such as geophysical indicators of weather events taken from the cores of trees, or coral reefs; the digitization of analog resources such as weather logs kept aboard sailing vessels from past centuries; or, through sophisticated computational techniques that stretch existing data to cover geographic regions and time periods with sparse data coverage. 

When it comes to data the field of climate science is entrepreneurial - it takes whatever resources are available, and where there are none, develops new techniques to produce a best estimate of climatological history.  \\

\textbf{How is knowledge produced from historical weather data?} \\

Climate science is a slow moving, and increasingly conservative domain of knowledge production. This is due in part to it's recent politicization, but also due to the fact that it relies upon the steady improvement of a pre-existing set of observations. Like historians who rummage through the archival records of man, climate scientists are the archival scavengers of the earth's record. As Edwards describes this view, `
\begin{quote}
``Their work is never done. Their discipline compels every generation of climate scientists to revisit the same data, the same events — digging through the archives to ferret out new evidence, correct some previous interpretation, or find some new way to deduce the story behind the numbers. Just as with human history, we will never get a single, unshakable narrative of the global climates’ past. Instead we get versions of the atmosphere, a shimmering mass of proliferating data images, convergent yet never identical.'' (2011, p. 431)
\end{quote}

The ``convergent yet, never identical'' nature of climate science described by Edwards is partially what makes comparison, sharing, and cooperation so incredibly important to the generation of reliable knowledge in this domain. As such, the sharing of datasets, the creation of centers of excellence around one particular type of model, or physical system are commonplace. 

Unlike other domains of knowledge production, climate research results are regularly summarized in national and international reports that act as ``consensus'' documents about the state of climate knowledge. These high-level documents serve as the basis of policymaking and international political and economic action related to climate change. This includes highly publicized documents such as as Assessment Reports (AR) issued by the International Panel on Climate Change (IPCC), as well as national level reports that are more regionally focused on extreme weather events and their societal or economic impact. 

\subsection{History and Policy of Meteorological Data} 

In this section I'll cover two important aspects of ICOADS background environment:

\begin{enumerate}
\item Early international cooperation and standardization efforts by Lt. Matthew Fontaine Maury, and 

\item A set of international resolutions, treaties, and bills in the US Congress that address the commercialization of meteorological data. 
\end{enumerate}

\subsection{Early International Cooperation}

Early meterological efforts to improve the reliability of weather data included efforts to both increase the speed at which information was exchanged, and coverage of observations. For instance, the foundation of the US Postal Office in 1792 included expedited services for weather related information (John, 1998), and as early as 1849 the Smithsonian began issuing meteorological devices to telegraph companies in hopes of increasing the speed at which weather observations could be collected and exchanged (NOAA, 2011). 

At this time, most data sources were land-based observations, or lower atmosphere measurements taken through rudimentary meteorological instruments. A critical dimension of this expanding network were the weather records of the sea - which posed a considerably more difficult challenge for data collection . In the middle of the 19th century a U.S. Naval lieutenant, Matthew Fontaine Maury, would attempt to organize an international network of ocean-based weather observation, and in the process set the stage for what is now a robust marine climatology enterprise. 

\subsection{Maury's Rubbish Data} 

Like many tales or heroism Lt. Matthew Fontaine Maury may have fallen into greatness rather than chosen it for himself. On his honeymoon in 1839, Maury was thrown from a stage coach and badly fractured his leg. The injury prevented him from returning to a regular post at sea for twelve months. Maury spent that year writing a series of anonymous publications about strategic military directions for the US Navy, and agitating for new and improved marine charts. His letters garnered the attention of many naval officers and congressmen, eventually leading to his identity being revealed, and his appointment to a new post, the director of the NAVY Depot of Charts and Instruments. 

While in this post Maury sought to improve the reliability of naval weather records. In doing so, he operationalized a massive effort to create charts of winds and sea currents from old log-books that ``had been stored away in the Hydrographic Department as rubbish'' (Corbin, 1888). The processes he put in place for creating these charts, described in many accounts of this period, are similar to contemporary data curation; similar resources are aggregated, their measurements are normalized and recordings are quality controlled, and summary documents are produced for reuse.

Having circulated and promoted the resulting charts widely, word began to rapidly spread of their usefulness. By 1854, the charts were used regularly by shipping vessels the world over (Lewis, 1996). A merchant's magazine of the time calculated that the total amount of money that the shipping industry could save using Maury's charts was on the order of \$2.25 million annually (Corbin, 1888 p. 56). A British report issued the following year estimated the savings to all British vessels at around \pounds 10 million\footnote{Both calculations are in dollar amounts of 1853, and 1854.}.  

After reports of the economic impact of his work, the Secretary of the Navy wrote to Congress pleading for renumeration on Maury's behalf. One letter draws particular attention to the free and open exchange that he ``unselfishly'' initiated. The secretary goes on to say that ``...Mr. Maury might have secured a copyright'' for the charts which were  ``sitting unused,'' but instead Maury created a free resource that would allow ships to ``sail more safely and quickly than ever before''. The letter emphasizes that Maury had, through salvaging rubbish data, created a ``common property of the world.'' (VMI as summarized by Corbin, 1888)

Along with the ``Charts and Winds'' publication Maury also distributed blank logs, and in the margins he prescribed a method of recording measurements such as wind direction, temperature, etc.  

The impact of Maury's approach to data aggregation and analysis was further cemented by a publication entitled ``The Physical Geography of the Sea and its Meteorology'' which went through over twenty printed editions garnering Maury some small, but international fame. 

In 1853 Maury coordinated an international congress on meterological observations in Brussels, Belgium. Participating nations included Germany, Austria, UK, Netherlands, Sweden, Norway Spain, as well as Brazil, Chile and Prussia (Corbin, 1888 p. 72). 

At the 1853  Congress Maury made two proposals: 

\begin{enumerate}
\item ``...all maritime nations should cooperate and make these meteorological observations in such a manner and with such means and implements, that the system might be uniform and the observations made on board the public ship be readily referred to and compared with the observations made on board all other public ships, in whatever part of the world.'' \\

\item ``...it becomes not only proper, but politic, that the forms of the abstract log to be used, with the description of the instruments to be employed, the things to be observed, with the manipulation of the instruments and the methods and modes of operation should be the joint work of the principal parties concerned.''\\ 
\end{enumerate}

Each participating nation agreed to carry out these specific terms, and to broad and sustained meteorological cooperation more generally. In a letter to  USA Congress, Maury notes his success as well as the precedence of this meeting: 
\begin{quote}
``Rarely has there been such a sublime spectacle presented to the scientific world before all nations agreeing to unite and cooperate in carrying out according to the same plan one system of philosophical research with regard to the sea. Though they may be enemies in all else, here they are friends. Every ship that navigates the high seas with these charts and blank abstract—logs on board may henceforth be regarded as a floating observatory — a temple of science. ''(Corbin, 1888. P. 275)
\end{quote}

Commenting on this foresight 150 years later, the director of the World Meteorological Organization (WMO) saw Maury's contribution as providing a normative framework for the field, lying down principles that were ``...so basic to our understanding of how meteorology should be done; (a) all nations should cooperate; (b) observations should be standardized; (c) the enterprise should be global; and (d) the parameters measured, the data recording and exchange, and the instruments and methods of observation should follow an agreed plan.'' (Rasmussen, 2003) 

In summary, Maury's legend teaches us two important things about contemporary work in climate science:

\begin{enumerate}
\item He demonstrated the immense value in aggregating and summarizing previously existing data, and in the process developed schemes to standardize the recording and observations of data, greatly reducing normalization and aggregation efforts in the future.

\item Maury established the principal of international cooperation for the free and open exchange of marine data. Twenty years after this congress, the International Meteorological Committee (today the World Meteorological Organization) was founded to institutionalize this international cooperation. It would take another century before the principal of free and open data exchange would be cemented in policy. 
\end{enumerate}

I turn now to the historical moments in which the WMO was formed, and the continued creep of commercial interest in this domain. 

\subsection{WMO and the Principle of Free and Unrestricted Exchange}

In the wake of the cooperative agreements made in Brussels during the 1853 congress, many nations began to establish national meteorological offices and weather services. But it wasn't until 1950 that a World Meterological Organization (WMO) was founded. In 1953, exactly 100 years from the first international congress in Belgium, the WMO articulated the following conventions: 

\begin{itemize}
\item To facilitate worldwide cooperation in the establishment of networks of stations for the making of meteorological observations or other geophysical observations related to meteorology and to promote the establishment and maintenance of meteorological centers charged with the provision of meteorological services;
\item To promote the establishment of systems for the rapid exchange of weather information;
\item To promote standardization of meteorological observations and to ensure the uniform publication of observations and statistics;
\item To further the application of meteorology to aviation, shipping, agriculture and other human activities; and
\item To encourage research and training in meteorology and to assist in coordinating the international aspects of such research and training. (WMO, 1953))
\end{itemize}

In the following years, two critical resolutions from the WMO were passed - both of which hold implications for the sustainability issues facing contemporary climate science research infrastructures.\\ 

\textbf{Resolution 35 - 1963}\\

In 1963, WMO Resolution 35 established the open exchange of data among WMO members, including maritime log-books for processing historical weather observations. The resolution was non-binding, but set out recommendations for how data \emph{should} be formatted for exchange, including International Maritime Meteorological (IMM) punched card and tape (IMMPC and IMMT) formats. This resolution was significant, because it sets a precedence for what is referred to as a ``principle of free and unrestricted exchange of meteorological data ''between members of the WMO. Like precedence in judiciary law, this principle will be continually referenced by WMO members advocating for the WMO to enforce open exchange of data through rigid policy language.\\

\textbf{Resolution 40 -1994}\\

In 1994 there was a strong lobby from a WMO working group on the commercialization of weather data to revise Resolution 35, along with a number of other previous resolutions guaranteeing the open and free exchange of data between all WMO member nations. The resolution draft was titled  ``WMO policy and new practice for the exchange of meteorological and related data and products including guidelines on relationships in commercial meteorological activities.'' The proposal outlined criteria for two tiers of WMO member data: 

\begin{itemize}
\item The first tier would include all data required to carry out WMO programs - including extreme weather event monitoring, etc. 

\item The second tier, included all non-essential data. This was to be exchanged at the discretion of individual nations. (Co-Data, 1997) 
\end{itemize}

By creating a broad ``non-essential'' tier of resources the proposal would allow commercial vendors to reach distribution agreements with individual nations, and in turn privatize the loosely defined non-essential data products. As a result, Resolution 40 became highly contested and a rift within WMO opened between free market advocates, and those who would defend a ``principle of free and unrestricted exchange of meteorological data ''(Flemming, 2013). 

The resolution that the WMO eventually voted on, and passed in 1995 states the following:

``As a fundamental principle of the World Meteorological Organization (WMO), and in consonance with the expanding requirements for its scientific and technical expertise, WMO commits itself to broadening and enhancing the free and unrestricted international exchange of meteorological and related data and products'' (WMO, 1995)

The two separate tiers of resource access were eliminated, but there remained an ambiguity in the resolution's language which read that ``data (and related products) required for carrying out WMO program ``shall''be made available by all members'', and all other data ````should'' be made available'' (WMO, 1995). This caused a considerable amount of confusion about what constituted data required for WMO programs, including data that could have an intellectual property claim by individual scientists. On this matter, minutes from a CO-DATA working group at ICSU are telling - summarizing the somewhat strange wording they note ``Group members argued that it was inconsistent to urge full and open access and yet establish a policy whereby scientists can have proprietary rights for a period of time.'' (1997)  These distinctions also did little to clarify how WMO members from developing countries could develop a cost-recovery model that charged for access to data. 

Although Resolution 40 avoided creating a commercialization of ``non essential'' meteorological data it did create ambiguity in what was considered to be a viable method of cost recovery. What's more, it did little to quell the potential of enclosure for the meteorological data commons. The impact of this ambiguity is also important for understanding why meteorological projects so often speak in collective terms, and why historical events, such as Maury's 1853 Congress, are so often described in the formal literature of marine climatology: these events are a subtle reminder that although policy creates the opportunity to act otherwise, it is the historical precedence of cooperation that \emph{should} guide contemporary actions. 

\subsection{Continued fight over intellectual property}
In the broader legal and political arena of the 1990's similar debates over copyright and intellectual property of data were taking place. I'll summarize three of these events relevant to ICOADS: 1. Language from a rejected WIPO treaty on intellectual property protection for database creators; 2. The failed bill H.R. 3531; and 3. A second failed bill, H.R. 2652, both in the US Congress. 

In 1996, an international treaty was developed and put up for consideration by the World Intellectual Property Organization (WIPO) to make databases a patentable intellectual property (including data held in weather and meteorological data centers). The proposal included consideration for the following key points related to climate data, and environmental information more generally: 

\begin{itemize}
\item Prohibit unauthorized extraction, use, or reuse of any database, or any substantial portion of a database (as defined by the database vendor), and effectively establish the basis for a pay-for-use system,

\item Apply to all privately generated data or repackaged U.S. government data, and

\item Establish strong civil and criminal penalties, including penalties for third-party liability (liability incurred by an unwitting intermediary or disseminator). (NCAR, 1997)
\end{itemize}

After much debate, the WIPO rejected the language of this proposal, in part, because it lacked a firm definition of what could be considered ``intellectual effort'' in assembling a collection of facts into a database. Instead, the ``Berne Convention'' which creates international copyright of digital materials, not including scientific data, includes the following language ``protection of intellectual property rights may extend to compilations of data or other material (databases), in any form, which, by reason of the selection or arrangement of their contents, constitute intellectual creations. (Where a database does not constitute such a creation, it is outside the scope of this Treaty.)'' (WIPO, 1996). 

In response to the both the WIPO's decision and Resolution 40, NCAR's then president Richard Anthes wrote the following, 
\begin{quote}
`` Despite the recent successes with WMO and WIPO, few people believe the debate is over for good. The issues involved in using the Web for dissemination of data, and in balancing the rights of the commercial sector with the legitimate needs of the research and educational communities, are pervasive and challenging enough to ensure continued discussion. Perhaps the biggest victory in the recent bout is that the arena for this discussion has expanded significantly. The atmospheric sciences community, which has written so persuasively in recent months, will remain engaged as the debate unfolds.'' (1997)
\end{quote}

And indeed, these debates continue up to present day WMO activity. For instance, in 2013 a panel of experts was commissioned by the WMO Executive Council to draft a report outlining the need for Resolution 40 to be revised. One of the panel's main questions was whether or not the ``principle of open and free exchange'' was being honored in the Global Framework for Climate Services (GCFS), a UN-led initiative to guide the development and application of science-based climate information and services in support of decision-making (WMO, 2015)\footnote{As of May, 2015 this panel has yet to report its findings.}.\\

Two other important attempts at privatizing meteorological data are worth noting:

\begin{enumerate}
\item In 1996, H.R. 3531 ``Database Investment and Intellectual Property Anti-piracy Act'' was introduced to the US House of Representatives. This bill sought intellectual property protection for digital technologies, including the patenting of scientific databases. Debate about the impact of the bill included extended discussion about how patents would have affect the USA's ability to participate in the exchange of scientific data supporting international research agendas.  The bill never made it out of committee and expired with the 104th Congress (Sarewitz, 1998).

\item The 105th Congress would try again. H.R. 2652, titled ``The Collections of Information Anti-piracy Act, '' was a broader bill that included a number of different ways for data and software to be patented. This bill was eventually passed by the House Representatives, but its language was folded into the Senate's ``Digital Millennium Copyright Act ''(DMCA). Like its predecessor H.R. 2652 expired with the 105th Congress. Through Senate negotiations around DMCA, language including the protection of scientific databases and data was eventually dropped. (Sarewitz, 1998) 
\end{enumerate}

\subsubsection{Conclusions from Policy \& History}
The review of relevant policy and historical events sketched above provides important background for the development of ICOADS - both from an ethical, as well as a legal prospective.  The default settings of cooperation are made clear with two conceptual turns of phrase: 1. The description of Maury's standardization work as providing a ``common property of the world'', and 2. The commons sentiment being further codified in WMO policy that protects a ``principle of free and unrestricted data exchange''; and the continued political defeat of bills attempting to make meteorological databases patentable. 

\subsection{Goals and Objectives}

At a broad level, the WMO resolutions described above establish a basis for the governance of exchange and access to marine surface data. However, to focus on formal policy alone would be to overlook a wealth of innovation and ingenuity in the design of cooperative institutions for the production and provision of these resources. The collaborative communities responsible for producing these resources are nested within a number of different organizational settings, and bound to different disciplinary and cultural norms. An overlap of institutional commitments has required each new climate or meteorological endeavor to reflect upon the established norms of the WMO, Maury, and other their commitment to principles of free and open data. Few of these institutions are bound, legally, to cooperate or share resource with one another. 

The initial goal of the COADS project was to fill a gap in the historical climate record through creating a systematic and high quality archive of marine data. An early publication notes that the impetus for COADS is a well understood, but under-appreciated fact:  ``The world ocean covers over 70 percent of the earth's surface. The history and future of global climate therefore cannot be understood without ocean weather observations.'' (Woodruff et al., 1987)

COADS ambition then was to be a  ``...readily accessible archive of global climate information'' not simply a dataset, as the name implies (Woodruff et al., 1987). This is an important distinction as it has ramifications for the definition of the resource set, and the property rights claimed by ICOADS users, developers, and provisioners. 

Critically important to this development was the inclusive nature of the project. A more contemporary publication from the core team of ICOADS developers summarizes the sentiment of this project nicely,``This is an open community and new participants are welcomed.'' (Woodruff et al., 2011) In an interview with a longtime international partner of the project, I asked about early work in developing a public good, and she replied this way, ``...its not, this isn't something that we specifically talk about, but it underpins everything we do. It is this idea of creating a public good, making everything accessible. Making it easy for people really underpins everything we do. I don't think we ever talk about it... but we're in the business of creating public goods.'' [ISC-02]

\section{Attributes}

This section includes an overview of characteristics defining the COADS resource set, and community members. 

\subsection{Resource Characteristics}

In the most basic sense, ICOADS is a set of computational infrastructures and databases containing historical marine surface records from the 17th century to the present day, available for download from two data centers (NCAR's Research Data Archive, and NOAA's National Climate Data Center). The resource set - that is, the various resources needed to make meaning of these records - is considerably more complex; including, documentation and metadata records about the data, software for accessing and sub-setting the data, and services for curating and provisioning the databases (as well as producing new versions of the databases). 

\subsubsection{Surface Marine Data}

Surface marine data include physical phenomena observed and captured at the Ocean and Atmosphere interface - typically considered to be ~15 meters into the atmosphere from the sea surface. Data points might include any of the following phenomena: Air Temperature, Cloud Amount/Frequency, Cloud Height, Cloud Types, Dew Point Temperature, Humidity, Ice Edges, Precipitation Amount, Pressure Tendency, Sea Ice Concentration, Sea Level Pressure (SLP), Sea Surface Temperature (SST), Surface Winds, Swell Visibility, Wave Frequency, Wave Height, and Wave Speed/Direction. 

Marine surface data have been collected using a variety of measurement techniques, including on-board ship instrument, moored and drifting buoys, radiosonde, and satellite. The processes of data collection in each of these platforms has changed significantly over a period of coverage which reaches back to the middle of the 17th century. For this reason the quality, reliability, coverage, accuracy, and accessibility of these historical records are, to say the least, variable. 

As such, doing climate research with marine surface data requires coverage that is both spatially and temporally significant. Spatially significant means that data points are sampled equally across the ocean's surface. Temporally significant means that data are recorded at regular intervals of time across those spaces. The infrequent and unsystematic nature of how data were collected by sea-faring vessels, which are the overwhelmingly dominant observing platform for marine data until the 1970's, are an artifact of the rudimentary practices of sailing in the seventeenth and eighteenth century: ships took indirect routes between ports, and often recorded weather data infrequently or haphazardly as a result. 

To overcome limitations in the historical record,  data are often interpolated or ``gridded ''(converting scattered individual data points from a single observed surface into a regular grid or raster of derived values). Gridded data are produced in monthly summaries so as to make large volumes of data accessible to end users.  

Data centers or individual research teams may each take a different approach to correcting errors, resolving gaps in spatial and temporal coverage, or quality controlling marine surface data. This means that although the same historical records can be used as inputs by different data centers, they can and do develop appreciably different blended data products. Taking different approaches to quality control also means that one archive, or research team, can specialize in a particularly difficult aspect of the historical climate record, and even build a reputation internationally for producing high quality, reliable products based on a single variable. 

Unlike the laboratory sciences, datasets in meteorology are not built from a single source, but instead drawn from a common, openly accessible pool of resources that are beyond the capabilities of any one institution alone to collect. These derivative products require substantive intellectual efforts to improve. This often makes claims on property rights related to marine surface data unclear, and occasionally contested. The ``common property of the world'' sentiment espoused by Maury largely carries over to the sharing of data products in this domain, however, as described in detail below, these issues are a form of social dilemma which threatens the sustainability of these resource sets.

\subsubsection{ICOADS}

As described above, ICOADS was meant to be a comprehensive archive of historical marine surface weather information. In an early project meeting, the lead PI of COADS put this directly, ``When we look backward in time we find the observational record frustratingly incomplete. We have satellite records for only a couple of decades and very incomplete radiosonde coverage for about four decades. To document longer term behavior we have only surface observations and for reasonable spatial coverage of this water planet they must include data from the ocean domain.'' (Fletcher, 1992)

The first release of COADS in 1983 included 11 different data products and their processing history (Jenne and Woodruff, 1986), a Fortran 77 program on magnetic tape used to read the packed binary-data (Woodruff et al 1987), as well as a number of publications describing the statistical trimming methods used to create derived variables (monthly summaries) of the aggregated data (Slutz et al., 1985; Fletcher, 1983). 

Below I describe four key components of the ICOADS resource set: 1. Data, 2. Metadata, 3. Standards, Software \& Statistics, and 4. Computational Infrastructure.  

\subsubsection{Data Products}

ICOADS data products have increased in size, and sophistication over time. The first release of COADS in 1983 included 71.9 million records. Release 2.5 of ICOADS includes over 250 million records. As new resources are discovered, digitized, or reanalyzed for improvements, the coverage of ICOADS data products continues to improve in terms of historical coverage. The first release of COADS included records dating back to the 1850's, but continued international cooperation, as well as recent contributions from projects like the RECovery of Logbooks And International Marine data (RECLAIM), and citizen science initiatives such as the Old Weather project have extended the records back until the middle of the 17th century (1662). 

\begin{figure}
\includegraphics[width=4in, height=2.5in]{icoads-cov}\\
\captionbelow{The expanding coverage of ICOADS over each new data release.}
\end{figure}

\subsubsection{Metadata}\\

Use of surface marine data requires extensive documentation and metadata. This includes formal manuals developed by the WMO to code ship, instrument, and data types, but also less formal documentation, especially for historical data, such as the marginalia in sailing logs, or even the journals kept by crew members that may have noted irregularities in weather during particular days. 

The accuracy of some observing platforms are well understood - such as drifting buoys which are rarely recovered and repaired and so their measurement accuracy may drift undetected and result in decreased accuracy over time (Woodruff et al. 2011, p. 17). Other platforms, such as data collected aboard early sailing vessels, may have limited metadata and are not well understood in terms of accuracy or quality. This variability in platform quality and metadata, as well as the discovery of new data sources (i.e. logbooks that have not been digitized, contextual documentation that is found in an archive, etc) requires a slow, but nearly constant revision of the historical record. 

Additionally, blending these different platforms into one cohesive data product is a non-trivial exercise that requires consistent applications of quality control, and a robust source of computing power to process and homogenize these records. For instance, in announcing the first release of COADS, the project partners make this iterative process clear, warning that,  
\begin{quote}
``Any conclusion drawn from the historical record should be qualified by the fact that the observation, reporting, collection, and digitization of these data have been subject to a great deal of methodological change. Besides introducing more or less unknown inhomogeneities into many variables, these changes have sometimes been processed incorrectly. The resulting errors, as well as simple recording or transmission errors, occur very frequently. While a major effort has been made to indicate reports containing errors, some kinds of errors cannot be trapped by statistical methods. A very common error in the original data was incorrect representation of latitude and longitude, and only in extreme cases were these identified. Thus it must be remembered that while millions of errors have been identified and eliminated from the trimmed summaries, the resulting data are still far from clean.'' (Slutz et al., 1985)
\end{quote}

Some project partners have made their entire career around improving metadata to marine surface data, and contributing those corrections and improvements back to international projects such as ICOADS (e.g. Kent et al., 2007). \\

\subsubsection{Standards, Statistics &\ Software}\\

Finally, two other products important to the ICOADS resource set are the standards it has developed for the documentation and exchange of marine climate data, and the statistical approaches it has created to summarize these types of data. Additionally, the software written and traded amongst community members for accessing and analyzing subsets of ICOADS data are becoming more important to the future work of this project. I'll briefly describe each of these products below. 

\begin{itemize}
\item

As marine surface records became more valued in climate reconstruction projects there also emerged a need to standardize the format of exchange for historical marine data. Until 2000 no uniform, internationally agreed upon format existed. Through a series of proposals, the International Maritime Meteorological Archive (IMMA), was recommended for a standard format for exchanging marine meteorological data, developed and used extensively for the 2.5 release of ICOADS. In short, IMMA is ``ASCII-based format with a fixed ‘core’ set of the most commonly reported meteorological variables sufficient for most users, also including the time, location, and individual platform identification (e.g. ship call sign or WMO buoy number, if available). (Woodruff et al., 2011)'' As a standard IMMA, and its broader adoption, facilitates the exchange of many international projects that model their work after ICOADs, such as the Climatological database for the World's Oceans (CLIWOC). More broadly, it is a contribution endorsed by JCoMM, and was one of the most successful innovations of the project beyond the provisioning of data. 

\item
The ICOADS community, like much of contemporary science, produces and uses a number of unique software packages for accessing and making use of subsets of ICOADS data. These include anything from simple scripts to fetch data from shared servers, to large Fortran libraries for reading and analyzing data in the IMMA format.

\item 
The summary statistics developed to grid ICOADS data are a non-trivial aspect of the innovation of this project. Originally meant to reduce the space that the data would take up on tape reels, the process of calculating monthly summaries, and gridding data at 1\˚ x 1\˚ resolution remains in use today largely for convenience of the community of end users - innovating with this scheme would be both inconvenient and costly with uncertain payoffs.} 
\end{itemize}

\subsection{Community Members}\\

\subsubsection{Data Users}\\ 

Users of ICOADS data are diffuse. In the climate science community, ICOADS data have been used to produce a variety of new data products, and used within a number of different international climate assesment reports. This includes staff scientists where ICOADS is archived (NOAA's NCDC, and NCAR's RDA), as well as scientists and graduate students at universities around the world. 

ICOADS data are occasionally used outside of climate science; examples include a digital historian who traced naval operations during World War I (Mullens, 2013), and a silicon valley entrepreneur giving a commencement address at the Naval Academy (Ondrejka, 2013).

\subsubsection{Data Producers}

Data producers in the ICOADS community are of three types: 

\begin{enumerate}

\item Historical Curators, including individuals whose research agenda are based around bias corrections in the historical climate record. 

\item Contemporary Collectors, including institutions that contribute data to the GTS through moored buoy censor networks, as well as ships which participate in the WMO's Voluntary Ship Observation (VSO) program.

\item Contemporary Curators, including software engineers that both develop new and provision existing data. This includes normalizing new data from the GTS, correcting minor errors in existing data, and providing improved software for subsetting ICOADS data products. Curation also takes place during the production of new ICOADS releases, including the cleaning (normalization and quality control), pre-processing, and scheduling of tasks on NCAR's supercomputers. ICOADS data curators are also the individuals that update and make improvements to publicly facing metadata, send reports or emails to current users to announce known errors, and fulfill individual subset requests. 
\end{enumerate}

\subsubsection{Data Provisioners}

In this section I focus on two institutions which serve ICOADS data, documentation, and software to end-users. \\

\emph{NOAA/ Earth Systems Research Laboratory (ESRL) Physical Sciences Division (PSD), Boulder, Colorado, USA (and) National Climate Data Center, Asheville, North Carolina, USA}\\

NCDC is a relatively new partners in the ICOADS project, and has taken on the service ICOADS data to expert end-users in the federal government (for reasons described below). As a result, Few user services are available, and downloading is only possible through FTP servers. As a result, NCDC serves mostly expert users and federal employees of NOAA. 

A major contribution to the provisioning of ICOADS data from the ESRL's PSD is made through the project primary investigator, Scott Woodruff. Until the winter of 2012, he and one other FTE are responsible for coordinating the documentation, updates, and software development for all on-going ICOADS work. The responsibilities of these individuals also includes handling much of the design and implementation of new releases or enhancements to existing data products served by NCDC.\\ 

\emph{The Research Data Archive (RDA) \emph{National Center for Atmospheric Research, Boulder, Colorado, USA}}\\

The RDA is a repository of atmospheric and oceanographic observational data, weather prediction model output, gridded analyses and reanalyses, climate model output, and satellite derived data that has been curated by staff in the Computational and Information Systems Laboratory (CISL) at NCAR since the mid 1970's (Jacobs and Worley, 2009).  The RDA is unique amongst earth science data archives in that it serves almost exactly the same amount of data as it stores- meaning that the RDA last year contained about  1.3 Petabytes of data, and in total it also served users, located all over the world, about 1 Petabyte of data. 

A major component of NCAR's mission is to serve the entire atmospheric science research community - from government scientists to undergraduates. As a result, the RDA's provisioning of ICOADS has focused on serving end-users of varying skills and technical competencies. For instance, over the last thirty years, the RDA's staff has developed a robust graphical user interface (GUI) allowing for unique subsets of ICOADS data - and also offer curation services to create unique subsets for users who cannot manage the GUI. The RDA has developed a number of metadata elements that are meant to help users discover the dataset more easily. The RDA also provides a suite of Fortran software that allows for observations made in the IMMA (International Maritime Meteorological Archive) format to be read and used in a variety of applications, and they also provide a global archive of year-month observations. In short, access to a number of different ICOADS products, and additional services surrounding the dataset leads to the RDA serving about 90\% of all ICOADS users on an annual basis (the rest being served through NCDC).\\ 

\section{Governance \& Rules-in-Use}

Governance and rules, as argued throughout Chapter 1 and 2 of this dissertation, are critical aspects of effective institutions for collective action. Governance and rules are studied through what the IAD and KCF call ``action arenas,'' which are analyzed by observation, participation, or archival works about how``individuals (acting on their own or as agents of organizations) observe information, select actions, engage in patterns of interaction, and realize outcomes from their interaction.'' (McGinnis, 2011; Contreras, 2014). 

In the sections that follow I divide ICOADS governance, and the evolution of its rules into three distinct regimes of governance; the events around the formation and dissolution of a governance regime can be seen as an ``action arenas,'' Within these ``action arenas'' I identify action situations where rules are proposed, enacted, or negotiated amongst participants in both the provisioning and production of ICOADS. 

I draw on the ethnographic, and interpretive interventionist data in constructing the narratives below. In particular, these sections  attempt to tie events of the past to contemporary issues through the actions, words, and writings ICOADS community members. I characterize the outcomes and patterns of interaction for each separate regime, but note that a full synthesis of the findings across all three regimes is found in Chapter 5. 

\subsection{Three Regimes of ICOADS}

When I began interviewing ICOADS Steering Committee members, I asked each participant a variation of the following question: ``If you were to look back at the history of ICOADS, what would meaningfully separate one period of the project's development from any other? ''The first four interviewees all offered the same time periods - COADS (~1980 - 2000), ICOADS (2000-2011), and the later (often difficult to name) period that resulted from funding problems beginning at NOAA (2012 - present). These periods were also named because they align, somewhat crudely, with each new data release (version) of ICOADS.  In further interviews I corroborated these periods with other ISC and ICOADS community members who, by and large, agreed that these were meaningful divisions to make in the evolution of the project. 

\subsection{Regime 1: 1981-2001}
 
Three important contextual factors that led to the initial realization of COADS: 

1.	In the late 1970's, many earth science disciplines had seized upon the opportunity of increased computing power and decreased cost of data storage to further a climate research agenda. Climate models were just beginning to couple separate physical systems (i.e. ocean and atmosphere, land and atmosphere, etc.) and the potential to create earth system models was brining with it the need for more authoritative, and more complete historical weather data resources (Trenberth et al. 2002). In the case of early COADS work, innovations with ``blending'' different data sources, and with ``trimming'' statistics to create monthly summaries made the potential for a historical data-set a reality. 

2.	Similarly, the geo-political landscape was becoming more peaceful, and cooperative than it had been in nearly half a century. Riding the success of the International Geophysical Year (1957-58), the exchange of archival data among cooperating nations became common place. A minor WMO resolution, 35 (Cg-IV) passed in 1963 further normalized this practice. During this period NOAA was able to obtain a number of international collections of punched card decks from major maritime counties - including archives from the Dutch, British, Japanese, Norwegian national meteorological offices, as well as a collection of rare Russian whaling logs from the 19th century. By the late 1960's efforts to combine the different logs had resulted in a series of ``Tape Deck Families'' (TDF) that could be shared amongst cooperating nations (Woodruff et al., 1987). Recognizing the potential value in this exchange, the WMO then initiated an international effort to improve bias in these records through the Historical Sea-Surface Temperature (HSST) Data Project; with the NSF later supporting the USA's participation through a funding program titled,  National Science International Decade of Ocean Exploration (Quayle, 1977; Woodruff, 1986). Because the records were so large, and processing so computationally intensive, the HSST divided the project up by Oceans - the USA responsible for Pacific, Germany for the Atlantic, and the Netherlands for the Indian Ocean. This initial effort generated a new set of SST data products, as well as new techniques for both processing and combing different data sources (buoys, ship logs, etc.). Notably, this was one of the first WMO projects to use magnetic tape reels to store data -making it considerably easier to access and use larger volumes of the archive.  

3.	Advances in supercomputing were enabling data processing tasks, in particular the calculation of floating point operations, at speeds that were previously unimaginable. Although still prohibitively expensive to most research projects this new calculative power brought with it the possibility to create comprehensive records by processing a large number of inhomogeneous datasets at a single location (Jaffe and Woodruff, 1983).  

\subsubsection{Pre-assembly}

While each of these factors was important for COADS to be initially assembled, it was perhaps the last point, which proved most fortuitous to the ongoing HSST effort to create a comprehensive archive. In 1980, Wilmot “Bill” Hess left NOAA’s Environmental Research Laboratory (ERL) to become the Executive Director of  NCAR. Shortly after his appointment Hess secured funding from DOE for NCAR to purchase a second supercomputer, a used Cray-1A\footnote{Besides having about 4.5 times the throughput of NCAR’s current computing infrastructure, the Cray 1A also included the first automatic vectorizing Fortran compiler (Enabling many GCM’s to be used and tested by NCAR’s community while they were developing the Community Climate Model 1 (Bath et al., 1987)).}. Since NCAR purchased the Cray 1A used, it came with a trial period that allowed NCAR engineers to calibrate and verify the machine’s capabilities – making certain that it met predetermined performance benchmarks.

\begin{figure}
\includegraphics[width=4in, height=2.5in]{HistoryOfComptuingAtNCAR}
\captionbelow{\textbf{Caption}: NCAR's supercomputing history, including the early CRAY 1a}
\end{figure}

Before moving to NCAR, Hess had worked with J.O. Fletcher and was sympathetic to the need for a comprehensive marine based climate record. Having seen the value of such an archive during the HSST project, Hess decided to allocate a full third of the new Cray’s trial time to the processing TDF data, that could be made into a single, homogeneous dataset. In a previously recoded interview Fletcher described how difficult ICOADS data processing was,``…just to read through it without doing anything to reformat, or anything else, at that time was about \$ 100,000 worth of computer time, just to read through the goddamn thing once''. (Shoemaker, 1997) 

In total, the donation included over 1000 hours of Cray-equivalent CPU time (Woodruff et al., 1987). In the initial publication that announced COADS to the climate community, Slutz et al acknowledge this debt, saying ``Throughout the effort, the support and encouragement of Dr. Wilmot N. Hess was crucial, as Director of ERL during the early stages and as Director of NCAR during the later stages. ''(1985)

\subsubsection{COADS versions 1, 1a, 1b, \& 1c}\\

Having collaborated widely on the initial assembly and processing of the data, NOAA, NSF (through NCAR) and the Cooperative Institute for Research in Environmental Sciences (CIRES) at the University of Colorado, began a formal cooperative agreement to maintain and distribute the Comprehensive Ocean and Atmosphere Dataset (COADS) into the future (Fletcher, 1983; Woodruff, 1985). 

In a foreword to one of the first COADS related publications, Fletcher describes the process of creating the archive, saying:
\begin{quote}
``It has taken four years and much effort by many individuals and several institutions to obtain and process the hundreds of tapes containing the basic data input. All of this effort was provided from ongoing activities; there was no appropriation identified for the task. It is a tribute to the spirit of cooperation among the participating organizations that the task has been successfully completed.'' (Slutz et al., 1985) 
\end{quote}

The fact that no direct government appropriations were used for the initial creation of the COADS meant that there was also no expectation about how it was to be served to end-users, updated, or maintained. While having no funding meant that this work was somewhat sporadic in being completed, it also meant that there were no cost-recovery expectations by federal funding agencies. 

It is important to note the historical context of this work; I've already covered the WMO resolutions which described a moral imperative to freely share marine data, but there were also a number of other legal and political pressures at the time to monetize scientific research. In particular the  1980 Bayh-Dole Act had amended existing patent law so that the intellectual property rights of inventions stemming from government funded research were no longer re-assigned to the federal government (Mirowski, 2011). Universities, small businesses, and non-profits, including FFRDC's, were allowed to retain ownership of inventions, and monetize these resources as they saw fit. While, it would have been a contentious move to try to capitalize on the international cooperation that resulted in COADS being assembled- the spirit of entrepreneurship around basic science research was pervasive in the early 1980's (Giddens et al, 2003). The decision of whether or not to create an artificial barrier to entry (i.e. a price or membership model for access to COADS) should be seen through this historical lens - the archive could have either been distributed for free; or NOAA, NCAR, and CIRES could attempt to recoup some expenses incurred in devoting labor and resources to creating COADS.  

The UK’s Meteorological Office had been simultaneously processing and developing a similar set of historical weather observations, the Main Marine Data Bank (MDB). The MDB was nearly as extensive in geographic coverage as COADS, and included a number of unique datasets that were not present in the initial release of COADS. At the time the UK Met Office was, like many nations in the 1980’s, selling these records to recoup the costs of production and service. COADS partners felt the best way to achieve an authoritative record were to combine efforts of the two different projects (Woodruff, personal communication). The UK partners agreed, but asked COADS to purchase the MDB. COADS partners initially rebuffed the offer and decided to focus instead on creating a free and openly accessible archive in the spirit of Maury's ``common property of the world'' (Parker et al, 2001).

In announcing the initial availability of COADS, early contributors often described this decision on ethical grounds. For instance, an early publication that detailed the first release of COADS stated that the project would consistently attempt to be authorotative, and to further  ``make this record available to the individual investigator in a form that is reliable and easy to use for anyone interested'' (Slutz et al, 1985). 

\textbf{COADS 1 (1979-1887)}

The first release of COADS included a date range beginning at the development Maury’s standard logbook (1854), up until NOAA began processing data from the set of Tape Deck Families described above (1979). In total, the tapes assembled for the initial release contained more than 71 million unique observations.Release 1 would included 11 different data products and their processing history (Jenne and Woodruff, 1986), a Fortran 77 program on magnetic tape used to read the packed binary-data (Woodruff et al 1986), as well as a number of publications describing the statistical trimming methods used to create derived variables (Slutz et al, 1985; Fletcher, 1983). 

\textbf{COADS 1a, 1b (1992-1999)}

COADS Release 1a, and 1b simply update and improve the coverage of interim data products that were developed until 1992. This release also saw the adoption of systematic way to add newly collected data to the existing records - a practice that would continue until 1995. COADS 1a was also the first release to make electronic documentation available for download through an anonymous FTP server. Release 1b improved data quality issues discovered in records from 1950-1979, and included a set of Russian marine logs from Arctic explorations. These interim releases also marked a point of controversy in the early project. Biases discovered in the blended products were numerous, and this led to disagreements about how the records should be updated to reflect known errors. The re-processing of gridded data would be expensive, and time consuming, but establishing COADS as a credible source was one of the foremost early goals of the project. 

\subsubsection{Review of Governance in Regime 1}

The blending different data sources to create an authoritative record of marine surface data was enabled by three factors: 

\begin{itemize}
\item WMO Resolution 35 (cg XI) and the success of the IGY in 1957-1958 created a more cooperative international exchange of marine data sources. 

\item The HSST project serves as a proof of concept - a comprehensive, and authorotative historical record of marine weather was not only possible, but would be exceptionally valuable to the climatology community

\item Supercomputing facilities, and statistical trimming techniques were available to make such a historical archive readily exchangeable and useful to ongoing research. 
\end{itemize}

\textbf{Governance Variables}

The production of early versions of COADS follows a very traditional self-organized model of governance. The discretionary spending used to fund the initial project, and the appeal to WMO resolutions for governance principles (i.e. free and open exchange, no hierarchy of contribution) contributed to the creation of a resource set that had low-rivalry, and low degrees of excludability - which was in opposition to competitor products which were being sold, excluding a broader climate community from accessing these resources. In short, this regime establishes a marine data commons that is to be informally governed through provision and production.\\ 

\subsection{Regime 2: 1999-2009}

Three institutional features are prominent in COADS second governance regime: 

\begin{enumerate}
\item The project is renamed to ICOADS in recognition of international contributions to release 2.

\item A governing body, The Joint Technical Commission for Oceanography and Marine Meteorology (JCOMM) is established by WMO, further nesting ICOADS work in the international climate and weather data enterprise, and providing a closer link to WMO as an international governing body for meteorological data. 

\item A set of meetings, the JCOMM Workshop on Advances in Marine Climatology (CLIMAR) and the Workshop on Advances in the Use of Historical Marine Climate Data (MARCDAT), to exchange results from recent research in marine climatology, and to coordinate future improvements and enhancements to the historical marine climate data record more generally.
\end{enumerate}

When asked to describe this period, a participant characterized this as a
\begin{quote}
``...time when other users of ICOADS started collaborating and feeding back on the project. So, people like, big people in the field,  started doing a lot of work with ICOADS and using it for various new products that they were doing and feeding all of that back, what they learned, to project PI's - that is really the start of the evolution of ICOADS.'' [ISC-04]
\end{quote}

\subsection{ICOADS Release 2}

In 1997, after incremental improvements to COADS 1, 1a, and 1b were made publicly available planning began for the development of a second release. The attention and enthusiasm that a decade's worth of science had produced for COADS brought with it new sources of data and funding (Woodruff et al, 2001). Producing a new release of COADS included reprocessing, and reanalyzing all existing data, as well as blending new sources discovered in the interim period (since 1979). This is a somewhat unique aspect of the production process in ICOADS new releases; in order to properly compare duplicate records, and integrate new data sources in existing gridding schemes, each data source has to be completely reprocessed, using new algorithms and new quality control benchmarks. This requires a substantial amount of coordination, and dedicated computing resources over what is typically a two to three year process of producing a new release. 

During this period of data processing, developers of COADS and the UK Meteorological Office's MDB negotiated terms upon which the two data products would merge - improving known errors in both datasets as well as moving towards a single, authoritative resource for the climatology community to use. In describing the negotiations around this period, every community member that I interacted with emphasized that this was not a natural progression or a necessarily cooperative one. 

One participant explained the state of data access in the following way, 
\begin{quote}
``For whatever reasons, it is much harder to prohibit data access in the US. The UK, you know the MET, the Marine Data Bank, they have a lot of overhead to make data easy to use. And if they understood it they didn't need to create those extra services. They don't. They couldn't invest in it. When you look back at the literature, pre-COADS. People use whatever. Sometimes those TDFs, but there is nothing very authoritative there. The Germans used their archive. I couldn't use MDB because I didn't work at the Met. Nobody had to share. And they didn't. So why did they merge? They had to. The science wasn't getting better without it'' [ISC02]
\end{quote}

Although there was widespread agreement that the combination of the two data sources would create a better product, another participant described the process as contested, 
\begin{quote}
``It was heavily debated...COADS was a NOAA project, and it was US centric- I don't want to say insular, but it was very ``Boulder'' oriented. And the debate was... that is a lot of money, a lot of resources went into collecting this data through the Voluntary Observation Ship program, and that is supported by individual nations. Not from a pool of funds. Those are investments...So that debate really focused attention on the fact that COADS should be looked at as an International dataset, because you know a lot of other countries were putting money and data into it.''
\end{quote}

A single graphic from Release 2.0 emphasizes the point made by this participant; In the figure below we see a breakdown of the nations that contributed archival data to the project. Looking backwards in time to the early 20th century, almost all of these records come from nations other than the USA and UK. Whats more, the HSST data makes up a bulk of this mid-century data products, and as was described above, that project was an international ``proof of concept'' on the cooperation required of producing comprehensive marine data products.\\

\begin{figure}
\includegraphics[width=4in, height=2.5in]{NationsContributingtoICOADS}\\
\captionbelow{\textbf{Caption}: Records in ICOADS as shown by contributing Nation}
\end{figure}\\

The proceedings of the first MARCDAT workshop, where the name change would be voted upon by JCoMM members (described below), records  the process as follows,
\begin{quote}
 ``In its final plenary session, the Workshop voted in support of  the name International Comprehensive Ocean-Atmosphere Data Set (I-COADS) for the new blended observational database. This name recognizes the multinational input to the database while maintaining continuity of identity with COADS, which has been widely used and referenced.'' (Diaz et al, 2001). 
\end{quote}
 The very next sentence in the workshop proceedings invokes the history of this enterprise, and is worth including here, 

\begin{quote}
 ``The Workshop was an appropriate lead-in to the conferences planned by JCOMM for September 2003 in Brussels, to commemorate the 150th anniversary of  the conference convened in Brussels in 1853 by US Navy Lt. Matthew Fontaine Maury to establish, inter alia ,the standardization of meteorological and oceanographic observations from ships at sea. Maury’s work (see Lewis, 1996) remains the foundation of  much operational and research maritime meteorology and oceanography''
\end{quote}\\

The process of renaming the project is particularly useful example of the difference between rules-in-form and rules-in-use. From my participants perspective, the process of deciding to rename the project was contentious, but necessary to advance the science being done. And yet, not one of them remembered that the matter was put to a formal vote. In their minds, the decision although contentious politically, was about scientific progress - as my participant said ``...the science wasn't getting better without it''  

\subsubsection{ICOADS Release 2.0 \& Interim Products}  

In September of 2002, a second full release of COADS, and the first to be titled I-COADS, was made publicly available. The data in this release would initially include two archives, one of which was a real-time mode - that included a blending of data coming from the GTS, and the second was a delayed mode, which would apply stricter quality controls to newer data, adding it on to existing historical records (Worley et al, 2003). The delayed mode archive therefore encompassed Releases 1a (1980-97), 1b (1950-79), and 1c (1784-1949), but reprocessed and improved by the addition of new data sources, including the entirety of the Marine Data Bank. 

Three additional innovations are important to this release period: 

\begin{enumerate}
\item Documentation was put on an ICOADS dedicated website (eDocs). This signaled a further move to decentralizing the knowledge of COADS - users could also now submit bugs, or report errors through this website instead of via personal communication with developers. This is a subtle, but important innovation that creates an ICOADS identity, instead of a NOAA or NCAR identity. 

\item Interim products would now be labeled like software releases, with a decimal point signifying a new incremental improvement (e.g. 2.1, 2.2, etc.). Interim products during this period occurred on regular intervals, meant to show responsiveness to a new international community of participants.

\item Release 2.1 in 2003 would adopt the new ASCII IMMA format, an innovation which would standardize the exchange of marine data, and influence policy about data sharing at both the national (NOAA) level, and the international (WMO) level. This further demonstrates the influence of ICOADS as an international authority, and marks a certain reputation amongst international partners as a standards bearer. 
\end{enumerate}

Release 2.1 would also see the resolution of the ICOADS gridding scheme improve, with ``2° latitude × 2° longitude and 1° × 1° boxes beginning in 1800 and 1960 respectively.'' (Worley et al, 2003). While this seems like a minor improvement, the change in resolution enabled easier integration with other environmental data available at the time (Woodruff et al, 2006).

\subsubsection{JCoMM}

Until 1999 the World Meteorological Organization coordinated data management and observing systems related to meteorology and oceanography through two different governing bodies - the Commission for Marine Meteorology (CMM), and UNESCO's IOC, jointly with WMO, through the Committee for the Integrated Global Ocean Services System (IGOSS). Citing a need to combine overlapping expertise, the WMO wrote that ``While enhancing safety at sea remained the primary objective of marine forecast and warning programmes, requirements for data and services steadily expanded in volume and breadth during the preceding decades. Moreover, many of other applications required observational data sets and prognostic products for both the oceans and the overlying atmosphere.'' (2011). To better coordinate the collection and curation of data in these two domains, the Joint Technical Commission for Oceanography and Marine Meteorology (JCOMM) was formed. To reflect the shared responsibility of this commission, it is co-chaired by both a meteorologist and an oceanographer. 

The combined domain expertise of JCoMM directly preceded the international relabeling of COADS, which blends oceanographic and atmospheric datapoints. The formalization of the governance of this domain in part helped the ICOADS project partners realize the international scope of their work, especially through the founding of marine climatology workshops described below.  

\subsubsection{CLIMAR \& MARCDAT ``Sustaining the International Community''} 

In parallel with the formation of JCoMM as an international governing body of the WMO, two related workshops were designed to bring together stakeholders of the marine climate data community. Advances in Marine Climatology (CLIMAR) brought together a community of developers and users of surface marine climate data, while the Advances in the Use of Historical Marine Climate Data (MARCDAT) workshop focused specifically on users of historical weather data for climate research. CLIMAR and MARCDAT are important to cooperation within those domain, and often result in concerted efforts to improve known errors, or in collaboration around on-going projects.

For example, through early international coordination at MARDAT, ICOADS project partners agreed to coordinate their work at reducing biases related to Sea Surface Temperature (SST) variables, and to increase (where possible) data coverage in order to contribute to planned international climate assessments. Diaz et al write about this process in the proceedings of the 2002 MARCDAT, ``A staged timetable for implementation was agreed: firstly, a two-year period would lead to the third C20C Workshop around April 2004; and, secondly, a period of  about five years would lead to the Fourth Assessment Report (AR4) of  the Intergovernmental Panel on Climate Change (IPCC).'' ICOADS partners successfully met these targets, and the contribution to AR4 was noted by lead authors of Working Group 1, who used these SST measurements extensively (AR4, 2007).

\subsubsection{Review of Governance in Regime 2}

ICOADS second regime developed an institutional capacity to collaborate more broadly, and to systematize a release schedule. During this period, the formalization of an international governing body for meteorology and oceanography (JCoMM) had a nesting effect for ICOADS own governance; it placed organizational pressure on ICOADS to conform to certain models of distribution promulgated by the WMO, but JCoMM also took on the responsibility of coordinating stakeholders through the MARCDAT and CLIMAR workshops. Additionally, the founding of JCoMM brought with it the opportunity to promote ICOADS own standardization work, such as the IMMA format for exchanging marine data, and to gather consensus on future directions of the archive so as to best meet broader climate assesment goals. 

This regime marks a clear division from an initial COADS governance which was ``self-organized '' to a ``polycentric''model which nests rule making and enforcement at different levels, and creates jurisdiction, or institutional overlap in promulgating new rules. 

\subsection{Regime 3: 2009- Present}

The defining features of ICOADS third (on-going) regime is the release of a valuable interim product, ICOADS 2.5, and the continued formalization of its governance body. In early 2012, funding was partially eliminated for critical members of the ICOADS team at NOAA. This caused disruptions for planned enhancements to the archive, but overall the focus of the ICOADS community remains on: 1. Coordination of ICOADS release 3.0 and the ICOADS Value Added Database (IVAD) which brings with it a new contribution mechanism for ICOADS stakeholders; and 2. Making a formal application to the WMO for ICOADS to be recognized as a Center for Marine-Meteorological and Oceanapgraphic Climate Data (CMOC).

\subsubsection{ICOADS 2.5}

In late 2009 data processing began for an interim ICOADS release 2.5. This product would further enhance real-time access to end users, substantially reduce the burden of sub-setting ICOADS, and included a number of improved bias adjustments. Previously released interim products (2.1 -2.4) had incorporated some new data, but a substantial amount of new digitized products from international partners was now available, including newly digitized data from the RECovery of Logbooks And International Marine data (RECLAIM). A journal article describing this new release indicates the further international cooperation in this process, ``Data provision, collation, and distribution remain the responsibility of the founding partners, but other countries and international organisations – including the Joint World Meteorological Organisation (WMO)–Intergovernmental Oceanographic Commission (IOC) Technical Commission for Oceanography and Marine Meteorology (JCOMM) – now make noteworthy contributions.'' (Woodruff et al. 2011). As a result, release 2.5 made substantive improvements in the coverage and completeness of data sampling over the ocean as shown in the figure below. 
\begin{figure}
\includegraphics[width=4in, height=2.5in]{icoads2-5-improv}\\ 
\captionbelow{\textbf{Caption}: From Woodruff et al. 2011 ``Annual percentage ocean area sampled for SST for R2.5 (red curve) compared to R2.4 (blue curve) (right axis). Annual percentage increase in global ocean area sampled for R2.5, compared to R2.4 (bars, left axis).''}
\end{figure}

\subsubsection{ICOADS role in Climate Reanalysis}

Climate reanalysis are relatively recent approach to data assimilation in earth systems science. The major goal of a reanalysis project is to give a numerical account of the very recent past (10, 20, or 40 year time steps) using a combination of archived climate and (very recent) weather data with whole earth models. (Dee et al, 2015)  Because the ``inputs'' for reanalysis assimilations require broad coverage (both space and time), ICOADS has been used regularly in reanalysis projects that include an ocean component (e.g. Trenberth et al. 2001). Reanalysis datasets are some of the most important, and valuable resources that the climate science community has produced in the last 20 years (Edwards, 2011 p. 323-8).

One participant described her work on input datasets for reanalysis projects in the following way, 
\begin{quote}
``My career would have been a lot different if I had focuses on the exploitation of the data rather than the improvement of the data. And that is why citations are important. If you start out with my metadata publications, it gets tens of citations, ICOADS papers get hundreds, but the gridded datasets, and especially the reanalysis stuff those get tens of thousands of publications. And you do look upwards, and think tens of thousands of citations?''
\end{quote}

This is to note that, the financial value of reanalysis datasets have been enormous, they've also paid dividends to those with their names attached. So, while individual scientists working to improve parts of ICOADS records fail to garner much attention, the project as a whole has been recognized throughout the climate community as a high-quality resource partially as a results of its continued use in climate reanalysis projects. This is an important outcome for the project, as developers of ICOADS 2.5 note in their description of enhancements to the archive, 

\begin{quote}
 ``...NOAA, the European Center for Medium-Range Weather Forecasts (ECMWF), the Japan Meteorological Agency (JMA), and the US National Aeronautics and Space Administration (NASA) have all taken advantage of ICOADS for their reanalysis. Communication between reanalysis centers and the ICOADS developers is excellent. Each new reanalysis is based on the most recent ICOADS update, thereby taking advantage of any improvements in data quality or quantity based on the efforts of marine data experts. ICOADS therefore reduces efforts required for data preparation and quality control for reanalysis projects. In a complimentary manner, reanalysis efforts (e.g. Compo et al., 2006) also uncover data problems in ICOADS that feedback to the developers and lead to future improvements.'' (Woodruff et al, 2011).
\end{quote}

\subsubsection{Defunding of NOAA Partners}

Amid political turmoil around federal budget allocations for FY 2013, NOAA announced in the winter of 2012 that it would eliminate future support of ICOADS through the Climate Program Office (CPO). A public announcement made in February of that year reads:
\begin{quote}
``For budgetary reasons, stemming from pending large cuts at the NOAA Climate Program Office (CPO), ESRL Directors have determined that it is no longer feasible for its Physical Science Division (PSD) to continue supporting any further ICOADS work — effective immediately.''(NOAA, 2012) 
\end{quote}	
Even though ICOADS acts as an input to many downstream data products, it is viewed as a research project by federal agencies.  In an extended conversation about this decision, a long-time ICOADS project partner explained this reasoning: 
\begin{quote}
``The number one thing to recognize from that period is, one of the drawbacks for most soft-money produced data products, whether its ICOADS or something similar, these are not sanctioned data products like the National Weather Service, these are done under research funding and most people think of research funding being time limited, and often times the folks that develop these products are not the best marketers, and they're not very adept at or even equipped to trace how they are used...'' [ISC-04]
\end{quote}

In response to this decision, ICOADS project partners were able to secure letters of support from prominent international partners. The initial impact of this defunding has been minimal to end-users who have experienced little disruption on their access of ICOADS, but this controversy greatly delayed work on a third release, and stunted the development of a new valued added database (IVAD).

When asked about gathering support from the international community, a long-time project partner offered the following description:
\begin{quote}
``I think what happened was a funding cut came down from NOAA...and whoever was making that decision did not understand how important ICOADS is to the research community. How widely used it is ... because there was no statistics, there is no metrics of how many people would all of the sudden have their data products disrupted if ICOADS goes away, and that resulted in a very quick and fairly effective outpouring of letters of support, from major agencies around the world that used ICOADS...There were directors of major research institutes and operational institutions around the world that said `Hey, we need this product continued''' [ISC06]
\end{quote}

In the late fall of 2012 NOAA agreed to restore partial funding of ICOADS, but shifted financial support from the Earth Systems Laboratory (ESRL) to the National Climate Data Center (NCDC).  This event had four substantive impacts on ICOADS, and its governance model:

\begin{enumerate}
\item After the Winter of 2012 defunding announcement, project partners immediately began negotiating the terms of a ``Letter of Intent (LOI) to Enhance Support for the International Comprehensive Ocean-Atmosphere Data Set (ICOADS) Program through International Partnership.'' The letter of intent is a non-binding document, similar to a memorandum of understanding, which has no legally enforceable obligations, but instead spells out the terms of cooperation between project partners in order to, ``more formally recognize existing and planned international contributions that build on the ongoing investment in ICOADS.'' Signatories include: the National Oceanic and Atmospheric Administration (NOAA) National Climatic Data Center (NCDC), Cooperative Institute for Research in Environmental Sciences (CIRES) University of Colorado, National Center for Atmospheric Research (NCAR), Deutscher Wetterdienst (DWD), Center for Earth System Research and Sustainability - University of Hamburg, UK Met Office, Climatic Research Unit (CRU) - University of East Anglia (UEA), and the UK National Oceanography Centre (NOC). 

\item An international steering committee (ISC) was formed in order to establish ICOADS future goals, including `` new tasks and their prioritization; cooperation on identifying partner-specific experts for each task; and facilitation of expanded international cooperation.'' The ISC includes a representative from each of the signing institutions of the letter of intent. The steering committee is to have the following structure: led by a chair on a rotating basis between signatories, each member will hold a formal voting right (how voting is conducted, who calls a vote, etc. is not described), and institutional entrance or exit from the committee will require a 30 day notice to other members. 

\item The letter also formalizes roles that each project partner should take on. These roles are defined in agreement with a set of ICOADS project goals, which include ``cooperation and collaboration on the rescue, digitization, assembly, processing, quality control (QC), and archival of surface marine meteorological and oceanographic data, for the express purpose of making such data openly and freely available. This availability will be accomplished primarily through the joint completion and issuance of major new ICOADS Releases, and intermediate products, together with their accompanying technical documentation and journal publications.'' (2013) These activities could describe each ICOADS regime, but the importance of these formally stated goals is that individual partners are assigned specific tasks in completing or contributing to these development efforts.

\item The move from ESRL to NCDC greatly impacted the dynamic of the ICOADS's leadership. The longtime PI of the project, based at ESRL, moved to part-time, and has since announced he will retire in Spring 2015. Additionally, three key resources - a full time programmer, remote-access systems for NetCDF users, and a set of virtual servers - were not replaced in the shift of responsibilities from ESRL to NCDC. All of this greatly impacted the speed at which work on a third release could proceed. 
\end{enumerate}

\subsubsection{CMOC}

ICOADS partners, after signing the letter of intent, were encouraged by JCoMM to seek out status as a Centre for Marine-Meteorological and Oceanographic Climate Data (CMOCs) from the WMO. As a CMOC, ICOADS could be recognized formally by JCoMM and the WMO as an essential research infrastructure for the weather and climate data enterprise. This would encourage a greater number of international partners to formally sign ICOADS letter of intent for cooperation, and allow project partners to more easily appeal to national funding agencies for financial support to contribute to ICOADS development. In my interviews with ISC board members, status as a CMOC was widely endorsed but often came with caveats of the drawback to further entrenching the ICOADS in the bureaucracy of the WMO. A participant explained the tradeoffs in applying for CMOC status as follows: 
\begin{quote}
``...one of the drawbacks to getting quote unquote recognized by JCoMM and WMO is they have a very heavy bureaucracy, they have a very heavy bureaucratic structure because JCoMM and WMO are both a part of the United Nations... the diplomatic structures there, it takes a long time to get things done... a single document can take years to get approved. One of the pluses, at least this is the way we see it, if you are recognized by this international body as being an important part of the either the climate or weather infrastructure, that data infrastructure gives you leverage in your individual countries to get dedicated resources.'' [ISC-01]
\end{quote}
Another drawback is that the WMO's Vision for a Marine Climate Data System (MCDS) is relatively new, and there is little administrative support for the initiative within JCoMM. As such, ICOADS would be the first CMOC recognized by the WMO. Many participants expressed concern that it is unclear what responsibilities the status will hold within the WMO, and what kinds of resources will be available to CMOCs for impacting broader WMO policy. 

\subsubsection{ICOADS Release 3.0}

The letter of intent signed by project partners emphasizes the following point, ``By signing, partners agree to work closely with the common goal of enhancing and internationalizing ICOADS, including the near-term goal of completing the next major delayed-mode update, Release 3.0.'' 

ICOADS partners had been planning the next full data release since 2011 as a result of  new sources of data had became available through the  Atmospheric Circulation Reconstructions over the Earth (ACRE) project, and dedicated supercomputing time at NCAR would become feasible for records processing. The latter is the result of a new supercomputing center (Yellowstone) coming online, which in-turn would free up existing high-performance machines. The defunding of ICOADS came at an inopportune time as developers at ESRL were planning to improve IMMA software for formatting ICOADS 3 data. Work on blending contemporary GTS data continued, but the delay in a new software package for IMMA formatting greatly delayed the re-processing of older records. Originally scheduled for completion in December of 2014, ICOADS partners have now targeted early 2016 as a release date. 

Release 3 will include a number of major improvements, but there are two important innovations to note: 

\begin{enumerate} 

\item For the first time, unique report identifiers (UID) system will be used to assign ids to individual records in ICOADS historical archive. This means that all 295 million records would be serialized, and will be more easily accessed by end users. Further, the scheme adopted for UIDs is also in use by other community projects (upper-atmosphere datasets), and is intended to greatly reduce the burden of integrating ICOADS in future reanalysis projects (Freeman et al, 2014). 

\item In 2011, the design of a new product called the ICOADS Value-Added Database (IVAD) was funded by NOAA’s Climate Observation and Monitoring Division. The concept is a novel design for aggregating different bias corrections, and improvements in the ICOADS archive so that users can ``select'' which corrections they want at a point of download. IVAD is an important scientific innovation, but this product also changes the nature of ICOADS division of labor in producing new resources. 

In the IVAD model, end-users will be able to follow an established protocol to submit bias corrections, and volunteer to do peer review for submitted corrections. Eventually these new datasets will be integrated into the existing archive such that an end user, for instance, looking for SST could pose a query, and get bias corrections with different features, select which features were important to her, and then download original and processed records side-by-side. This moves ICOADS from a real-time ``develop and release''dataset, to a peer-produced resource that improves incrementally with direct user contributions. 
\begin{quote}
``this process... I mean, this is very unique - this is a concept that has not been implemented for the in-situ observing systems. Most of the datasets out there - the data quality is asssesed by a single organization. And they put the quality flags on data points, they decide whats a good value and whats a bad value - not always very clear either. There is not a whole lot of feedback loop on these products.''
[ISC 03]
\end{quote}

IVAD should sound similar to open-source software projects, as I was told a number of time that this was partially the inspiration for having contributors feedback their ``patches'' to the project. In some sense, IVAD will turn historical records of ICOADS releases from a primary resource to a ``kernel'' - one which can be developed around and on top of. Like the governance of the second regime, the impact of IVAD on the ICOADS project is that of nesting one system inside another. In this case, ICOADS will further establish itself as a platform on which others develop and innovate with applications that make it more accessible, and easier to use in different contexts, by different users. I return to the larger implications of this nesting process in the following chapter. 

\end{enumerate}

\subsubsection{Review of Regime 3}

Regime three includes the very recent past of ICOADS governance, and its on-going work to develop an innovative new release. Although much shorter than the two previous regimes, this period has two significant implications for an evolving ICOADS governance model: 

\begin{enumerate}
\item A letter of intent, signed by seven international project partners, establishes a number of formal governance mechanisms, including policy instruments and positional rules. 
\item The design and integration of IVAD will dramatically change the production process of ICOADS, and in turn will create a new set of benefits and costs for the governance model to balance. 
\end{enumerate}


\section{Summary}

In this chapter I have summarized the differences between the IAD and KFC, and justified my innovation with both of these frameworks to present an analysis of the ICOADS case study data. I then provided background information important to understanding the context in which ICOADS was developed, including a number of formal policies at the international level, and failed legislation at the national level. I then offered an overview of ICOADS various resource sets, and community member roles. Finally, I justified the division of the ICOADS case study into three separate governance regimes, offered an analysis of how rules, and governance variables were effected by the ``action arenas'' of each regime - including my participants (ICOADS Steering Committee members) own perspectives about these events. In the next Chapter I summarize the major findings of this analysis and answer the research questions formally stated in Chapter 2.  



