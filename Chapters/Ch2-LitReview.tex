\chapter{Chapter 2}

\section{Introduction}

\emph{This chapter reviews relevant literature on governance and sustainability of shared resource systems, including the various conceptual models which inform commons theory. I focus in particular on aligning the rules-based approach of socio-ecological systems with those of sociotechnical systems.}\\

\emph{I begin with an overview of economic theories of organizing around shared resource systems. I then describe the emergence of commons theory, and its relationship to sustaining socio-ecological systems. Finally, I relate previous research on the natural commons to emerging notions of the knowledge commons, which brings with it new models of peer production, and use of networked information communication technologies.}
	
\section{Historical Roots}

The first half of the twentieth century was dominated by an assumption that there were two major models of economic organization: the market and the state. Classical economic thinking holds that the government (state) should be responsible for provisioning the types of public services that may go under-produced if left open to pure marketplace competition. In contrast, the market should be as unregulated domain where pricing signals direct self-interested actors exchanging goods and services to satisfy their preferences. In short, the market is the domain of production and exchange between bakers, butchers and brewers; and the government is the domain of provision, such as those required for lighthouses, railways and infrastructures to effective operate. 

\subsection{The Market: Coase, Williamson, and Transaction Economics}

In 1937 Ronald Coase began to upend common assumptions about economic efficiency of organizing in the marketplace. This began with a publication titled "The Nature of the Firm"; a short, and largely anecdotal article in which Coase asks a simple question: Why do firms - hierarchicaly organized models of production-  exist if markets are efficient for satisfying individual preferences? His explanation follows:

...an alternative form of economic organisation which could achieve the same result at less cost than would be incurred by using the market would enable the value of production to be raised...the firm represents such an alternative to organising production through market transactions...\emph{Within the firm individual bargains between the various cooperating factors of production are eliminated and for a market transaction is substituted an administrative decision. The rearrangement of production then takes place without the need for bargains between the owners of the factors of production.} (Coase, 1932)

According to Coase, the relationship between firms and markets is as follows - when the cost of achieving an outcome is cheaper by organizing activities in a hierarchy, firms will emerge, but when the marketplace creates a pricing system with less costs, individuals (or small institutions) will follow pricing signals. This explains why, even self-interested actors can be observed cooperate within free marketplace settings. Coase's insights turn questions of efficiency in economic organization from how to achieve absolute market optimization, to which model of organizing economic activity could create the greatest wealth. Economists interested in the private sector should therefore focus on how choices between such alternatives are made. Further, Coase turned marketplace behavior from a theoretical problem, to an empirical one by showing the market vs. firm distinction is an organizational phenomena. As Coase put it, the task for those interested in the organization of economic systems of production is this: 

...to bridge what appears to be a gap in [standard] economic theory between the assumption (made for some purposes) that resources are
allocated by means of the price mechanism and the assumption (made for other purposes) that that allocation is dependent on the entrepreneur-coordinator. We have to explain the basis on which, in practice, this choice between alternatives is effected. 
(1937, p. 389)

 The focus of this line of thinking then became the grounds for transaction cost\footnote{In this work, the unit of analysis is a transaction, defined as ``the three principles of conflict, mutuality, and order. This unit is a transaction'' (Common, 1932, p. 4).} economics" which Oliver Williamson would go on to use for a predictive theory of economic organization, which he described was meant to ``breathe operational content into governance and organization'' of production systems, where governance could be defined as ``...the means by which to infuse order, thereby to mitigate conflict and realize mutual gain...'' (Williamson, 2010). 

The research agenda of transaction economics then required reformulating the problem of economic organization, from the individual behavioral explanation of rational actors, to one built on comparative contractual terms. This was accomplished by the following steps:  

\begin{itemize}
\item Naming the key attributes with respect to which transactions differ, 
\item Describing the clusters of attributes that define alternative modes of governance (of which markets and hierarchies are two), 
\item Joining these parts by appealing to the efficient alignment hypothesis, wherein 
\item Predictions would be derived to which empirical tests would be applied and 
\item Public policy ramifications could be worked up. (Williamson, 2010)
\end{itemize}

In parallel with Williamson, Elinor Ostrom was working to ``breathe empirical life'' into the governance assumptions that classical economics had made about the provisioning of environmental goods, especially community managed resource systems. This research agenda took on assumptions that solutions to the provision of public goods, those typically the domain of the state, were most efficient when organized by a central authority. Through a diverse body of work that included the study of police departments, watersheds, fisheries, and forests, Ostrom and colleagues showed that, like Coase and Williamson, classical economic ideas about efficiency rarely matched reality. What the parallels in these two research agendas teach scholars of political economy, in any domain, is that understanding attributes of governance, how they differ in efficient application, and the policies that can be consequently derived, will require a systematic empirical approach that can be applied across cases. 

\subsection{Section Summary}

Both of these ideas - the emergence of the firm, and transaction cost economics - were critical for the study of alternative institutional arrangements and governance mechanisms in the marketplace. First, Coase showed that it was not by happenstance, nor ideology that drove organizational structures to adopt one governance regime over another; instead this was the result of individuals trying to efficiently solve marketplace dilemmas by assessing costs and benefits related to following a hierarchy, or following pricing signals. Later, Williamson demonstrated that an analysis of transaction costs could be systematized, making the study of contracts and institutions more useful to public policy decisions that regulate marketplaces.  

I turn next to the debates about property and the organization of governance regimes for the provisioning of public goods; the controversies out of which commons scholarship was borne. The next two sections trace the intellectual history of the commons. I then describe the influence that new modes of production, namely peer-production, are having on organizing shared resources systems in order to establish the commons based research agenda to which the findings of this dissertation will contribute.  

\subsection{The Tragedy of `'The Tragedy of the Commons'}

In 1968 Garret Hardin the bioligist penned a short, but highly influential article in `Science' that described a hypothetical dilemma facing herdsman in the shared resource system of an open pasture where consumtpion is unrestrained. As he writes, 

\begin{quote}
Each man is locked into a system that compels him to increase his herd without limit - in a world that is limited. Ruin is the destination toward which all men rush, each pursuing his own best interest in a society that believes in the freedom of the commons.
Freedom in a commons brings ruin to all. \citep[p. 1244]{hardin1968tragedy}.
\end{quote}

 Hardin's  `Tragedy of the Commons' is a standard externality problem that is the result of failed collective action, built on an assumption that resource users act independently, and of their own self-interest. The solution set that Hardin offers consists of two approaches from classical economics: Enclosure, in which the land is privately parceled out for individual ownership, or management by a centralized state government which could prohibit overgrazing through regulation. 

A number of scholars have since pointed out, Hardin's scenario is flawed in a number of ways; the resource system he describes is not necessarily a commons, but an open-access regime, and the assumptions made about rational actions independent of collective conscious turns out to be greatly oversimplified when compared with real world scenarios (Taylor, 1990). But, Hardin was correct in recognizing the troubles that befall shared resource systems when communication between users is limited.

To better understand shared resource management, and how it operates in practice it is necessary to first review the classification of goods based on two dimension: their rivalry and their subtractability. Then, by looking closely at the governance regimes that are adopted to manage access and use of goods across this classification, it will become both clearer why the tragedy of the commons scenario is flawed, and what is really at stake in the design and adoption of different governance models for shared resource systems. 

\subsection{Classifying Goods: From Samuelson to Ostrom}

The classical economist Paul Samuelson was the first to formally describe differences between private and public goods. He was careful to point out that although his examples represented two extremes most government activity could be analyzed as some blend of two poles: private and finite, or public and infinite (1950). Samuelson arrived at these two distinctions by way of a theory about rivalrous vs. non-rivalrous commodities. A good with rivalry can be owned or consumed by a limited number of individuals - the good is finite in this sense, and its value is largely a function of the limited nature of its ownership. A good with no, or even low degrees of rivalry can be shared among a large number of individuals without the value of the good being degraded - its value then is a function of its accessibility and shareability. 

Initially, it was rivalry that concerned political economists thinking about the production and provision of goods. Over time, Sameulson's second dimension was more carefully considered, and it it is from this body of literature that common pool resource theory emerged (Agrawal, and Boettke, 2011). In the second division of goods by type, the question theoretically turns on whether or not it is economically efficient to create barriers (exclusion) to the consumption of a resource. 

A swimming pool is an excludable resource where it is economically efficient to bar entry so that a select (proper) number of swimmers can enjoy the pool. The ocean, on the other hand, is a resource for which it is economically inefficient to prohibit swimming access. Artificially creating a barrier to prevent use of an Ocean by swimmers or sailors would be prohibitively expensive for each nation of the world, and no one person has any greater claim of right to access the ocean than any other. The resource therefore remains relatively non-excludable \footnote{Beaches of course have swimming rules, and there are boundaries erected for the safety of swimming in open waters, but the larger point is that creating barrier to entry, for sheer size and right to access, is prohibitive}.  

\includegraphics[width=4in, height=2.5in]{CommonsMatrix.png}

Each of these categorizations have to do with resources (goods) as property - where some can have exclusive ownership, while others can have the power of ownerships symmetrically distributed (collectively shared) by a community of interested parties. Commons - including open access commons described by Hardin - are marked by the absence of asymmetrical power to determine the rights of the resource. In other words ``they are marked by privileges and immunities for an undefined public, rather than rights and powers for a defined person or persons.'' (Benkler, 2014 p. 3). Hardin's argument in the tragedy of the commons confused the type of resource with its governance regime. Hardin described a completely open access public goods regime used to manage common-pool resources, claiming that the failure was the result of self-interested individuals incapable of acting in a way that did not maximize their own self-interest. 

Elinor Ostrom pioneered the study of ``common pool resources'' (CPRs), or goods with rivalrous and non-exclusive access such as grazing pastures, aquifers, or forests (1990). CPRs are rivalrous because they are depleatable, but were difficult to exclude access, either because of the size of the resource or the consumer's claim of a right to access - such as the Ocean.   Over three decades worth of work, Ostrom and the ``Bloomington School'' of scholars convincingly showed that there is not a one fits all solution for governance (Agrawal and Boettke, 2010). Common-pool resources can just as often be successfully managed by communities with bundled legal rights (a commons) as they can with formal government intervention. But, a repeated finding of this work is that for small, well-bounded common-pool resources, a community-based self-governance model outperforms a state model, especially in negotiating terms for long-term sustainable collective action (Ostrom, 2010). 

In short, there is no governance panacea; a single property-rights regime can produce, and effectively provision several kinds of goods, and several kinds of goods can be effectively produced or provisioned under different regimes (Hann 1998; Berge 2002; Benda-Beckman et al. 2006). What works in one domain may fail spectacularly in another. The questions that result from this realization are twofold:

\begin{enumerate}
\item Which type of governance regimes most effectively provision the types of goods or services which require collective action (e.g. public goods, common-pool resources, infrastructures, etc.)?
\item How do novelties in the design of these institutions impact the ability to provision resource sets? That is, if context matters, what effect does it have on the design similar commons in different contexts, and different commons in similar contexts. 
\end{enumerate}
Across the different contexts that she studied, Ostrom's work establishes five different kinds of communal property rights:  

\begin{enumerate}

\item Access – the right to enter a defined physical area and enjoy non-subtractive benefits (e.g. hiking, skiing); 
\item Extraction – the right to take products of a resource system (e.g. to fish, divert water);
\item Management – the right to regulate use patterns and make improvements on the resource;
\item Exclusion – the right to determine who will have access and withdrawal rights;
\item Alienation – the right to sell or transfer management and exclusion rights. Varying combinations of these rights are associated with statuses. Owners have all of these rights; authorized users can only access a property and withdraw resources.
\end{enumerate}

In most commons settings these bundled rights are established through rules, some formally and legally enforceable and others informal and community policed. Ostrom draws a distinction between these two rules - those written down and codified being the ``rules-in-form'' and those unspoken, tacit, and community enforced being the ``rules-in-use''. Critical to understanding the proper functioning, and long-term success of commons then is finding ways to observe rules-in-use for effectively dealing with social dilemmas, such as collective action problems found in the `Tragedy of the Commons'. Michael Taylor offers a succinct definition of collective action, as follows ``Collective-action problems occur when there is a divergence between the interests of the individual and those of the society. In these cases, it is not rational for individuals to cooperate, even though cooperation would bring positive results for all.'' (Taylor 1990) In some sense - this is the broad economic concern for efficiency - asking what system, arrangement of resources, or policy intervention creates the greatest good. But collective action is a micro-economic concern, because it focuses on how these types of dilemmas are solved amongst individuals with competing interest. These solutions, especially in the context of environmental common-pool resources, can be attributed to rule types that establish and in some cases prescribe a number of different expectations for actors in the commons:

\begin{itemize}

\item Boundary rules specify how actors are to be chosen to enter or leave a situation

\item Position rules specify a set of positions (council members, president, etc.) and how many actors hold each one

\item Information rules specify channels of communication among actors and what information must, may, or must not be shared. 

\item Authority rules specify which actions are assigned to a position at a node

\item Aggregation rules (such as majority or unanimity rules) specify how the decisions of actors at a node are to be mapped to intermediate or final outcomes

\item Scope rules specify the outcomes that could be affected

\item Payoff rules specify how benefits and costs are to be distributed to actors in positions. (McGinnis, 2011)

\end{itemize}

In addition to rule sets, Ostrom and colleagues synthesized many studies of CPRs throughout the world by way of a systematic frameworks for both gathering and analyzing empirical data. The results from this synthesis are a set of design principles for shared resource systems that, at a very high level, account for the features which often (but not always) lead to commons success: 

\begin{itemize}
\item Clearly defined boundaries should be in place.
\item Rules in use are well matched to local needs and conditions.
\item Individuals affected by these rules can usually participate in modifying the rules.
\item The right of community members to devise their own rules is respected by external authorities.
\item A system for self-monitoring members' behavior has been established.
\item A graduated system of sanctions is available.
\item Community members have access to low-cost conflict-resolution mechanisms. 
\item Nested enterprises — that is, appropriation, provision, monitoring and sanctioning, conflict resolution, and other governance activities—are organized in a nested structure with multiple layers of activities.
\end{itemize}
	(Ostrom 1990, 90–102)

According to Ostrom (1999a, 2000a), community governance is most successful in sustaining shared resources when social capital is symmetrically distributed (i.e. few gatekeepers), communities are small, local, well mapped geographically, and those that are most dependent on the resource have a what is known as a ``low discount rate'' or a ``willingness to sacrifice current payoffs for higher payoffs in the future.'' (Acheson, 2011). 

\subsubsection{Comedy of the Commons}

While Ostrom and the Bloomington School thrust commons theory into the academic spotlight during the early 1990's, it is often assumed that the social dilemmas of common-pool resources are the social dilemmas faced by all commons. The types of commons that deal with socio-ecological systems (pastures, forests, aquifers, huertas, etc.) often face dilemmas that are based on excludability - such as the free rider problem which emerges in open-access regimes (e.g. Acheson, 1990), or the zero-contribution thesis of collective action which occurs in large cooperatives (Olson, 1965). But, excludability is only one type of social dilemma faced by commons. In commons that include the production and provision of public goods, which are often assumed to be efficiently provisioned by governments, problems of congestion (overuse) and free-riding emerge that are similar to, but different than those faced by CPRs. 

Work that addresses these dilemmas began with Carole Rose, a lawyer, in the late 1980's writing about the "inherent public property" of a certain class of goods - such as waterways, the atmosphere, and public beaches. Rose asked how, and to what degree, common law establishes a set of rights for public access to these resources (1986), recognizing that it is both costly to exclude users, and that in the absence of congestion these types of goods create positive spillover effects. In short, public goods are the infrastructure for a number of downstream innovations - and the freedom to operate within those infrastructures should be inherent, not just by virtue of their being large and difficult to exclude access, but in their governance. Rose coined a phrase to refer to the spillover effects of these common rights as a "comedy of the commons''; and she then went on to show,  through a number of case studies, how in distributing access and usage rights of public goods to individuals made resource users complicit in the effective provisioning of the shared resource system. As Rose argues, investment in such coordinating costs would be prohibitively expensive using either a market or state based model of organization. It was only through a commons, with symmetrical rights to operate effectively shared by all resources users, that such cooperation could be sustained. 

\subsubsection{Section Summary}

In the following sections, I explore commons outside of the CPR context. I look specifically at how governance regimes for sustainable commons evolve with the introduction of networked information communication technologies, focusing on science and technology research settings. Much of this work is predicated on the idea that for informational goods, collective action problems emerge around both the production and provisioning of goods. Where scholars like Ostrom, Acheson, and Rose are concerned with resource sets that are already in place, other commons which are produced, such as open-source software (Schweik and English, 2012); Genomic data archives (Contreras, 2012); and Telecommunications infrastructures (Frischmann, 2005) face dilemmas of collective action at both a point of initial production, as well as sustaining and provisioning the resources over time. In these domains, the initial terms upon which cooperation is founded play an important, and non-trivial role in the structure of later governance models (O'mahony, 2007). 

To conclude this section, I've described three things: 

\begin{enumerate}

\item A widely used categorization of resource (goods) based on two criteria: Rivalry and Excludability

\item A distinction between the classification of property types (goods), and governance; and, how confusions in this distinction has impacted the historical conceptions of the commons.  

\item The empirical research agenda launched by Elinor Ostrom to study the governance of commons, including 
CPRs where social dilemmas related to excludability often emerge, but are sustained through establishing and enforcing rules of two types: Rules-in-Use, and Rules-in-Form.       
\end{enumerate}

\section{Commons Unmodified}

Rose's work on public goods was an important precursor to commons theory developed for networked communication technologies (Benkler, 2001), and public infrastructures (Frischmann, 2005) as well as informational goods such as creative works in the public domain (Lessig, 1999) or open-source software (Lakhani and VonHippel, 2003). 

Yochai Benkler provides an framework for distinguishing between Rose and Ostrom's commons: 

\begin{itemize}
\item
\textbf{Modified} commons are aimed at describing the types of resource systems studied by Ostrom et al - physical, well bounded, and long-enduring. Pastures, aquifers, huertas, are the quintessential case of the modified commons.  
\item
\textbf{Unmodified commons} refer to public goods, which share the a of excludability with modified commons, but also lack rivalry in their consumption. Environmental unmodified commons include beaches, open-roads, the Atmosphere, and Oceans. In the informational sphere, unmodified commons are the platforms and infrastructures upon which many facets of the modern economy are based; examples include the broadband spectrum, Internet exchange protocols, suites of open-source software, and weather data. 
\end{itemize}

For modified commons studied by the Bloomington School, the following were generally true:

\begin{enumerate}
\item Resource boundaries were clear. 
\item Resource systems were small and easy to observe
\item Solving social dilemmas was of high importance to appropriators (funders)
\item Institutions were long enduring (centuries), and evolved rules and sanctions over equally long periods of time. (Hess and Ostrom, 2005)
\end{enumerate}

Nearly each of these features are different in an unmodified commons. They are typically large systems with unclear boundaries.Inforamtoion-based unmodified commons, or knowledge commons,  often form rapidly, and without plans for long-term maintenance in mind. This is especially true of the types of research and development in science, which is typically funded through five year grants of government subsidy (Mirowski, 2011). Finally, solving social dilemmas is of high importance to appropriators, but the size, lack of clear boundaries, and the rapid formation of unmodified commons greatly complicate this process.

\subsubsection{The Knowledge Commons, Unmodified}

Recall, from Chapter 1, that knowledge commons - including resource sets made up of informational goods - are to sociotechnical systems, as the natural commons - containing environmental resources -  are to socioecological systems. Given these differences, two other important distinctions mark the unmodified knowledge commons, and I argue that both of these features are critical to understanding the role of governance for sustaining shared resource systems of this variety:  

\begin{enumerate}
\item Unmodified knowledge commmons,  are almost always purposefully engineered systems. That is to say, they are designed and built instead of naturally occurring. Simon's treatise on a `'science of the artificial' is an important antecedent to this thread of unmodified commons work. Simon draws a distinction between what he calls science as an analysis of nature, and engineering as synthesis of the artificial (1981). His distinction is aimed at supporting a thesis about design, which explains that artificial phenomena ``are as they are only because of a system's being modeled, by goals or purposes, to the environment in which it lives. If natural phenomena have an air of `'necessity' about them in their subservience to natural law, artificial phenomena have an air of ``contingency'' in their malleability by environment.'' (1981, p. X)

According to this distinction, artificial things differ from nature in the following ways:

\begin{enumerate}
\item Artificial things are synthesized (though not always or usually with full forethought) by human beings.
\item Artificial things may imitate appearances in natural things while lacking, in one or many respects, the reality of the latter.
\item Artificial things can be characterized in terms of functions, goals, adaptation.
\item Artificial things are often discussed, particularly when they are being designed, in terms of imperatives as well as descriptives. (Simon, 1981 p. 13-17)
\end{enumerate}

The purposeful design, and implementation of sociotechnical systems brings with it - as Simon notes of the artificial - a need to attend to the normative implications of those built environments. To analyze the effectiveness of governance in the unmodified commons, one needs to also be concerned with a set of values, or a background environment in which ``artificial things'' are designed and built.

Additionally, Simon notes that artificial things can be characterized by their functions, goals, and adaptation to contexts. Artificial things, like informational goods, are abiotic, but they nevertheless evolve in a context of use. One of the central concepts of Chapter 5's analysis is that like biotic things, selective pressures play a major role in how artificial things are adapted to meet those needs. I argue that an account of this brand of commons then needs to attend to the intentions of design, and to the adaptation of artificial things, in use. Similar to Ostrom's rules, Simon's point about artificial objects having functions and adaptations within different contexts creates a distinction between ``things-in-form'' and ``things-in-use'' - and as Simon instructs, one understands the ``evolution and future of a system through its history.'' (1981, p. 56) 

\item Related to the normative implications of design and engineering, there is a blurring of traditional user, producer, or consumer roles within the unmodified commons. This is especially true within shared resource systems related to informational goods, where resource flows are no longer bi-directional (producer to consumer), but are instead n-dimensional; producers of an informational good in one context often become users of the same resource in a different context. Producers are often working to provision goods, and users are often working to produce goods. Within contemporary sociotechnical systems many of the same actors are playing many different roles.. For instance, IBM both produces software that is proprietary and is simultaneously one of the largest contributors to (provisioner's of) open-source projects in the world. It provisions certain open-source libraries by providing labor and funding for their maintenance, and simultaneously it uses those libraries in its own distribution of software packages (O'mahony & Lakhani, 2011). This breakdown in traditional roles requires analysis of the functioning of an unmodified commons over time, and across different levels and scales of analysis. For instance, one can't simply look to measures of informational inputs and outputs of a sociotechnical system in order to understand whether a balancing of those stocks is sustainable. [Needs some further fleshing out **]

Many studies from informatics (e.g. Wiggins and Crowston, 2011) and social network analysis (e.g. Borondo et al. 2013) have shown a defining characteristic of contemporary scientific work is that the traditional boundaries between user and producer are blurred. Stakeholders both use pooled resources, and contribute to the provisioning of those resources over time (Benkler, 2006). These blurred distinctions result in a number of social dilemmas, such as the continuation of a ``Matthew Effect'' in hyper-authorship networks (Glänzel and Schuber, 2005), identity disambiguation in the process of attributing knowledge claims (Fegley and Torvik, 2013), large collaborations that do not have a clear chain of ownership for intellectual property (Reichman and Uhlir, 2003), and the distributed nature of most eScience infrastructure development, which complicates a clear distinction between production and provision (Fry et al, 2011). Often the most successful collaborative projects include scenarios where an actor is enabled to modify, change, or improve a given technology for his or her own research – while simultaneously making those modifications, changes and improvements available for others to use.

\end{enumerate}

Purposefully built systems containing ``artificial things'' that evolve through use; and the collapse between traditional roles of economic actors in producing, using, provisioning informational goods are what marks sociotechnical systems as a unique branch of the unmodified commons research agenda. Next, I turn to reviewing literature which further describes social dilemmas that result from these unique factors, and the ways in which institutions successfully overcome these dilemmas. I focus specifically on previous studies of sustainability in science and technology settings., and conclude by describing the emergence of peer-production and its relevance to the research project at hand.  

\section{Sustainability}

The socioecological systems literature contains a number of different conceptualizations of sustainability, three of which I'll review here: resource sufficiency, functional integrity, and for lack of a precise term the ``normative account''. I describe these approaches in order to characterize contemporary research on the sustainability of sociotechnical systems found science and technology studies. 

\textbf{Resource sufficiency} is a technical approach to environmental stewardship that is generally concerned with two measurement problems related to sustainable practices. The first measurement problem is to accurately capture the rate of consumption of a resource. The second measurement problem is concerned with estimating the stock of available resources. To characterize sustainability of a socioecological system through resource sufficiency is to match consumption practices with rates of stock regeneration.

 \textbf{Functional integrity} is a systems-based approach to sustainability practices. In this paradigm, a ``practice that creates a threat to the system's capacity for reproducing itself over time is said to be unsustainable'' (Thompson 1997). A functional integrity approach requires modeling reproduction cycles, and the practices that effect these cycles over time. Functional integrity could also be conceived of as risk management approach to systems sustainability.  

Comparing and contrasting these two approaches: resource sufficiency is utilitarian in that it conceives of sustainability as an outcome of balancing resource stocks and flows (Moore, 2011). Functional integrity leans towards a communal, albeit cautious, view of risk in relation to the stability of broad systems, including social, economic, political, as well as environmental. 

A third approach offered by both environmental economists and philosophers is to extend the systems thinking of functional integrity to a normative, values-based model. A normative approach emphasizes the impact that production and provisioning - otherwise grouped together as development practices - have on the environment, society, and the economy. An example of this approach is as follows, ``We define sustainable agricultural development in this paper as an agricultural system which over the long run, enhances environmental quality and the resource base on which agriculture depends, provides for basic human food and fiber needs, is economically viable, and enhances the quality of life of farmers and society as a whole.'' (Davis and Lanham, 1995, p. 21-22). The normative approach is unique in that it conceives of a target that sustainable development practices or policies \emph{should} be aimed.

\subsubsection{Sustaining Sociotechnical Systems}

In the following section, I map these three notions of socioecological sustainability to contemporary work on sociotechnical systems sustainability. Literature reviewed in this section comes from the field of science and technology studies (STS_, and Computer Supported Cooperative Work (CSCW). 

\emph{Ribes and Finholt - 2009} 	

Ribes and Finholt studied the design and deployment of ``e-infrastructure'' by doing a longitudinal cross-case analysis of the daily work practices of ``sustaining research facilities'' in universities and federally funded research centers (FFRDCs) (2009). They describe the problematic nature of research funding cycles that are misaligned with sustainability work, saying in particular that “long-term infrastructure is primarily an institutional consideration, beyond the scope of any single project or discipline.” (2009 p. 377) Their findings are broadly construed as implications for the funding of future cyberinfrastructure work, as they describe the various “tensions” in the deployment of a technology, and its maintenance over a relatively short period of time (2 years).

Overall, this study uses a \emph{resource sufficiency} approach to systems sustainability. The authors describe how resource flows, constrained by financial support from the USA government, impact stocks of knowledge and infrastructural maintenance required for a sustainable research enterprise.  

\emph{Lee and Bietz}
A series of studies by Bietz and Lee explains the ``work'' of sustaining genomics cyberinfrastructures, which they describe through the concept ``synergizing.'' Sustainable sociotechnical systems in this domain are made possible through, “...work that developers of infrastructure do to build and maintain productive relationships among people, organizations, and technologies.” (2012) Bietz and Lee’s concern for invisible or overlooked work in genomics database maintenance emphasizes the emergent phenomena of a complex system. As such, they conclude that CI sustainability is ``less about maintaining any particular technology than it is about being prepared to accommodate technological, scientific and organizational change'';  accommodating this change is best achieved when maintaining systems level functionality (2012, p. 8). They briefly touch on the flexibility in the rules developed by the genomics community to govern shared resources, but do not tie these to sustainability per se. 

Lee and Bietz conceive of sustainability as \emph{functional integrity} of a system, acknowledging that ``emergent'' phenomena within a work-group are impacted by removing one or another member. The authors stress the need for accommodating change by identifying disruptive forces within a cycle of knowledge production, and suggest that the work of adaptation be labeled ``synergy''

\emph{Kee and Browning }
Kee and Browning focused on the funding infrastructure of developing new, shared high-performance computing centers for scientific applicaitons (2010). They propose to treat these issues as “dialectical tensions” , with  “five sets of seemingly opposing forces on three levels of institutions, individuals, and ideologies” (2010, p. 285) Kee and Browning’s conclusions are that the longevity and sustainability of an infrastructure is compromised “when it is based on short-term commitments and part-time attention of a distributed group of technologists, its progress is compromised.”  (2010 p. 286) Their proposal is conduct further work on the long-term, historical implications of such small funding cycles. 

Like Ribes and Finholt, this study takes a \emph{resource sufficiency} approach to sustainability. The author's focus on balancing inputs to a sociotechnical system, such as dollars in grant funding and private sector support, with tangible, measurable outputs. The ``dialectical tensions'' are positioned as naturally occurring between desires to increase one set of stocks or flows over another. 

\emph{Vertesi and Dourish} 
At NASA's Jet Propulsion Laboratory (JPL) Vertesi and Dorusigh studied two research teams working on the same spacecraft mission. This work produced a cross-case comparison of data sharing and resource pooling. They describe a number of sociotechnical features that complicate the open exchange of data, including an analogy to different types of data having currency within a larger research ``data economy'' (2011). Their findings emphasize a need to develop systems with a concern for “data infrastructures that reflect and secure data sharing practices specific to each scientific collaboration.” (2011, p. 9)

Vertesi and Dourish are somewhat unique in this set of studies in that they focus on a specific target for whom a system should be sustainably developed. They are concerned with how policy aimed at increasing data sharing in federal laboratories will impact the sustainability of effective collaborative arrangements at JPL. But, in practice this work resembles a \emph{functional integrity} approach to sustainability by focusing on cycles in which data are produced, and how the practices and policies of managing that data then impact long-term collaborations. 

\emph{Fry, Schroder and den Besten}
In a series of “in-depth interviews” with principle investigators of eScience projects in the UK, Fry, Schroder and  den Bensten investigated “research governance at the institutional level and local research practices at the project level.” (2008, p.6) Their work asked research questions about both the sustainability of eScience infrastructure models, and further, to what extent they promoted openness in sharing research products and collaborating. Their findings indicate that, “The fundamental challenge in resolving openness in practice and policy, and thereby moving towards a sustainable infrastructure for e-Science, is the coordination and integration of goals across e-Science efforts, rather than one of resolving IPR (Intellectual Property Rights) issues, which has been the central focus of openness debates thus far.”  (2008, p.6). The authors note that they identified a number of conflicts between contractual governance at a micro-level and project settings (institutional contexts) at a macro-level – which introduced uncertainty for the sustainability and trust between different collaborators. 

Fry, Schroder and den Besten are engaged in a \emph{normative} account of sustainability, as they conclude with a discussion of different approaches to evaluation between participants engaged in science research, and their funding agents; “Such a schism may be because contractual and institutional arrangements in science have traditionally focused on final products (or outcomes) of projects such as publications, compounds or genes, rather than engaging at the level of knowledge creation. This is a particular issue for e-Research as there is more of an impetus to disseminate and share by-products such as software code and data not used in final results.” (2008, p. 23). They are concerned not just with target communities of sustainability, but with an approach that sees sustainability in terms of justice for future generations of scientists that will need to draw upon a stock of knowledge produced by this work. 


\emph{Synthesis of Studies}

Socioecological studies, even when conceptualizing of sustainability differently, are able to accomplish integration of results for two reasons: 1. They can rely on well-accepted quantitative measures, such as landings yields (fisheries) or wood density (forests), and 2. Over time, the sub-domains that study these different systems have developed ways to conduct meta-analysis by way of systematic frameworks that guide the collection of new data, or re-analysis of existing data. 

The first approach to integrating results of sociotechnical systems sustainability will be difficult given a lack of reliable quantitative measures for their performance. None of the studies reviewed above have a quantitative component, and it remains unclear how sustainability might be correlated with metrics, even exploratory metrics (e.g. Bollen et al, 2010), for evaluating performance of these types of systems. Additionaly, the lack of rigor in evaluating these systems from a quantitative standpoint, as I argue in the next chapter, may in fact lead to their instability. The second approach to synthesizing sustainability studies - developing systematic empirical frameworks - will be the the main contribution of this dissertation. I demonstrate in Chapter 4 that socioecological frameworks can be adapted to a sociotechnical settings, and that this can lead to better integration of findings about sustainability. 

In final section, I describe research about the governance of systems for peer-production. I argue that this is an important, but underconsidered concept in literature that deals with the governance of contemporary research and development settings.

\subsection{Peer-production}

The decentralized and non-proprietary phenomena of organizing work on the Internet found early success in  Free / Libre Open Source Software projects, such as Linux and Apache (Lahkani, 2001). In these settings people who were paid to contribute often worked alongside volunteers, in an almost non-hierarchical manor. Motivation for participating in these types of production systems include; the freedom to reap rewards of one’s own work, the prospect of future employment, status within a community, reciprocity for help with similar software projects, social justice activism, and simply wanting interesting problems to solve (Scheick, 2007).

Benkler describes the organization of this type of work as a model of “peer-production” :

``a form of open creation and sharing performed by groups on-line that: set and execute goals in a decentralized manner; harness a diverse range of participant motivations, particularly non-monetary motivations; and separate governance and management relations from exclusive forms of property and relational contracts (i.e., projects are governed as open commons or common property regimes and organizational governance utilizes combinations of participatory, meritocratic and charismatic, rather than proprietary or contractual, models)'' (2013).

The new mode of organizing that is enabled can be further refined as commons-based peer production, which is

``...radically decentralized, collaborative, and non-proprietary; based on sharing resources and outputs among widely distributed, loosely connected individuals who cooperate with each other without relying on either market signals or managerial commands.'' (2006, p. 60)

While this early definition characterized participation in peer-production as “loosely connected” Haythornthwaite later offered two models of peer-production on the Internet - Lightweight and Heavyweight: 

\begin{itemize}

\item The lightweight model of peer-production is recognizable in the open-source software projects mentioned above (e.g Apache, Mozilla, or even Linux). In a lightweight model of peer production many loosely connected individuals contribute effort to accomplishing small well-defined tasks. The strength of a lightweight model of peer-production is that participants can self-select tasks, and maximize their own skills in contributing while simultaneously learning new skills in collaborating. 

\item The heavyweight model of peer-production is exemplified by virtual organizations (VOs) made up of strongly connected and highly committed members whose tasks are loosely coordinated, and their contributions are accepted based on quality control mechanisms like peer review (2009). Heavyweight peer production often includes the development of a high-quality dataset, or a piece of software with a very specific use.  

Research on the process of collective action in peer production systems (i.e. Wikipedia, reddit, slashdot, etc.) demonstrate that a mix of both experts and non-experts to coordinate, and cooperatively produce complex works that are (often) of superior quality than their market-based competitors (e.g. Wikipedia vs.Encarta). These systems are often described, in general terms, as having an egalitarian nature of “communal information goods” (Fulk, Flanagin, Kalman,Monge, & Ryan, 1996; as quoted in Shaw and Hill, 2014). Some of the peer-production systems listed above have created democratic organizations that have had transformative and lasting success in industries that are traditionally dominated by private firms (i.e. Apache Server software, Linux and Android operating system, etc.) But, just as often FLOSS projects are the work of a single individual (Ingo, 2006) and by most measures highly uncooperative (Hill, Burger, Jesse, & Bacon, 2008;); Wikipedia is a single success in what were many similar wiki-based encyclopedia projects that failed to attract participants (Kittur and Krut, 2008; Ortega, 2009); and wikis - lauded for being the most democratic of peer production systems tend to resemble oligarchies when studied over time (Hill and Shaw, 2013). Most recently, commons and peer production scholars have started to turn their attention from what has made the Wikipedias of peer production successful towards what has caused similarly structured projects to fail (Shaw et al; 2014). This is a familiar turn, one experienced in early work on CSCW group-ware that first focused on it's successes (Winograd, 1986) and later on their failures(Grudin, 1988). This trope is also recognizable in cognate field's like innovation studies where Michael Porter explained a firm's success through ``competitive advantage '' (1980), while his protége Clayton M. Christensen later explained a firm's failure through a theory of disruption (1997).

All of this is to say that while peer-production is a novel organization form, it does not necessarily lead to successful or even democratic divisions of labor. Peer production satisfies costs and benefits of a networked communication structure just as any other method of production, it just so-happens that it can outperform traditional models of organizing labor given the right context. These contexts remain under-theorized, and whats more the effects of governance on peer-production remain an understudied phenomenon in the current literature of science and technology studies. Where research on open-source software projects suggests that the introduction of peering is reflected in their governance structures (O'mahony and Ferraro, 2008), there have been few similar studies conducted on the impact of peering in research and development settings.

\section{Summary}

In the following Chapters, I will demonstrate how new modes of production (including peer-production) have been governed within the ICOADS community. I will attempt to show how, in combination, changes in provisioning and producing shared resources introduces social dilemmas for a sociotechnical system, like ICOADS, to solve through commons governance. And, I will demonstrate how adapting systematic approaches from the study of commons governance in socioecological systems can help reduce the complexity of understanding these issues as they relate to sociotechnical systems. 

In this chapter I have: 
\begin{itemize}
\item Reviewed the classical economic approach to production and provisioning of goods; including the market and state models. 
\item Described how, initially, Coase and Williamson studied the role of governance in organizing at an institutional level.
\item Discussed the missteps of the ``Tragedy of the Commons'' logic, and how a classification of goods based on two dimensions - rivalry and excludability - helps make sense of the types of externality dilemmas that Hardin had described.
\item I reviewed a number of different findings from Ostrom's work on the successful design and arrangement of institutions for collective action, including a set of rules and design principles that have proved to have high explanatory power for long-enduring socioecological systems. 
\item I then reviewed conceptions of sustainability from the sociecological systems literature, and attempted to show how these are manifested in contemporary science and technology studies. I used this exercise to argue that a systematic approach to studying these phenomena was necessary to build knowledge about which arrangements are successful for sustaining sociotechnical systems. 
\item I introduced the concept of peer-production, and delimited its scope so as to understand that it is one of several models of explanation for the division of labor in contemporary research and development settings. 
\end{itemize}






