\chapter{Chapter 1}

\subsection{The Commons}

There is widespread agreement that scientific knowledge is a public good that can be shared broadly without decreasing its value. As a result, the ability to easily draw upon an open stock of scientific knowledge produces externalities that positively benefit society. 

Like scientific knowledge, the products of scientific research and development (data, software, journal publications, instruments, etc.) create many positive spillover effects for society. But, unlike knowledge, immediate access and broad sharing of these resources affect their value. An open stock of scientific products creates competition for use, social dilemmas around rights of access, and costs related to barriers of entry. The production and provisioning of scientific products are, therefore, typically subject to one of two traditional forms of economic organization:

\begin{enumerate}
\item The market, where intellectual property rights and patents create exclusivity for private ownership, or 
\item The state, which regulate and intervenes in markets by subsiding innovation, and consequently creating open-access resource systems.
\end{enumerate}

In policy analysis of science and technology the two approaches are often referred to as a tension between the commercialization of science on the one hand, and the so called ``norms''of science on the other (Merton, 1963; Eisenberg 1989; Rai 1999; Reichman & Uhlir 2003; Mirowski, 2012; Madison, 2014). 

Both the the market and the state model are costly in a contemporary research and development environment that requires broad collaboration: Markets tend to create wealth by providing limited protection (e.g. patents) to innovators, but these barriers to access often impede the types of transformative breakthroughs that are necessary for combating global scale issues, such as the sharing of data around an Ebola outbreak in Western Africa (Lofgren et al, 2014); Governments tend to create equitable access and opportunity for innovation, but rarely have the capacity to sustain an initial investment longer than a five year grant cycle, leaving innovation unrealized (Mirowksi, 2011). 

Throughout the economic literature of the 20th century, the market and the state appear as two conceptual poles for the production and provisioning of scientific research, but they are not the only options available (Ostrom, 2009). A third approach is the commons - a term informally used to refer to any area of economic activity where there is ``“freedom-to-operate under symmetric constraints, available to an open, or undefined, class of users'' (Benkler, 2014). In practice, the commons resemble institutional arrangements predicated on self-organization and community governance at local levels; and broad cooperation, shared norms, and collective rule-making procedures at larger-scales (Ostrom, 1990). The commons offer an institutional alternative to markets and governments, but are not fully separate from government subsidy, nor without answer to contemporary capital-based marketplace models. That is to say, the commons offer an institutional arrangement which can mix elements of free market ideology alongside hierarchical governance structures. In short, the commons are an institutional hybrid, but not a compromise (Frischmann, 2005). 

The commons have proven to be an institutional arrangement that is especially effective at sustaining socioecological systems that couple people, technologies, and environmental issues of a complex nature (Dietz et al, 2010). Examples of these success can be found at a number of different scales, ranging from effective international treaties negotiated to curb harmful atmospheric emission, such as the Montreal Protocol (Epstein et al., 2014), to the preservation of irrigation systems throughout local municipalities, such as those found in Nepal (Lam, 1998). However, the commons are not a panacea for all of the myriad complex social dilemmas that face the production and provisioning of shared resource systems. Like governments and marketplaces, the commons has its own set of costs and benefits. The work of political economists like Elinor Ostrom shows us that governance models that work in one socioecological setting, may prove ineffective in another (Ostrom, 2010).


\subsection{Science Commons}

Various forms of commons are in use or currently being explored within organizations invested in scientific knowledge production. These include small community resources such as the Weather Research and Forecasting model (WRF) model in meteorology, all the way up to national level funding strategies found in the European Commission's 8th Framework Program for Research and Technological Development. One of the most important examples comes from the National Institutes of Health (NIH) Big Data 2 Knowledge (BD2K) Program which initially described its mission as commons-based:

``The Commons is above all else a conceptual framework for a digital environment to allow efficient storage, manipulation, and sharing of research objects...The Commons belongs to and affects the whole research community... the concept relates to the entire global biomedical research enterprise, the NIH does not own it, so it is not the NIH Commons; similarly it is not just for data and hence is not the Data Commons. Rather it is the concept of sharing digital research objects from any domain, where sharing implies finding, using, reusing and attributing. The Commons could be considered analogous to the Internet – each user has his/her own definition of exactly what the Internet is, but all are able to use it every day for their own purposes. No one seems to own it yet it works because each participant abides by a \textbf{simple set of agreed-upon rules}.'' (Borne, 2014; emphasis mine)

While commons governance \emph{can} be just a simple set of rules they can also be complex, multi-scale and multi-level institutional arrangements. Just as in socio-ecological systems, the sociotechnical systems that enable scientific knowledge production are likely require different kinds of governance systems in different contexts. Sustaining the different sets of resources systems needed to address grand-challenge science issues like climate change or biodiversity loss will require diverse institutions cooperating effectively in different arrangements. In short, the understanding the institutions that effectively enable contemporary science work is a complex and multi-dimensional problem that requires a systematic, and sustained research agenda as much as it does an individual study. 

This research project seeks contributes to evidence-based policy on the sustainability of sociotechnical systems by developing a systematic approach to studying institutional arrangements that enable long-term scientific cooperation. This includes two components: 

1. The use of an emerging Knowledge Commons Framework (KCF) to conduct a case study of the International Comprehensive Ocean and Atmosphere Dataset (ICOADS), a project in marine climatology that has successfully sustained a cooperative model of knowledge production for over thirty years. I compare the results of this work with three previously completed case studies of genomics, astronomy, and biomedicine.

2. By adapting and modifying a protocol from the Social-Ecological Systems Meta-Analysis Database (SESMAD) for systematically coding variables related to different components of the Knowledge Commons Framework. My adaptation of this protocol focuses specifically on governance. I show how standard coding of relevant variables allows for meaningful inter-case comparison with data collected about ICOADS, and has the potential to be used in diverse sociotechnical settings. 

\textbf{Structure of Document}

In the rest of this chapter, I will describe existing and emerging approaches to commons governance, outline the difficulties of governing for sustainability, define a set of key concepts used throughout the project, and finally describe the setting and context for a case study of ICOADS.

In Chapter 2, I review relevant literature and further describe the design choices for this case study. 

Chapter 3 describes data collection methods, and the approach to organizing and analyzing the results of my case study. 

Chapter 4 presents findings from the case study research, which is organized by the Knowledge Commons Framework (Frischmann, Madison, and Strandburg, 2014). Within this framework, I also draw on a systematic approach to coding variables related to governance by adapting portions of the Social-Ecological Systems Meta-Analysis Database (SESMAD) schema (Cox et al, 2014). 

Chapter 5 summarizes the findings of the case study, and provides direct answers to a set of formally stated research questions. In doing so, I  compare results from the ICOADS case study with previous case studies completed using the Knowledge Commons Framework. 

Chapter 6 states the implications of these results for policy, practice, and theory of the commons; and for the governance of sociotechnical systems more generally.

\section*{The Struggle to Govern the Commons}

The commons is a general term that can refer to a resource management approach, as well as the the shared resource system itself (Hess and Ostrom, 2005). This dual-distinction is evident in the etymology of the word ``commons'';  the Latin root being "\emph{communis}, which signifies something held in common by a group, but also a user community bound by responsibilities as well as rights" (Disco and Kranarkis 2013, p. 14). Until very recently the study of how commons function and evolve has focused almost exclusively on natural systems; examples include grazing pastures in Switzerland, aquifers in Mexico, and forests in Western Africa. Previous scholarship in this domain, including the ``tragedy of the commons'' (Hardin, 1968), had created an artificial distinction between private ownership and government regulation. The refutation of this commonly accepted wisdom was pioneered by Elinor Ostrom, who focused on how institutions for collective action develop rules to govern shared resource systems, and what role these institutional arrangements have in creating cooperative, sustainable commons (1990). Ostrom and colleagues used a variety of methodological approaches, including laboratory work, field studies, and surveys, to show that state or market models rarely correspond with the way that successful shared resource systems are governed in the real word. Instead, they often found that users, producers, and provisioners of a natural resource system created self-governing systems which outperformed state and marketplace models. 

A salient example of the unintuitive results of this work comes from the fisheries literature where landings regulations (total amounts of a catch measured in millions of kilograms) are managed under different governance regimes. In the same geographic locations facing the same biophysical conditions, certain fisheries collaps, like the \emph{gadoids} managed under State regulation, while other fishiered thrive, such as \emph{lobsters} managed under a self-organized commons regime.  

\includegraphics[width=4in, height=2.5in]{Fisheries}

\captionbelow{Two landings in millions of kilograms over time. Dotted lines show lobster populations thriving, and Gadoids collapse, both in the same location, but under different governance regimes.}

To put it plainly, governance matters. But, a large body of research on the commons shows us that the effectiveness of rules, sanctions, and policy instruments depends greatly on the context in which they are deployed; gadoids and lobsters may have different reproductive cycles which require different harvesting rules; technologies for trolling gadoids grounds may have advanced more rapidly than the caging techniques used in the lobster industry; or, it may be that a chemical change in the feeding grounds impacted the two species differently. Without a systematic and controlled way to study these different variables, the evaluation of policy effectiveness is thin, and likely incomplete. In developing a comprehensive framework for studying these types of related socioecological systems, commons scholars like Ostrom have been able to generate a deep understanding of which types of social dilemmas require which kinds of governance systems (Acheson, 2012). An empirical, systematic approach greatly increases the chances that a particular policy or a particular design intervention will be effective for sustaining cooperative arrangements.

\subsection{A Systematic Approach Governing Knowledge Commons}

the NIH BD2k may see the commons as a solution to social dilemmas faced by producing and provisioning research infrastructure, it's unclear how effective this model will be, for whom will it be effective, and for how long. What's needed then is a systematic approach to science commons in sociotechnical systems, just as a systematic approach to natural commons has been taken for socioecological systems. Just as what works in one socioecological context may not work in another, a one-to-one mapping will not be perfect between the natural resource systems of Ostrom's study, and the purposefully designed and engineered systems that are the subject of this dissertation. Sociotechnical systems are subject to a variety of different social dilemmas than the natural commons; including the coupling of social and technical phenomena which are not well suited for the types of frameworks that have been developed to study ``biophysical characteristics'' of the natural commons. 

An emerging concept in the commons literature is the notion of ``culturally constructed'' resources managed as ``knowledge commons.''(Madison, Frischmann, and Strandburg, 2010). Where natural commons are the institutions, resource sets, and interactions governed in a socioecological system, the knowledge commons are the cooperative arrangements that share, curate, produce, provision, and sustain informational resources within a sociotechnical system. Governance in the knowledge commons is, similarly, critical to sustainable knowledge production. As Frischmann et al., explain, ``The nested, multi-tiered character of productive and sustainable knowledge and information systems and the diversity of attributes that contribute to successful governance regimes are key to understanding knowledge commons as mechanisms for knowledge production, collection, curation, and distribution in the context of modern information and IP law regimes.'' (2014)

Recognizing that a binary distinction between markets and states is an oversimplification of the organizational models of governance in contemporary science, the commons begin to appear ubiquitous in contemporary research and development settings: knowledge commons are the shared libraries of open-source software that run high-performance computing centers; knowledge commons are the research data archives which enable whole earth climate simulations on a petabyte scale; and, knowledge commons are the instrumentation at the center of astronomical observatories, such as the aptly named ``Very Large Telescope'' in Chile. 

But, although knowledge commons appear to be a pervasive phenomena in science and technology settings there is little empirical understanding of how these types of informal institutional arrangements are successful over time. There is a lack of understanding about which governance arrangements enable sustainable knowledge production in which contexts, And, there is yet to emerge a systematic way of collecting data to compare across these different contexts. This dissertation therefore pursues two broad research questions:

\textbf{RQ 1}. What are the effective institutional arrangements (governance) for sustainable scientific knowledge commons? How do these arrangements differ between domains of knowledge production?

\textbf{RQ 2}. What are the necessary components of an analytical framework that allow for meaningful comparison between and within domains of knowledge production that require shared resource systems? 

These questions are answered by conducting a case study of a knowledge commons in the domain of climate science. By adapting two frameworks used to study socioecological systems to analyze data collected in the case study, I attempt to show the benefit of this approach as applied to the sustainability of knowledge commons in science. I achieve this by comparing these case study results to three previously completed case studies, and synthesizing the findings across all four cases. 

Before introducing the case study subject, the International Comprehensive Ocean and Atmosphere Dataset (ICOADS), I describe a set of basic concepts that will be used throughout this document. 

\subsection{A Controlled Vocabulary}

 This section includes a controlled vocabulary for a set of terms which are important to the overall thesis being pursued. What follows should not be read as strict definitions, but rather a guide for set of concepts which may be used in different ways, by different disciplines. 

\textbf{Commons} 
The commons is a generic term that can refer to a resource management approach,  as well as the shared resource system being managed. In the broadest sense, the commons are marked by ``privileges and immunities for an undefined public, rather than rights and powers for a defined person or persons...The main function of commons is to institutionalize freedom to operate, free of the particular risk that any other can deny us use of that resource set, subject to symmetric known constraints and the risk of congestion applicable to that resource set, under those rules, within the expected population of users.'' (Benkler, 2014).  The two classes of commons that have been described thus far are, on the one hand pastures, forests, and irrigation districts of the natural world, and on the other, high-performance computing, software libraries, and data archives of the digital realm. This dissertation deals exclusively with the latter type of commons. 

\textbf{Sustainability} 
Sustainability can be a relative term. In the context of this project, a sustainable knowledge commons is a sociotechnical system that over the long run, enhances both the quality and the resource base on which science depends, provides for the continued support of resources (such as data, or software, or instrumentation), is economically viable, and enhances the quality of science being conducted. In this sense, a sustainable knowledge commons doesn't just persist over time, but evolves given different external and internal pressures. This idea is a central theme of Chapter 5. 

\textbf{Sociotechnical Systems} 
This project conceives of a sociotechnical system as the mutual constitution of people and technologies in social, political and economic settings that require collective action in order to effectively function over time (Sawyer and Jahairi, 2013). The contextual and embedded nature of sociotechnical systems makes governance, institutional arrangements, and symmetric information exchange paramount to their success. 

\textbf{Infrastructure} 
Infrastructure, like sustainability, is an inherently relational concept. In a science and technology setting, infrastructures can be defined in relation to organized practices of communities, disciplines, or fields of study (Star and Ruhleder, 199). The term of art for scientific infrastructures has recently become cyberinfrastructure (Atkins, 2000), which I take to mean ``...the set of organizational practices, technical infrastructure and social norms that collectively provide for the smooth operation of scientific work at a distnace. All three sets are objects of design and engineering; a cyberinfrastructure will fail if any one is ignored''(Edwards et al., 2007, p. 6)

Given these explanations, a natural question is what are the blurred lines between infrastructures, cyberinfratructures, sociotechnical systems, and commons? 

Commons focus on the institutional arrangements, with an emphasis on the rules and governance of people, resources, and the bundle of property rights that are negotiated for their long-term sustainability. Infrastructure, and cyberinfrastructures, are a congruent, and complimentary view of these interconnected elements. Infrastructures can (and often are) managed as a common property (Frischmann, 2005), for which collective action is required to keep the ``smooth operation of scientific work at a distance'' occurring. Although two literatures - commons and infrastructure studies -  make reference to one another, analysis of their relationships are rarely combined (a notable exception, Frischmann, 2005). For instance, commons scholars increasingly acknowledge the importance of developing an account of the infrastructural resources that shape, and are shaped by long-term interactions within shared resource systems (Dietz et al, 2010). Part of the goal of this research project is to better align the findings from a long line of cyberinfrastructure studies, and the emerging knowledge commons frameworks described in Chapter 2. 

\textbf{Governance} 
A helpful definition of governance is that it includes,  ``a complex of public and/or private coordinating, steering and regulatory processes established and conducted for social (or collective) purposes where powers are distributed among multiple agents, according to formal and informal rules'' (Burns and Stöhr 2011: 234). As stated above, a sociotechnical perspective  recognizes that collective action is needed to sustain the social relations, orderings, and enforcement of cultural norms, as well as the technical components that allow a commons to effectively function over time. A governance model, a term also used throughout the dissertation, implies the sets of``institutional arrangements (such as rules, policies, and governance activities) that are used by one or more actor groups to interact with and govern'' shared resources (Cox et al, 2015). 

Governance models typically differ in their centralization (or decentralization) of decision making power - such as self-governing or monocentric governance. This dissertation explores polycentric governance models that nest authority at multiple levels, types, sectors, or jurisdictions. The overlap of these different levels create collective action dilemmas, which require multiple rule types to function efficiently. Polycentric models, as I discuss in the following chapter, are increasingly effective for helping sociotechnical systems cope with social dilemmas related to sustainability.

\textbf{Resilience}\\ 
The notion of resilience in socioecological systems is partially evolutionary, and partially ecological. Holling formulated this idea in the early 1970s, defining resilience as a ``measure of the persistence of systems and of their ability to absorb change and disturbance and still maintain the same relationships between populations or state variables.'' (Holling, 1974) In Chapter 5 I focus on transitions of ICOADS between different governance regimes; the ability of a commons to routinely make these governance transitions is described through \emph{resiliency processes}. The NSF's Critical Resilient Interdependent Infrastructure Systems and Processes (CRISP) program offers a helpful definition of a resilient process for infrastructures; these are ``the features of an infrastructure’s inherent capacity to resist disturbances, initial loss of service quality, and trajectory of service restoration. Conceived as a process, infrastructure resiliency can be achieved by a myriad strategies in addition to simple repair and replacement.'' (2013). The ``myriad strategies'' of ICOADS resiliency are a subject explored throughout this project. 

\subsection{Research Context and Rationale}

I turn next to an introduction of the subject of my case study, and the major subject of data collected for this dissertation. A thorough discussion of the background environment for ICOADS appears in Chapter 4. 

\subsubsection{Context}

The International Comprehensive Ocean and Atmosphere Dataset (ICOADS) is
a cooperative project that curates, develops and distributes quality controlled
data, metadata, historical documentation, and software to the
climate science community. The project, originally named COADS, was
initiated in 1981 by researchers at the Earth System Research Laboratory
(ESRL), the National Climatic Data Center (NCDC), and the National
Center for Atmospheric Research (NCAR). Over time the project grew to
include international collaborators and the name was changed to the
``International COADS'' in order to reflect the contributions of organizations like
the World Meteorological Organisation (WMO), the Intergovernmental
Oceanographic Commission (IOC), and the Technical Commission for
Oceanography and Marine Meteorology (JCOMM).\\

Contemporary data curated by ICOADS come from a variety of sources,
including the Global Telecommunications System (GTS), in-situ
measurements taken by sea-faring vessels, earth observing satellites,
and both drifting and moored buoys. Historical data come from an effort in the early 1980s to aggregate existing marine data records from maritime archives around the world, as well as a continual stream of
historical records that have been rediscovered and newly digitized; this now includes the use of crowd-sourcing efforts to transcribe weather recordings taken by military, shipping, and whaling voyages from the 17th, 18th, and 19th century. These early records are significant culturally and historically, as ``Sailors were among the first to systematically record the weather because the states of ocean and atmosphere controlled their progress and survival (Woodruff et al, 1986 citing Quayle, 1977)''.\\

The labor-intensive process of taking heterogeneous records from different observing platforms, and uniformly processing and integrating the data into a larger set of historical observations is the major value of ICOADS ongoing work. This includes the preservation of provenance metadata that is recorded for each individual record, allowing for researchers to trace backwards in time to verify whether the source of a climate anomaly is genuine or the result of a data processing
error.

Free and open access to ICOADS has helped it become recognized by the
climate community as the ``most complete and heterogeneous collection of
surface marine data in existence'' (Woodruff et al., 2011). ICOADS data have
been used extensively in international climate assessments such as the IPCC AR4 and AR5 reports, as well as reanalysis
projects that combine historical data with contemporary weather observations to create authoritative datasets for the climate modeling community (Kalnay et al., 1996). 

The success and widespread use of ICOADS has not resulted in greater stability for the funding of the project. This is due in part
to the difficulty in calculating the research impact of many different ICAODS products (Weber et al., 2014), as well as an overall decrease in federal funding for the maintenance of research infrastructures (Berman & Cerf, 2013). The politicization of climate related research has also impacted ICOADS funding in recent years as congressional pressure to defund ``climate research'' continues to mount. In the winter of 2012, this became a major issue for the sustainability of the project, as NOAA announced that:

``For budgetary reasons, stemming from pending large cuts at the NOAA Climate Program Office (CPO), ESRL Directors have determined that it is no longer feasible for its Physical Science Division (PSD) to continue supporting any further ICOADS work effective immediately'' (Lawrimore, 2012)

In response, project partners at the National Center for Atmospheric Research (NCAR), the UK Meteorological Office, and the Deutscher Wetterdienst (German Meteorological Service) signed a memorandum of understanding to continue contributing to the maintenance of the project, and its various resources. The research of this dissertation was conducted beginning at the point that signatures for a memorandum of understanding were signed. 

\subsubsection{Rationale}

For many of the reasons described above, ICOADS offers a unique case study of sustainability in a knowledge commons. These include, but are not limited to:

\begin{itemize}
\item 
Project partners are diverse in terms of their organizational affiliations, and expertise. As such, ICOADS itslef is nested within a number of overlapping governance structures.
\item
Climate science is a unique domain of knowledge production in that it requires a broad scheme of cooperation in order to generate verifiable research results. And further, these results are some of the most scrutinized, and politically charged forms of knowledge production. 
\item
The resource sets provisioned by ICOADS have persisted over a thirty year period that has seen a number of fluctuations in funding, the politicization of the subject matter, and rapid technological change. ICOADS therefore presents an opportunity to better understand how diverse resource sets, consisting of software, data, human expertise, and computational infrastructures are sustained over time. 
\end{itemize}

The specific research questions that this case study will answer are related to the ways in which, over time, ICOADS project governance has evolved in order to sustain a unique set of shared resources. Stated formally, the research questions answered by this case study will be:
\begin{itemize}
\item \textbf{RQ 1} : What are the different governance models that ICOADS has effectively used to manage shared resources over it's thirty year existence (1983-present)?\\

\item \textbf{RQ 2} What causes a governance system to shift from one regime to another?\\
\end{itemize}

A a regime shift is, following Smith, Stirling, and Berkhout (2005) understood to be a function of two processes:
\begin{enumerate}
\item Shifting selection pressures (either external or internal pressure to change) bearing on the regime.
\item The coordination of resources available inside and outside the regime to adapt to these pressures.
\end{enumerate}

\item \textbf{RQ 3} What types of disturbances are ICOADS resilient or vulnerable to?\\

In order to identify selection pressures, governance shifts, and disturbance types, I use a protocol adapted from socioecological commons to code variables related to ICOADS different governance models. This process is described in detail in Chapter 2.  
\end{enumerate}

\section{Summary}

In this chapter I have defined a set of research problems that have been under-examined in the current literature, namely how knowledge commons in scientific research and development settings are sustained over time. I have identified a number of systematic approaches used to study socioecological systems, and suggested ways in which these can be a usefully adapted to the study of sociotechnical systems sustainability. I offered a controlled vocabulary for concepts that will be important to the entirety of this document, including sustainability, commons, governance, sociotechnical systems, infrastructure, and resilience. Finally, I briefly described the subject of my case study, ICOADS, and justified the choice of this case.

I also offered two sets of research questions; the first set of questions are aimed at understanding sustainability in knowledge commons more generally. This set of questions will be answered by comparing the results of a case study of ICOADS with previously completed case studies in Biology, Astronomy, and Biomedical research networks. The second set of questions - which are specific to the ICOADS case study - are answered through the use of a range of data collected and analyzed through a framework that is described in detail in Chapter 3. In the next chapter, I review literature relevant to this study, and further contextualize this research project. 



