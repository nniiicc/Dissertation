\section{Ch 1 Summary}

In this chapter I have defined a set of research problems that have been under-examined in the current literature, namely how knowledge commons in scientific research and development settings are sustained over time. I have identified a number of systematic approaches used to study socioecological systems, and suggested ways in which these can be a usefully adapted to the study of sociotechnical systems sustainability. I offered a controlled vocabulary for concepts that will be important to the entirety of this document, including sustainability, commons, governance, sociotechnical systems, infrastructure, and resilience. Finally, I briefly described the subject of my case study, ICOADS, and justified the choice of this case.

I also offered two sets of research questions; the first set of questions are aimed at understanding sustainability in knowledge commons more generally. This set of questions will be answered by comparing the results of a case study of ICOADS with previously completed case studies in Biology, Astronomy, and Biomedical research networks. The second set of questions - which are specific to the ICOADS case study - are answered through the use of a range of data collected and analyzed through a framework that is described in detail in Chapter 3. In the next chapter, I review literature relevant to this study, and further contextualize this research project. 

\section{Ch 2 Introduction}

\emph{This chapter reviews relevant literature on governance and sustainability of shared resource systems, including the various conceptual models which inform commons theory. I focus in particular on aligning the rules-based approach of socio-ecological systems with those of sociotechnical systems.}\\

\emph{I begin with an overview of economic theories of organizing around shared resource systems. I then describe the emergence of commons theory, and its relationship to sustaining socio-ecological systems. Finally, I relate previous research on the natural commons to emerging notions of the knowledge commons, which brings with it new models of peer production, and use of networked information communication technologies.}\\

\section{Ch 2 Summary}

In the following Chapters, I will demonstrate how new modes of production (including peer-production) have been governed within the ICOADS community. I will attempt to show how, in combination, changes in provisioning and producing shared resources introduces social dilemmas for a sociotechnical system, like ICOADS, to solve through commons governance. And, I will demonstrate how adapting systematic approaches from the study of commons governance in socioecological systems can help reduce the complexity of understanding these issues as they relate to sociotechnical systems. 

In this chapter I have: 
\begin{itemize}
\item Reviewed the classical economic approach to production and provisioning of goods; including the market and state models. 
\item Described how, initially, Coase and Williamson studied the role of governance in organizing at an institutional level.
\item Discussed the missteps of the ``Tragedy of the Commons'' logic, and how a classification of goods based on two dimensions - rivalry and excludability - helps make sense of the types of externality dilemmas that Hardin had described.
\item I reviewed a number of different findings from Ostrom's work on the successful design and arrangement of institutions for collective action, including a set of rules and design principles that have proved to have high explanatory power for long-enduring socioecological systems. 
\item I then reviewed conceptions of sustainability from the sociecological systems literature, and attempted to show how these are manifested in contemporary science and technology studies. I used this exercise to argue that a systematic approach to studying these phenomena was necessary to build knowledge about which arrangements are successful for sustaining sociotechnical systems. 
\item I introduced the concept of peer-production, and delimited its scope so as to understand that it is one of several models of explanation for the division of labor in contemporary research and development settings. 
\end{itemize}

\section{Ch 3 Introduction} 

\emph{This chapter describes in detail the range of empirical research methods used to carry out this project. This includes the design of a case study, as well as the various methods used for data collection and analysis}

\emph{I begin by describing the case study design, including the overall structure of the case and its background. I then describe the methods of data collection used in this work, including a set of ethnographic studies carried out over a three year period, three informetric studies that provide a quantitative account of ICOADS use and acknowledgment, and a set of semi-structured interviews conducted in year three of the study. I conclude by describing the use of two existing frameworks to organize and analyze these different data sources, and the validity constructs that have guided this analysis.} 

\section{Ch 3 Summary}

In this Chapter, I have described the range of empirical research methods used to collect and analyze data; including the case study design, process of data collection, as well as the analysis and organization of the results into a modified knowledge commons framework, and the coding of variables related to evolution of ICOADS governance which draws upon the SESMAD framework. Results from this analysis will be presented in Chapter 4 (following). In Chapter 5, I use these findings to answer the stated research questions of the case study. I then compare these results to previously completed case studies, and answer the stated research questions of the dissertation overall. 

\section{Ch 4 Introduction}

\emph{This chapter presents an analysis of ICOADS governance using the knowledge commons framework. I begin with an overview of concepts from previous chapters. I then use data collected from the case study of ICOADS to fill in the major framework categories - Background Environment, Attributes, Governance \& Rules-in-Use, and Outcomes. Further synthesis of these findings appear in Chapter Five.}

\section{Ch 4 Summary}

In this chapter I have summarized the differences between the IAD and KFC, and justified my innovation with both of these frameworks to present an analysis of the ICOADS case study data. I then provided background information important to understanding the context in which ICOADS was developed, including a number of formal policies at the international level, and failed legislation at the national level. I then offered an overview of ICOADS various resource sets, and community member roles. Finally, I justified the division of the ICOADS case study into three separate governance regimes, offered an analysis of how rules, and governance variables were effected by the ``action arenas'' of each regime - including my participants (ICOADS Steering Committee members) own perspectives about these events. In the next Chapter I summarize the major findings of this analysis and answer the research questions formally stated in Chapter 2.  

\section{Ch 5 Introduction}
\emph{In this chapter I summarize the findings of the ICOADS case study, and answer three case study specific research questions. I then review three previously completed case studies using the Knowledge Commons Framework, and compare and contrast findings from all four cases. I use this cross-case comparison to answer the overall research questions of this dissertation.}

\section{Ch 5 Summary}
In this chapter I have summarized the research findings from a case study of ICOADS governance, and answered three research questions related to the effectiveness of ICOADS' governance. I then compared these findings with three previously completed case studies, and answered the overall research questions of this dissertation. In doing so, I demonstrated how a systematic approach to coding state variables could be adapted from socioecological to sociotechnical settings. In the final chapter, I describe limitations of this work, and the implications of this research for policy, practice, and theory of the commons. I conclude with future directions for this work. 

\section{Ch 6 Introduction}

\emph{In this chapter I restate the limitations of this work, and describe the implications of this dissertation's findings for policy, practice, and theory of the commons. I conclude with future directions for studies of sociotechnical systems sustainability.}\\