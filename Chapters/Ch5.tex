\chapter{5}

\emph{In this chapter I answer the dissertation's formally stated research questions. I begin first by summarizing the findings of the ICOADS case study, and directly answering three related research questions stated in the case study design found in Chapter 3. Next, I review findings from three previously completed case studies using the Knowledge Commons Framework, and then compare and contrast findings from all four cases. I use this cross-case comparison to answer the overall research questions of this dissertation.}

\section{Case Study: ICOADS}

The case study data organized by the Knowledge Commons Framework in Chapter 4 included the following categories Background Environment, Community Attributes, Governance and Rules-in-Use. The Governance sections were divided into three ``action arenas'', each consisting of a set of governance outcomes in the form of coded state variables, which help characterize a system at a particular period of time (Walker et al, 2005), as well as a categorization of the types of rule-in-use during that period of time. 

The process of developing a coding scheme for these variables took the following steps:

\begin{enumerate}
\item Identifying a list of relevant variables from the Socioecological Systems Meta-analysis Database (SESMAD) project
\item Identifying relevant literature to generate new, or support existing variable definitions
\item Coding each regime based on the relevant ethnographic and interpretive interventionist (interview) data. 
\item Reviewing the codings with key informant for the sake of external validity
\item Iteratively editing definitions for consistency 
\end{enumerate}

The outcome of this work is summarized in the table below: 

% Please add the following required packages to your document preamble:
% \usepackage{graphicx}
\begin{table}[h]
\resizebox{\textwidth}{!}{%
\begin{tabular}{|l|c|c|c|}
\hline
\textbf{Variable Name}             & \textbf{Regime 1}                  & \textbf{Regime 2}                     & \textbf{Regime 3}                   \\ \hline
Begin date                         & 1981                               & 2001                                  & 2011                                \\ \hline
End Date                           & 2000                               & 2010                                  & Present                             \\ \hline
Description                        & In Text                            & In Text                               & In Text                             \\ \hline
Multiple levels                    & Single-Level Governance            & International Governance              & International Governance            \\ \hline
Institutional diversity            & Low                                & Medium                                & Medium                              \\ \hline
Centralization                     & 4 (Highly Centralized)             & 3 (Somewhat Centralized)              & 2 (Somewhat Decentralized)          \\ \hline
Governance scale                   & State-based                        & International                         & International                       \\ \hline
Scale match                        & No-Match                           & Match                                 & Match                               \\ \hline
Selective Pressures                & Competition, Resource Availability & Competition, Technological Innovation & Obselescence, Resource Availability \\ \hline
\multicolumn{1}{|r|}{Articulation} & Medium                             & Low                                   & Low                                 \\ \hline
\multicolumn{1}{|r|}{Capital}      & Low                                & High                                  & Medium                              \\ \hline
\multicolumn{1}{|r|}{Coordination} & Low                                & High                                  & High                                \\ \hline
Governance trigger                 & Slow  Continuous Change            & Slow  Continuous Change               & Sudden Disturbance                  \\ \hline
Diagnostic Capacity                & Neither                            & Treating proximate causes (symptoms)  & Treating underlying drivers         \\ \hline
Autonomy                           & High- complete autonomy            & Moderate- some autonomy               & Moderate - some autonomy            \\ \hline
Metric diversity                   & Low- no metrics for success        & Medium- some metrics for success      & Medium- some metrics for success    \\ \hline
Rules-In-Use                       & Position, Authority                & Information, Payoff                   & Boundary, Aggregation, Scope        \\ \hline
Policy instrument                  & Incentive Based Insturments        & Incentive Based Instruments           & Input and Output Based Standards    \\ \hline
\end{tabular}
}
\caption{Coded State Variables from the each regime of the ICAODS case study.}
\label{my-label}
\end{table}

Comparing single variables across the regime, we can observe the following trends: 

\textbf{Institutional diversity} ICOADS moved from having low institutional diversity, meaning that a homogenous set of institutional arrangements had been applied to diverse set of resources, to a medium level of institutional diversity, where the institutional arrangements - including JCoMM, the ICOADS Steering Committee, and the working groups of ICOADS that form for short amounts of time during CLIMAR and MARCDAT meetings - can all be seen as evidence of the project diversifying its governing structure.

\textbf{Centralization} Similarly, governance became less centralized over time - moving from the ordinal variable code ``4: A Highly Centralized'' governance model, to a ``2: Somewhat Decentralized'' mode' of governance. This is in line with institutional arrangements becoming more diverse, as described above, and the scale of the governance model increasing, as discussed next.

\textbf{Governance Scale} The governance scale variable is meant to code what jurisdiction the governance system has authority over. For instance, can the governance model apply rules, sanctions, bans, or assign tasks at state, national, or international levels? ICOADS moved from a national scale - where partners at NOAA, CIRES, and NCAR were subject to its governance- to an international scale; first with granting partners voting rights in the renaming of the project (2001), and then further formalizing the internationalization of the project with the signing of a letter of intent to cooperate (2013) by seven different institutions. 

\textbf{Scale Match}
The scale of a governance can then be coded, within different states of a system, as having a binary ``match'' or ``no-match'' In ICOADS first regime, a set of internationally archived records are used to produce resources that are provisioned with a state-based governance model. This sets up a tension where the broader ICOADS community felt there was not a match between the source of the inputs (international weather records) and the governance of the outputs (a state-based model). Regime one therefore does not provide a match between stakeholders and governance. In Regime 2 and 3, governance expands through the formation of JCoMM and this results in an international governance structure that now matches the scale of the resources and their users, producers, and provisioners. 

\textbf{Selective Pressures}
The types of pressures exerted on ICOADS governance relate to its shifting between regimes. These variables are meant to define and provide context for those pressures, but these categorizations should not be read as directly causing a regime shift. They are one of many stresses, disturbances, and motivations for a regime shit. 

The first shift between regime 1 and 2 is a result of ``competition'' and ``resource availability'' pressures. Competition for users, and for satisfying a need for a comprehensive marine dataset influenced ICOADS leadership to formally recognize the contribution of its international partners, as well as merge with its chief competitor, the Marine Data Bank. By expanding the scope of legitimate stakeholders for producing and provisioning the commons, ICOADS governance also necessarily had to become less ``Centralized'', increase its ``Institutional Diversity'', and adjust its ``Governance Scale''. These three variables (Centralization, Institutional Diversity, and Governance Scale) correlate strongly with ``Selective Pressures'' when a regime shift occurs.  

\textbf{Sub-variables of Selective Pressures}
\emph{Articulation, Capital, and Coordination}

Selective pressures have three sub-variables which are meant to further categorize responses from the commons governance. 

\begin{itemize}
\item Articulation explains the extent to which a governance model is capable of recognizing, describing, or defining selective pressures. 
\item Capital, both social and technical, explains the extent to which a governance model can access and extract resources from a network of stakeholders.
\item Coordination explains the extent to which a governance model is able to organize a response to selective pressures
\end{itemize}

In Regime 1, ICOADS stakeholders had a partial ability to articulate problems that were introduced by competition and resource availability, but had low social and technical capital for drawing on resource from the ICOADS user-base to respond to these pressures. The coordination of ICOADS partners was, initially low, but improved when JCoMM was introduced as a parent organization. 

In Regime 2, selective pressures include technological change, seen through innovations in the assembly of reanalysis data products, and competition for resources, due to proliferation of high quality subsets of ICOADS. Articulation of these problems remained low; the number of sub-sets were so numerous and so diffuse that they could no longer be included in ICOADS documentation. Both social and technical capital, as well as coordination within the second regime were high as ICOADS was able to coordinate development for integration with reanalysis projects, and position the resources to be used in major international climate assessment reports.

In the third, and most current regime, ICOADS suffered initially from resource availability but seems to have solved some of these problems by reaching out to international partners for support. However, ICOADS continues to face pressure to innovate and risks being branded obsolete, in relation to already existing high quality subsets and reanalysis products. Articulation within the third regime was low, as the identification and recognition that de-funding seemed to take even most senior NOAA project partners by surprise. In part, this demonstrated that levels of social capital were high in the third regime, as NOAA consequently acknowledged that restoration of partial funding was due to the outpouring of support from the international community. However, this revealed that technical capital remained low because the interruption of work on IMMA software greatly delayed progress on an anticipated and improved ICOADS third release\footnote{This is an admitted flaw to the way that the ``Capital'' variable is currently coded: When social and technical capital do not align the results are a compromised ``medium'' coding, when in fact it should be ``high'' and ``low''}. Coordination was high in Regime 3, as demonstrated to the ability to respond with letter of intent to cooperate, and the formal assignment of tasks to project partners that agreed to work towards a third release. 

Selective pressures, along with governance triggers and diagnostics are all highly related variables and should have a consistent relationship, which I describe in the next two sections.

\textbf{Governance Triggers}

``Governance triggers'' is a variable meant to provide information about the accumulation of pressures that leads to a change in governance, such as a policy instrument being adopted or a regime shift. Governance triggers are characterized by the notion of disturbance from socioecological systems management. A disturbance is some event, or set of selective pressures which cause a change in state variables. Governance triggers can be either ``sudden disturbances'', such as a defunding event from NOAA that shifts state variables; or, a governance trigger can consist of ``slow continuous change'', such as the obsolescence of a technology over time as ICOADS is currently experiencing in its third regime. 

Regime one and two of ICOADS can therefore be characterized as having slow continuous changes. Disturbances during these periods are related to organizing, creating rules, and fostering cooperation amongst diverse partners. Even during periods of scientific controversy, such as the discovery of anomalies in sea-surface temperature measurements in a historical period of the 1940-1960s (Rayner et al., 2005), there was a relatively stable ICOADS response and a cooperative effort to improve the errors (Woodruff et al., 2011). 

The third regime, which is marked by a de-funding event is the first time that sudden disturbance causes a shift in state variables, such as the governance formalizing by signing a letter of intent to cooperate, and the formation of an international steering committee to formalize governance processes. 

\textbf{Diagnostics}
The diagnostic variable attempts to categorize the response of a regime to selective pressures. This variable is related to governance scale and scale match, as well as the selective pressures sub-variables; 

\begin{itemize}
\item A governance with a strong scale match and ability and ability to articulate, draw on capital, and coordiante responses will be able to successfully transition between regimes. 

\item Without a scale match, or articulation of selective pressure, the governance regime will lack the ability to make a diagnosis of why a selective pressure is occurring and subsequently respond to either the symptioms or the underlying causes. 
\end{itemize}

Regime 1's governance had no scale match, a medium level of articulation, but low capital and low coordination to respond to selective pressures. The governance model failed to respond to issues of competition, and resource availability and neither treated underlying causes of the selective pressures, nor its symptoms until a regime shift was required to do so.

Regime 2 faced selective pressures from Regime 1, competition and resource availability, by treating the symptoms of the underlying controversy - which was to recognize formally the international partners that had contributed to the project. Regime 2 has a governance scale match, but low ability to articulate the problem, and as a result of it's continued scientific successes, the ability to draw on social and technical capital. Regime 2's governance model could also coordinate responses to selective pressures such as competition and technological innovation. However, the failure to articulate why selective pressures were manifested during this period also leads to a failure to treat of underlying causes. Treating the symptoms related to selective pressures like technological change and competition for resources, Regime 2 was able to coordinate a new method of scheduling releases, a new method of coordinating work (CLIMAR and MARCDAT), and begin development of a new database model through IVAD, but in failing to recognize underlying problems related to measurement and evaluation of the importance of the project to one of its main funders (NOAA), resulted in a sudden disturbance event that nearly ended the project.  

\textbf{Metric Diversity}
All three regimes suffer from evaluation problems that stem, at least in part, from a lack of metrics that can clearly communicate ICOADS' impact to organizations that bear the costs of provisioning a shared set of resources\footnote{This is a larger problem with data centers and infrastructures more generally (see Weber et al., 2012 for full discussion)}. In discussing the first and second regime, metrics were rarely mentioned by either my participants or in the literature of ICOADS. More broadly, there was no recognition throughout regimes one and two that an underlying problem in sustaining ICOADS was a lack of formative evaluation. Metric diversity, in this case, is strongly tied to diagnostics; an indicator of healthy or robust knowledge commons may indeed be a diversity of metrics for evaluating success, attainment of goals, or more broadly conveying the importance of a resource set.

\textbf{Policy Instruments}

Policy instruments can be categorized as either regulatory or incentive based. Regulatory policy instruments are aimed at affecting behavior of commons stakeholders through coercion or punishment, for instance by enforcing penalties or sanctions stated by ``rules-in-form''. Policy instruments can also be incentive-based, which is meant to elicit desired behavior through reward. In ICOADS, incentive-based instruments were used throughout all three regimes. Regime 1 and 2 relied exclusively on incentive-based policy instruments, as these regimes had little authority in the way of sanctions or enforcement of rules to establish any kind of regulatory policy about what kind of actions were, or were not accepted. Regime 3, through formalizing responsibilities in producing Release 3.0 would adopt some minor, input and output standards about which institutions were to contribute what types of resources, and what entitlements came with these responsibilities. 

\textbf{Rules-in-Use}
In Chapter 4, I also used a set of categorized rules to describe what types of regulatory and incentive based policy instruments were being used by ICOADS governance. It is important to note that rules have varying degrees of formality - some can be neatly fit into a category and named (such as boundary rules), others are more broad and are best described a range of activities (such as scope). Where this is the case, I provide examples of the types of rules that are in use, and their consequences for governing. 

\begin{table}[h]
\resizebox{\textwidth}{!}{%
\begin{tabular}{|l|l|l|l|l|l|l|l|}
\hline
                  & \textbf{Position}        & \textbf{Boundary}        & \textbf{Authority}       & \textbf{Scope}           & \textbf{Aggregation}     & \textbf{Information}     & \textbf{Payoffs}          \\ \hline
\textbf{Regime 1} & \cellcolor[HTML]{C0C0C0} &                          & \cellcolor[HTML]{C0C0C0} &                          &                          &                          &                           \\ \hline
\textbf{Regime 2} & \cellcolor[HTML]{C0C0C0} &                          & \cellcolor[HTML]{C0C0C0} &                          &                          & \cellcolor[HTML]{C0C0C0} & \cellcolor[HTML]{C0C0C0} \\ \hline
\textbf{Regime 3} & \cellcolor[HTML]{C0C0C0} & \cellcolor[HTML]{C0C0C0} & \cellcolor[HTML]{C0C0C0} & \cellcolor[HTML]{C0C0C0} & \cellcolor[HTML]{C0C0C0} & \cellcolor[HTML]{C0C0C0} & \cellcolor[HTML]{C0C0C0}  \\ \hline
\end{tabular}
}
\caption{Accumulation of rules across three regimes of ICOADS (1983-2015)}
\label{my-label}
\end{table}
\\
\textbf{Regime 1}\\

Throughout the early development and refinement of COADS, there are rules and norms established through historical events, such as the HSST project which demonstrated the feasibility of assembling a comprehensive dataset for marine surface records. However, it should be noted that the HSST was possible only by dividing up and assigning responsibility for different parts of the data curation. Rules, even informally introduced and enacted, are numerous in Regime 1, but can generally be classified as fitting two types:\\

\begin{itemize}
\item \textbf{Position Rules} specify a set of positions that actors can play. Each position has a unique combination of resources, opportunities, preferences, and responsibilities.

\item \textbf{Authority Rules} specify which set of actions is assigned to which position. (McGinnis, 2011)  
\end{itemize}

In COADS initial governance - which was ad-hoc, and loosely coordinated, position rules are established by data provisioners whom assigned responsibilities to one another, and to end-users. Users were free to obtain data, and encouraged to report errors and participate in improving the archive (Woodruff et al., 1987). Data producers who contributed statistical techniques, algorithms, software, or data processing were assigned rights and responsibilities to stewardship of these resources, as well as any version of the archive that might obtain as part of this work. These types of rules are found throughout the formally published literature, as well as document specifications which state who will be responsible for what duties in creating COADS release 1. In an interview about this early process, one participant characterized the early coordination in the following way: ``...the problem is that, ICOADS has a single point of failure. I mean (PI) was ICOADS - he was the identity. And while we didn't have a communication problem through language or responsiveness, we bumped up against competing interests - it was a much bigger project and scope than even (PI) realized when we first started putting data out...before the CLIMAR meetings started.'' [ISC- 06]

Next, I turn to the second regime, which as this quote indicates is when formal meetings begin to demonstrate the value of COADS to an international community of users. \\

\textbf{Regime 2}\\

The second regime introduces three new rule types to ICOADS governance.  

\begin{itemize}
\item \textbf{Information rules} specify channels of communication among actors and what information should, must, or must not be shared. 

\item \textbf{Payoff Rules} specify how benefits and costs are required, permitted, or forbidden to players.

\item \textbf{Scope Rules } specify a set of outcomes, such as what should be produced, what levels of maintenance a resource set requires. Scope rules are closely related to payoff and information rules. (McGinnis, 2011)  
\end{itemize}

Information rules developed during this regime are seen in the specifications for formatting data for exchange, and recommendations from the second CLIMAR meeting in 2004 which read ``There were seen to be shortcomings in the access to ICOADS data. There are many, overlapping sources of  data and products, and the problem of optimising data provision is complex. Many users are working with outdated versions of COADS. Often data are available, but it is difficult for the uninitiated to discover what is there. There should be a Web-based route map to the best available data which should be widely advertised to all the various user communities'' (Parker et al., 2004). The ``route map'' took the form of website developed to host specification documents, and link out to existing WMO standards (Woodruff et al., 2006). ICOADS ``route map'' is established by 


 development of a rule type doesn't imply that a social dilemma has been solved. In this case, blending of different data products and an overall growth of the archive leads to complexity for end users. Introducing informational rules is a way to treat symptoms of a larger problem in managing a sustainable growth of the project. 

Payoff rules are established in granting international partners formal recognition through a name change. Recall that when asked about the process, one of my participants replied ``...that debate really focused attention on the fact that COADS should be looked at as an International dataset, because you know a lot of other countries were putting money and data into it.'' The settlement of that debate brought with it a larger set of responsibilities for international project partners, and their engagement in provisioning and producing new interim products throughout the regime demonstrates how and when benefits and costs are to be distributed amongst these different groups. 

Scope rules are another artifact of the CLIMAR and MARDAT meetings, where strategic directions are set, and future goals are established. Establishing scope rules also represents a significant shift in the governance model, as it becomes increasingly less centralized.\\

\textbf{Regime 3}\\

\textbf{Types of Rules}

Two new types of rules are added, and one rule type is further developed in this regime:

\begin{itemize}
\item \textbf{Aggregation rules} (such as majority or unanimity rules) specify how the decisions of actors at a node are to be mapped to intermediate or final outcomes.  

\item \textbf{Boundary rules} specify how participants leave or enter positions of authority. (McGinnis, 2011)  
\end{itemize}

Aggregation and boundary rules are established by mechanisms introduced by the letter of intent to cooperate. When asked about how these rules are put into practice, a participant described this situation as follows 

``The governance is still in flux- the committee are new, the terms of references are still new, and we still have to formalize a bit more how members serve, how long do they come and go, etc.'' [ISC 04]

In March of 2015 when a discussion about the appoint a permanent chair for the steering committee was raised, the interim chair reminded the steering committee that he was retiring. and expressed no desire to remain in the role during retirement. Although this was public, no one understood nor knew the procedure to make a nomination for a new chair, or whether on-going questions about future support of ICOADS at NOAA should impact this choice. As it stands, no decision has been made, and existing by-laws offer little guidance on how to proceed. 

Additionally, the \emph{Authority rules} of previous regimes are both formalized and enacted in new ways in the third regime. This includes formally dividing responsibility for the future provisioning of ICOADS resources, and the creation of a set of core institutions - beyond the original USA partners of NOAA, NCAR, and CIRES - that formally acknowledge their intention to sustain ICOADS over a five year period (beginning, December 2013). 

\textbf{Summary of Rules and State Variables}

The systematic approach to characterizing state variables shows a number of important relationships between these variables and their impact on governance. 

\begin{itemize}
\item Governance scale, scale match, and the sub-variable articulation all correlated strongly with governance triggers. These are not causes for, but contexts of state changes such as a regime shift. 

\item Selective pressures, and the sub-variables articulation, capital, and coordination are illustrative of both the ways in which governance states experience, and react to disturbances. These variables are tied strongly to the rules-in-use, and policy instruments. 

\item Diagnostic capabilities help explain the effectiveness of a governance model at understanding the underlying causes of governance triggers, and selective pressures.  

\item The analysis of rules-in-use shows how rule types accumulate over time, and that as a governance system is formalized, more bureaucracy is added to specify not just who holds power or authority, but how that power is transfered (boundary rules), and how responsibilities assigned by an authority are directly mapped to outcomes (aggregation rules).

\item I caution against judging the addition of new rules, and the expansion of existing rule types (authority rules in Regime 3) as either positive or negative evolution of the governance system. It should be noted that these play functional roles. In later regimes, ICOADS governance assumes a self-preservation role as it fights against funding shortfalls and threats of obsolesce. The rules enacted to guard against these changes are a manifestation of those state variables.  
\end{itemize}

Drawing on the data analysis above, I now answer the formally stated case study research questions.\\

\subsection{Research Question 1}

\textbf{RQ 1} \emph{What are the different governance models that ICOADS has effectively used to manage shared resources over it's thirty year existence (1983-present)}

ICOADS governance regime evolved steadily over its thirty year period, with only three definitive shifts. The are marked, as a I described in Chapter 4, by the need to produce new releases, but also by unique forms of selective pressures, and requisite responses. 

ICOADS first regime used what appears to be a mono-centric system of governance. This model assigned rules and coordinated actions between only the partners that \emph{produced} COADS release 1; namely NOAA, NCAR, and CIRES. A monocentric system can be defined as one where a “...vested in a single decision structure that has an ultimate monopoly over the legitimate exercise of coercive capabilities.” (Ostrom 1972 as quoted in McGinnis 1999, p. 55–56). This evidenced by the types of rules that are established. Position and authority rules are each aimed at establishing a basis of power: Who is in charge? What responsibilities do those positions then hold for the governance of the resource set? 

Who assumes authority, in Regime 1 is based on contribution to the release, namely the project partners that first coordinated around Tape Deck Families, worked to aggregate and bring together data sources, developes software and statistical trimming techniques, and then provided the computing power to make such processing possible. Although prohibitively expensive the authority vested in these institutions draws upon a normative framework from the 19th century (Maury) in `												

ICOADS second regime, which was marked by the introduction of a parent organziation in JCoMM and a set of formal workshops through which governance was determined. In effect, these semi-formalisms shifted the model from monocentric to self-goverance, where self governance is described as the  ``capacity of communities to organize themselves so they can actively participate in all (or at least the most important) decision processes relating to their own governance.'' (McGinnis, 2011) This is evidenced by the monocentric governance model being, at least symbolically, voted upon by stakeholders at CLIMAR 1 (1999). Introducing a new name and a new system of releasing data, communicating with end-users, and coordinating the tasks of improving ICOADS data for use in large-scale climate efforts was done, almost exclusively through informal participation of community members, but served nonetheless to shift the power structure from a centralized to a decentralized form.  

The shift between ICOADS second and third regime includes a nesting of its self-organization into a polycentric model. Although governance model became much more formal,  instead of reverting to a centralized form of governance, it became nested within a number of national, international, organizational, and disciplinary jurisdictions of authority. 

Polycentricity is defined by Aligica and Tarko as ``structural feature of social systems of many decision centers having limited and autonomous prerogatives and operating under an overarching set of rules.'' (2011), it was defined by Michael Polanyi in the The Logic of Liberty (1951) and later used by Vincent Ostrom in the 1970's during debates about the efficiency of administrative structures in American municipalities (Aligica and Boettke, 2009). Critical to the study at hand, is the fact that Polyani originally developed the idea of polycentricy to describe the governance of scientific knowledge production - which pursued knowledge in a variety of different disciplines, and was hence subject to a variety of different values and norms, but was able to effectively function because there was no appeal to an ultimate power authority. As Aligica and Tarko unpack this portion of Polyani's concept,``...an abstract and underoperationalized ideal cannot be imposed on the participants by an overarching authority. Thus, the authority structure has to allow a multitude of opinions to exist, and to allow them not just as hypotheticals but as ideas actually implemented into practice. The attempt to impose progress toward an abstract ideal is doomed to failure, as progress is the outcome of a trial-and-error evolutionary process of many agents interacting freely.'' (2011) It is the absence of centralized authority and power which allows for methods of production to be distributed in contemporary research and development settings, and as the latest regime in ICOADS has demonstrated, the division of labor within these types of systems is becoming increasingly similar to collective action platforms which employ peer-production. 

ICOADS governance is, a versioned and stylized form of a polycentric model. At the uppermost level is the World Meteorological Organization\footnote{The WMO is, itself, nested within the United Nations} which promulgates resolutions and standards for all of meteorology to follow. Nested within the WMO is the Joint WMO-IOC Technical Commission for Oceanography and Marine Meteorology (JCOMM), an intergovernmental that promotes standards, and coordinated marine observation systems, as well as data management. Nested within JCoMM are a series of workshops, CLIMAR and MARCDAT, which coordinate the research activities and agendas of marine climatology. The ``backbone of this work'' is ICOADS (Kent et al., 2001), a knowledge commons whose governance is now shaped by a steering committee with representatives from seven different international research centers and universities. Each representative is further nested within archives, research groups within these organizations, as well as the informal institutions that their own research contributes to, both beyond and within marine climatology.  

ICOADS polycentric system of governance operates efficiently because there is no one authority that transcends any one level of governance. Each level is able to appeal to a different level for directions on standards, operating procedures, or - as the policy of WMO Resolution 40 stated ``Shall'' and ``Should'', but there is no legal recourse for non-compliance. There is no polycentric police, judge, or jury. There is no legal or financial obligation, for example, if the organizations that signed a ``letter of intent to cooperate''  choose to leave ICOADS.  

The polycentric governance of ICOADS creates a freedom to operate within ones own nation, discipline, or technical competence, but at the same time provides a rule set, and and enforcement strategies at various levels, types, and sectors of governance. 

[Show through examples ]


\subsection{Research Question 2}



\textbf{RQ 2} What causes a governance system to shift from one regime to another?

Following Smith, Stirling, and Berkhout (2005), I will attempt to explain a regime shift as a function of two processes:

\begin{enumerate}

\item Shifting selection pressures (either external or internal pressure to change) bearing on the regime, and

\item The coordination of resources available inside and outside the regime to adapt to these pressures. 
\end{enumerate}


\textbf{Shifting selection pressures}

In regime one, selection pressures were manifested by competition and resource availability. As described above, competition for a niche within the cliamte science community for a comprehensive ocean-atmosp

Competition, Resource Availability	Competition, Technological Innovation	Obselescence, Resource Availability



What Geels and Schot call pathways to transition between sociotechnical regimes which are meant to refine the idea of multi-level or mutli-scale in polycentric governance. The approach conceives of regime change through four sustainability pathways called transformation, reconfiguration, technological substitution, and de-alignment & re-alignment (Geels and Schot, 2007). 

\subsection{Research Question 3}

\textbf{RQ 3} Which types of disturbances are ICOADS resilient or vulnerable to?


Distrubances are changes to a variable state. Selective pressures are a kind of disturbance - in that they explain what 


\subsubsection{Resilience}


The types of disturbances that ICOADS  are resilient to... 

\subsubsection{Vulnerabilities}

The selective pressure ``Obsolesence'' should not be read a a characterization of ICOADS as obsolete, only that it faces such a labeling as other products receive increased attention. It is in this sense that ICOADS is becoming an infrastructure like entity - it is the upstream input to many downstream innovations. Naturally, improving the efficiency and quality of the upstream resource can lead to many positive side-effects, but it is rare that these types of resources are 

``remains invisible until breakdown'' (Star and Ruhleder, 1997)


Vulnerable to competition, and funding... 

ICOADS implementing IVAD as a method of heavyweight peer production is a source of resiliency, it is shortening a feedback loop 

``Whats unqiue about IVAD, by design, is that when a user selects a particular subset they get references and explanations about the adjustments or bias corrections that have made along with the dataset. So submissions to IVAD necessarily include a component of peer review - one needs not only to document the types of corrections made, but also demonstrate how in the formal literature these corrections have been used, and typically this also includes some statement about what improvement. ''

\subsection{Conclusions from Case Study}

Moving from behavioral level of ``motivations'' and ``practices'' of sharing, producing, and provisioning research objects, including data (Faniels and Jacobson, 2011), software (Howison and Herbsleb, 2012), or computing (Ribes and Finholt, 2009; Kee and Brown, 2012) to institutional level analysis requires a number of new  approaches, one of which is introduced here. I show how a systematic approach to coding variables related to a particular aspect of this resource system can help reduce the complexity of 

- Metric diversity, in this case, is strongly tied to diagnostics; an indicator of healthy or robust knowledge commons may indeed be a diversity of metrics for evaluating success, attainment of goals, or more broadly conveying the importance of a resource set.

\subsubsection{Path Dependence and Lock-in}
A missing component of analysis in the current literature 

\subsubsection{'The Business of Public Goods'}

Public Entrepreneurship ... 

\subsubsection{Polycentric Governance: Principle of Complimentarity}

Complex systems, whether observed through direct methods like ethnography, indirect methods like informetrics, or through explanations of participants within the system, such as the interpretive itneractionist approach, will suffer from the principle of complimentarity - this holds that there are generally four ways of describing a system at multiple levels :

Redundant - a redundnant system description is when one actors description can be fully incorporated by another

Equivalent - an equivalent system description takes place when two actors give the same, description of a system

Independent - Includes two system level desctiptions with no overlap 

Complementary - When a set of descriptions lacks any of the above. Complimentarity usually results from partial descriptions, which can be (typically) eliminiated if we pursue more details, and more (Easterbrook, 2004)

However, the principle of complimentarity holds that complex systems can only ever be partially described, and as such they can 






This dissertation began by noting the differences between scientific knowledge, and the objects that result from knowledge production, notably that the former is assumed to be a public good while the latter is subject to only a few effective management regimes. Explaining how it is that knowledge functions, 

For insance, Velden unpacks theory selection with Gläsersian terms, noting that ``Researchers orient their actions towards creating knowledge that they can offer as novel contributions to the shared knowledge base. Hence the shared knowledge base ensures coordination of a collective of autonomous producers.  This coordination is decentralized, and not enforced by institutions or direct coordinating actions. The social order of a scientific production community is an emergent property'' (p. 262)


Velden's model concieves of openness (and cooperation) and secrecy (and competition) in scientific fields as a systemic tension inherent to the collective production of scientific knowledge. 


\section{Cross-case Comparison}

\subsection{Knowledge-Commons Framework}

[Look to the concluding chapter]

\subsubsection{The Genomic Commons}


Contreras contribution to the knowledge commons literature is a rich and detailed history of what he calls the Genomics Commons, or databases and archives of genomic data that have emerged from new sequencing technologies and large-scale international collaborations such as the Human-Genome Project (2014)


 and the surrounding policies which allowed for a genome commons to emerge. As he notes, this case study provides an empirical challenge to the accepted wisdom of intellectual property rights and state governance 

The major 

\subsubsection{Galaxy Zoo, and the Astronomy Commons}

Madisson pursues a comparative project asking ``How do the outcomes produced by commons governance differ from outcomes that might have been available if alternative governance had been employed?'' He uses a case study of Galaxy Zoo, as a knowledge commons, to compare with the Nearby Supernova Factory project, whose organizational structure resembles a traditional firm. 

Like Contreras, Madisson suggests that the Galaxy Zoo project is not only exemplary of the Mertonian norms of science, but that it exemplifies Hagstrom's claim that science as a ``gift culture'' trading in ideas instead of material gains. This is a claim supported by an observation that the project's PI, Chris Lintott, forefitted his proprietary claim to data transcribed by crowdworkers, and the project's commitment to the credit and acknowledgement of these contributions in resulting journal publications( p. 235). 

Madisson makes an important point that the openness of resource sets, both those produced and provisioned by Galaxy Zoo, is not a binary distionction; it is instead a continuum of sorts. The data, and underlying software that produces Galaxy Zoo is free, and openly accessible, but in the case of the data, it was only realesed to the public after an intial round of project specific publications were in press (). 

Another major conlcusion from this work is that ``...institutional order and knowledge governance, such as commons, are mutually constitutive. One largely exists because of the other.'' (Madisson, p. 211)

\subsubsection{Urea Cycle Disorder Research Network Commons}

\subsection{Comparing Results}

One of the ways to initially compare acorss these different knowledge commons is to ask what the background environment supplies in the way of a ``default'' assumption about the production and sharing of resources. Since each case study is embedded in a scientific domain of knowledge production, the disicipline or field specific to this context is likely to have some influence on these practices. Whitley's work on the differences of knowledge production practices in fields of science provides a helpful framework for understanding these issues.

Whitley describes ``the nature'' of intellectual fields through variations in two dimensions: mutual dependence and task uncertainty.

1. Mutual Dependence is the dependence on another field for making contributions to science, or the research of other scientists in doing work within one's own field. As Fry and Talja point out ``‘Mutual dependence’ also accounts for the extent to which a field adopts evaluation criteria and standards from other fields for the assessment of work externally produced, rather than developing its own criteria.'' (2007, p. 118)

2. Task Uncertainty is the degree that members of a field understand and articulate problems to be solved, or, agree on methods or techniques needed to solve these problems.

We should expect that for a commons to form in the first place there should be a low degree of task uncertainty. Producing a shared resource, in some ways, is agreeing upon the validity or meaning of that resource. Similarly, we should expect that each doamin invested in commons-based work has a high degree of mutual dependence; asI've argued throughout, the commons approach is a hybrid institution especially well-suited for sustaining knowledge production dependent upon collective action. 

This is a helpful framework though, because it shows a certain set of variables which demonstrate when, and under what circumstances a commons may be an effective governance regime for scholarly fields. It also helps dilieneate some of the overaching and broad claims that are made about Mertonian norms applied to science. To go even further, I would argue that Merton's norms are an awkward framework to be applied in a research agenda whose major goal is, at least in part, stated as a  providing an alternative to functionaist notions of property rights efficiency. They are awkward becuase Merton's normative theory is grounded in functionalism - he sees the actions, behaviors and adopted rules of a culture through latent and manifest functions that are, above all else, constructed for the sake of preserving a culture. The flexibility, nestedness, and 

\subsection{Research Question 1}

\subsection{Research Question 2}

\subsection{Conclusion}

\section{Chapter Summary}

Metric Diversity:

``sponsors allocating resources to different projects still need a means for assessing the aggregate value of the project to the scientific community. Project reports  seek to communicate  this aggregate value  by painting a picture of the benefits individual users accrue through  the  software,  accompanied  by  an  optimistic  estimate of the number of users. For grant-based projects that are  not  sold  for  money, user counts serve as the currency for success'' (Batcheller, 2011 p. 176)

