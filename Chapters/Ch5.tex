\chapter{}

\section{Introduction}
\emph{In this chapter I summarize the findings of the ICOADS case study, and answer three case study specific research questions. I then review three previously completed case studies using the Knowledge Commons Framework, and compare and contrast findings from all four cases. I use this cross-case comparison to answer the overall research questions of this dissertation.}

\section{Case Study: ICOADS}

The case study data organized by the Knowledge Commons Framework in Chapter 4 included the following categories: Background Environment, Community Attributes, Governance and Rules-in-Use. The Governance sections were divided into three ``action arenas,'' each consisting of a set of governance outcomes. In the following sections, I review this data in the form of coded state variables for each regime. State variables characterize a system at a particular period of time (Walker et al., 2005). When formally defined, they can guide the work of comparing between system states, allowing for an analysis that is both reproducible by other researchers and reusable in new contexts (Cox et al., 2014). The process of developing a coding scheme for these variables took the following steps:

\begin{enumerate}
\item Identifying relevant variables from the Socioecological Systems Meta-analysis Database (SESMAD) project (Cox et al., 2014) for a sociotechnical system.
\item Identifying relevant literature to generate variable definitions.
\item Coding each regime based on relevant data from the ICOADS case study. 
\item Reviewing the codings with my key informant for the sake of external validity
\item Iteratively editing definitions for consistency .
\end{enumerate}

The outcome of this work is summarized in the table below: 

% Please add the following required packages to your document preamble:
% \usepackage{graphicx}
\begin{table}[h]
\resizebox{\textwidth}{!}{%
\begin{tabular}{|l|c|c|c|}
\hline
\textbf{Variable Name}             & \textbf{Regime 1}                  & \textbf{Regime 2}                     & \textbf{Regime 3}                   \\ \hline
Begin date                         & 1981                               & 2001                                  & 2011                                \\ \hline
End Date                           & 2000                               & 2010                                  & Present                             \\ \hline
Description                        & In Text                            & In Text                               & In Text                             \\ \hline
Multiple levels                    & Single-Level Governance            & International Governance              & International Governance            \\ \hline
Institutional diversity            & Low                                & Medium                                & Medium                              \\ \hline
Centralization                     & 4 (Highly Centralized)             & 3 (Somewhat Centralized)              & 2 (Somewhat Decentralized)          \\ \hline
Governance scale                   & State-based                        & International                         & International                       \\ \hline
Scale match                        & No-Match                           & Match                                 & Match                               \\ \hline
Selective Pressures                & Competition, Resource Availability & Competition, Technological Innovation & Obselescence, Resource Availability \\ \hline
\multicolumn{1}{|r|}{Articulation} & Medium                             & Low                                   & Low                                 \\ \hline
\multicolumn{1}{|r|}{Capital}      & High                                & High                                  & High                              \\ \hline
\multicolumn{1}{|r|}{Coordination} & Low                                & High                                  & High                                \\ \hline
Governance trigger                 & Slow  Continuous Change            & Slow  Continuous Change               & Sudden Disturbance                  \\ \hline
Diagnostic Capacity                & Neither                            & Treating proximate causes (symptoms)  & Treating underlying drivers         \\ \hline
Autonomy                           & High- complete autonomy            & Moderate- some autonomy               & Moderate - some autonomy            \\ \hline
Metric diversity                   & Low- no metrics for success        & Medium- some metrics for success      & Medium- some metrics for success    \\ \hline
Rules-In-Use                       & Position, Authority                & Information, Payoff                   & Boundary, Aggregation, Scope        \\ \hline
Policy instrument                  & Incentive Based Insturments        & Incentive Based Instruments           & Input and Output Based Standards    \\ \hline
\end{tabular}
}
\caption{Coded State Variables from the each regime of the ICAODS case study.}
\label{my-label}
\end{table}

Next, I summarize observations from the coded state variables. Formal definitions are offered before each explanation of how the variable was coded for the ICOADS case study\footnote{The properties that correspond with each state variable can be found at http://git.io/vJnPX.} Just as the previous chapter, I occasionally draw upon interview data from ``interactionist'' studies. Where this is the case, I distinguish between participants with the following convention - ICOADS Steering Committee (ISC) and randomly assigned number (e.g. [ISC-06] Participants are not identifiable through this coding or the selection of interviews used in this chapter.\\  

\subsection{Discussion of Coded State Variables}

\textbf{Institutional diversity}\\ 

\emph{Definition: ``Ostrom (2005) has argued that institutional diversity is important for the same reason that biological diversity is important: that different institutional arrangements are frequently a response to local conditions and thus a diversity of arrangements are needed in order to adapt to a diversity of conditions.'' (Cox et al., 2014) This is particularly true of conditions in a sociotechnical system which couple diverse technologies and people. A diversity of arrangements used to sustain a sociotechnical system can also be seen as a way to avoid path dependence}\\

From Regime 1 to 3 ICOADS went from having low institutional diversity to a medium. Low institutional diversity implies that that a homogenous set of institutional arrangements were applied to a diverse set of resources. Originally, COADS was governed by just two institutions (both federally supported research laboratories). Over time, this resulted in social conflict which led to the introduction of new institutional arrangements - including ICOADS, JCoMM, and the ICOADS Steering Committee - all of which can be seen as evidence of the project becoming more heterogeneous in its governing structure.\\

\textbf{Centralization}\\

\emph{Definition: A centralized governance system has few actors/actor groups that hold a disproportionate amount of authority of over actors or parts of a commons. More decentralized governance systems have flatter hierarchies.}\\ 

Similarly, ICOADS governance became less centralized over time - Regime 1 was ``highly centralized'', and Regime 2 and 3 became a ``somewhat decentralized'' as it continued to decentralize its power and decision making authority. This is in line with institutional arrangements becoming more diverse, as described above, and the scale of the governance model shifting from state to international, as discussed next.\\

\textbf{Governance Scale}\\

\emph{Definition: This variable defines the scale at which a governance system operates, including:
\begin{itemize}
\item International regime: more than one country is involved;  
\item State-based (national) policy: at the level of one country; 
\item Sub-national policy: within a country (e.g., states, provinces, regions);  
\item Local: a finer scale than subnational (e.g., community-level, several laboratories, etc.)
\end{itemize}}\\

The governance scale variable is meant to code what jurisdiction the governance system has authority over. For instance, can the governance model apply rules, sanctions, bans, or assign tasks at state, national, or international levels? ICOADS moved from a state-based national scale to an international scale; first with granting partners voting rights in the renaming of the project (in 2001), and then further formalizing the internationalization of the project with the signing of a letter of intent to cooperate (in 2013).\\ 

\textbf{Scale Match}\\

\emph{Definition: Scale match explains the relationship between a governance scale, and a resource scale. For instance, spatial mismatches occur when the spatial scales of management and the spatial scales of a knowledge commons do not align appropriately.  For example, if a local community is attempting to manage an infectious disease reagent, there is a mismatch in the scale of management and the technologies range of use.}\\

The scale of a governance can be coded, within different states of a system, as having a binary ``match'' or ``no-match.'' In ICOADS first regime, a set of internationally archived records are used to produce resources that are provisioned with a state-based governance model. This sets up a tension where the broader ICOADS community felt there was not a match between the source of the inputs (international weather records) and the governance of the outputs (a state-based model). Regime 1 therefore does not provide a match between stakeholders and governance. In Regime 2 and 3, governance expands through the formation of JCoMM and this results in an international governance structure that now matches the scale of the resources and their users, producers, and provisioners.\\ 

\textbf{Selective Pressures}\\

\emph{Definition: Selective pressures are understood to cause governance adaptation. As Smith, Stirling, and Berkhout write, ``Without at least some form of internal or external pressure...it is unlikely that substantive change to the developmental trajectory of a regime will result. ''(2005) }\\

The types of pressures exerted on ICOADS governance relate to its shifting between regimes. These variables are meant to define and provide context for those pressures, but these categorizations should not be read as directly causing a regime shift. They are one of many stresses, disturbances, and motivations for a regime shit.

For example, the first shift between regime 1 and 2 is a result of ``competition'' and ``resource availability'' pressures. Competition for users, and for satisfying the need for a comprehensive marine dataset influenced ICOADS leadership to formally recognize the contribution of its international partners, as well as merge with its chief competitor, the Marine Data Bank. By expanding the scope of legitimate stakeholders for producing and provisioning the commons, ICOADS governance also necessarily had to become less ``Centralized'', increase its ``Institutional Diversity'', and adjust its ``Governance Scale''. These three variables (Centralization, Institutional Diversity, and Governance Scale) correlate strongly with ``Selective Pressures'' when a regime shift occurs.  

Selective pressures also have three sub-variables which are meant to further categorize responses from the commons governance.\\

\textbf{Sub-variables of Selective Pressures}\\

\emph{Articulation, Capital, and Coordination}\\ 

\begin{itemize}
\item Articulation explains the extent to which a governance model is capable of recognizing, describing, or defining selective pressures. 
\item Capital, both social and technical, explains the extent to which a governance model can access and extract resources from a network of stakeholders.
\item Coordination explains the extent to which a governance model is able to organize a response to selective pressures
\end{itemize}

In Regime 1, ICOADS stakeholders had a partial (medium) ability to articulate problems that were introduced by ``competition'' and ``resource availability''. This regime also had high social and technical capital , as was evidenced by the allotment of supercomputing time to develop COADS 1, and the ability to coordinate trading of archived marine data amongst international partners. The coordination of these partners, beyond simply contributing data, was somewhat low - as evidenced by early literature that described the project as a ``national'' effort (e.g. Slutz et al., 1983) 

In Regime 2, selective pressures include ``technological change'', seen through innovations in the assembly of reanalysis data products, and ``competition for resources'', which were due to the proliferation of high quality subsets of ICOADS. Articulation of these problems remained low; the number of sub-sets were so numerous and so diffuse that they could no longer be included in ICOADS documentation. Both social and technical capital, as well as coordination within the second regime were high as ICOADS was able to coordinate development for integration with reanalysis projects, and position the resources to be used in major international climate assessment reports.

In Regime 3, the most current regime, ICOADS suffered initially from the selective pressure``resource availability,'' but seems to have responded to some of these problems by coordinating international partners for support. However, ICOADS continues to face selective pressure to innovate, which it has not been able to respond to, and thus risks being branded ``obsolete'' in relation to already existing high quality subsets and reanalysis products. 

In the overall state of regime 3, articulation was low. This is evidenced by the fact that the ESRL defunding seemed to take even most senior NOAA project partners by surprise. But, this also demonstrated that levels of social capital were high in the third regime, as NOAA consequently acknowledged that restoration of partial funding for ICOADS was due to an outpouring of support from the international community. Technical capital was low, as the interruption of work on IMMA software greatly delayed progress on an anticipated and improved ICOADS third release\footnote{This is an admitted flaw to the way that the ``Capital'' variable is currently coded: When social and technical capital do not align the results are a compromised ``medium'' coding, when in fact it should be ``high'' and ``low''}. Coordination was high in Regime 3, as demonstrated by the ability of ICOADS to respond with letter of intent to cooperate, and the formal assignment of tasks to project partners that agreed to work towards a third release. 

Selective pressures, along with governance triggers and diagnostics are all highly related variables and should have a consistent relationship, which I describe in the next two sections.\\

\textbf{Governance Triggers}\\

\emph{Definition: This variable is meant to provide information about the accumulation of selective pressures. Governance triggers are characterized by the notion of disturbance from socioecological systems management. A disturbance is some event, or set of selective pressures which cause a change in state variables. Governance triggers can be either ``sudden disturbances'', such as a defunding event from NOAA that shifts state variables; or, a governance trigger can consist of ``slow continuous change'', such as the obsolescence of a technology over time as ICOADS is currently experiencing in its third regime.}\\

Regime 1 and 2 of ICOADS can be characterized as having slow continuous changes. Disturbances during these periods are related to organizing, creating rules, and fostering cooperation amongst diverse partners. Even during periods of scientific controversy, such as the discovery of anomalies in sea-surface temperature measurements in a historical period of the 1940-1960s (Rayner et al., 2005), there was a relatively stable ICOADS response and a cooperative effort to improve these errors (Woodruff et al., 2011). 

Regime 3, which is marked by an unanticipated defunding event, is the first time that sudden disturbance causes a shift in state variables. The ability to diagnose this problem, and coordinate responses to this event can be seen as important characteristics of a sustainable knowledge commons.\\ 

\textbf{Diagnostics}\\

\emph{Definition: Characterization of the scale at which a regime responds to a selective pressure, or disturbance. A diagnostic capacity can be to treat underlying drivers of a distrubance, treat the symptoms of a driver, or neither.}\\

The diagnostic variable attempts to categorize the response of a regime to selective pressures. This variable is related to governance scale and scale match, as well as the selective pressures sub-variables: 

\begin{itemize}
\item A governance with a strong scale match and ability to articulate, draw on capital, and coordiante responses will be able to successfully transition between regimes. 

\item Without a scale match, or effective articulation of selective pressures, the governance regime will lack the ability to make a diagnosis of why a selective pressure is occurring. Subsequently, the governance regime may respond to only the symptoms of the problem, and not address issues related to the underlying causes of that problem. 
\end{itemize}

Regime 1's governance had no scale match, a medium level of articulation, and low coordination to respond to selective pressures. The governance model failed to respond to issues of competition and resource availability; and, it neither treated underlying causes of the selective pressures, nor its symptoms until a regime shift was required to do so. The ability to shift governance models smoothly, was partially a result of the high social and technical capital that 

Regime 2 had a match between governance and resource sets, but a low ability to articulate the underlying causes of its selective pressures. Failing to recognize underlying problems related to measurement and evaluation of the project resulted in a sudden disturbance event that nearly ended the project. In treating the symptoms related to selective pressures like technological change and competition for resources, Regime 2 was able to coordinate a new method of scheduling releases, a new method of coordinating work (CLIMAR and MARCDAT), and begin development of a new database model through IVAD.

Regime 3 of ICOADS finally begins to treat underlying drivers of its disturbances, and social dilemmas - notably by formalizing commitments from international partners, and beginning to diversify the metrics that it optimizes its work for (e.g. Weber et al., 2011)\\  

\textbf{Metric Diversity}\\

\emph{Definition: The range of evaluative criteria that a project optimizes for, or is judged upon.}\\

All three regimes suffer from evaluation problems that stem, at least in part, from a lack of metrics that can clearly communicate ICOADS' impact to organizations that bear the costs of provisioning a shared set of resources. In discussing the first and second regime, metrics were rarely mentioned by my participants, nor in the literature of ICOADS. More broadly, there was no recognition throughout regimes one and two that an underlying problem in sustaining ICOADS was a lack of formative evaluation. Metric diversity, in this case, is strongly tied to diagnostics; an indicator of healthy or robust knowledge commons may indeed be a diversity of metrics for evaluating success, attainment of goals, or more broadly conveying the importance of resources to a broad set of stakeholders.\\ 

\textbf{Policy Instruments}\\

\emph{Definition: A taxonomy of the basic types of policies and institutions that a governance system uses in order to effect actor behavior and achieve commons outcomes. These include:  outcome, input, and usage-based standards; incentive-based instruments; and, information and insurance provisions.}\\

Policy instruments can be broadly categorized as either regulatory or incentive based. Regulatory policy instruments are aimed at effecting behavior of commons stakeholders through coercion or punishment. Incentive-based policy instruments are meant to elicit desired behavior through reward. In ICOADS, incentive-based instruments were used throughout all three regimes. Regime 1 and 2 relied exclusively on incentive-based policy instruments, as these regimes had little authority in the way of sanctions or enforcement of rules to establish any kind of regulatory policy about what kind of actions were, or were not accepted. By formalizing partner responsibilities in regime 3, including those related to the production of a third release, ICOADS governance has adopted some minor ``input and output standards''. In general, the lack of diverse tool uses has been problematic for ICOADS governance, which has not  \\ 

\textbf{Rules-in-Use}\\

To reiterate from previous chapters, rules here are understood to ``specify the values of the working components of an action situation; each rule has emerged as the outcome of interactions in an adjacent action situation at a different level of analysis or arena of choice'' (Ostrom et al., 1994, p. 41–42) Rule types inlcud: Position, Boundary, Authority, Scope, Aggregation, Information and Payoff. In the following section, I describe the rules-in-use for each of the three ICOADS governance regimes.  

\begin{table}[h]
\resizebox{\textwidth}{!}{%
\begin{tabular}{|l|l|l|l|l|l|l|l|}
\hline
                  & \textbf{Position}        & \textbf{Boundary}        & \textbf{Authority}       & \textbf{Scope}           & \textbf{Aggregation}     & \textbf{Information}     & \textbf{Payoffs}          \\ \hline
\textbf{Regime 1} & \cellcolor[HTML]{C0C0C0} &                          & \cellcolor[HTML]{C0C0C0} &                          &                          &                          &                           \\ \hline
\textbf{Regime 2} & \cellcolor[HTML]{C0C0C0} &                          & \cellcolor[HTML]{C0C0C0} &                          &                          & \cellcolor[HTML]{C0C0C0} & \cellcolor[HTML]{C0C0C0} \\ \hline
\textbf{Regime 3} & \cellcolor[HTML]{C0C0C0} & \cellcolor[HTML]{C0C0C0} & \cellcolor[HTML]{C0C0C0} & \cellcolor[HTML]{C0C0C0} & \cellcolor[HTML]{C0C0C0} & \cellcolor[HTML]{C0C0C0} & \cellcolor[HTML]{C0C0C0}  \\ \hline
\end{tabular}
}
\caption{Accumulation of rules across three regimes of ICOADS (1983-2015)}
\label{my-label}
\end{table}
\\
\textbf{Regime 1}\\

Rules in Regime 1 can generally be classified as fitting two types:\\

\begin{itemize}
\item \textbf{Position Rules} specify a set of positions that actors can play. Each position has a unique combination of resources, opportunities, preferences, and responsibilities.

\item \textbf{Authority Rules} specify which set of actions is assigned to which position. (McGinnis, 2011)  
\end{itemize}

In COADS initial governance - which was ad-hoc, and loosely coordinated - position rules are established by data provisioners whom assigned responsibilities to one another, and to end-users. Users were free to obtain data, and encouraged to report errors and participate in improving the archive (Woodruff et al., 1987). Data producers who contributed statistical techniques, algorithms, software, or data processing were assigned rights and responsibilities to stewardship of these resources, including any version of the release one that they obtained archived as part of this work. 

In an interview about this early process, one participant characterized the early coordination in the following way: ``...the problem is that, ICOADS has a single point of failure. I mean (PI) was ICOADS - he was the identity. And while we didn't have a communication problem through language or responsiveness, we bumped up against competing interests - it was a much bigger project and scope than even (PI) realized when we first started putting data out...before the CLIMAR meetings started.'' [ISC- 06] This demonstrates the centralization of power during this regime, and as such, requires each new regime to revisit authority rules.\\

\textbf{Regime 2}\\

The second regime introduces three new rule types to ICOADS governance.  

\begin{itemize}
\item \textbf{Information rules} specify channels of communication among actors and what information should, must, or must not be shared. 

\item \textbf{Payoff Rules} specify how benefits and costs are required, permitted, or forbidden to stakeholders.

\item \textbf{Scope Rules } specify a set of outcomes, such as what should be produced, what levels of maintenance a resource set requires. Scope rules are closely related to payoff and information rules. (McGinnis, 2011)  
\end{itemize}

Information rules developed during this regime are seen in the specifications of formatting data for exchange, and recommendations from the second CLIMAR meeting in 2004 which reads, ``There were seen to be shortcomings in the access to ICOADS data. There are many, overlapping sources of  data and products, and the problem of optimising data provision is complex. Many users are working with outdated versions of COADS. Often data are available, but it is difficult for the uninitiated to discover what is there. There should be a Web-based route map to the best available data which should be widely advertised to all the various user communities'' (Parker et al., 2004). The ``route map'' took the form of website developed to host specification documents, and link to relevant existing WMO standards (Woodruff et al., 2006). The process of creating this resource moved to formalize what information should be available, when, and to whom\footnote{ The development of a rule type doesn't imply that a social dilemma has been solved. In this case, blending of different data products and an overall growth of the archive leads to unmanageable complexity for end users. Introducing informational rules is a way to treat symptoms of a larger problem in managing a sustainable growth of the project, but rules aren't themselves symptoms of a solution.}

Payoff rules are established in granting international partners formal recognition through a name change. Recall that when asked about the process, one of my participants replied ``...that debate really focused attention on the fact that COADS should be looked at as an International dataset, because you know a lot of other countries were putting money and data into it.'' [ISC-01] The settlement of the name change debate brought with it a larger set of responsibilities for international project partners, and their engagement in provisioning and producing new interim products throughout the regime. 

Scope rules are another artifact of the CLIMAR and MARDAT meetings, where strategic directions are set, and future goals are established. Establishing scope rules also represents a significant shift in the governance model, as it becomes increasingly less centralized.\\

\textbf{Regime 3}\\

Two new types of rules are added, and one rule type is further developed in this regime:

\begin{itemize}
\item \textbf{Aggregation rules} (such as majority or unanimity rules) specify how the decisions of actors at a node are to be mapped to intermediate or final outcomes.  

\item \textbf{Boundary rules} specify how participants leave or enter positions of authority. (McGinnis, 2011)  
\end{itemize}

Aggregation and boundary rules are established by mechanisms introduced by the letter of intent to cooperate. When asked about how these rules are put into practice, a participant described this situation as follows, ``The governance is still in flux- the committee are new, the terms of references are still new, and we still have to formalize a bit more how members serve, how long do they come and go, etc.'' [ISC 04]

In March of 2015 when a discussion about the appoint a permanent chair for the steering committee was raised, the interim chair reminded the steering committee that he was retiring. and expressed no desire to remain in the role during retirement. Tellingly, no one understood nor knew the procedure to make a nomination for a new chair, or whether on-going questions about future support of ICOADS at NOAA should impact this choice\footnote{As it stands, no decision has been made, and existing by-laws offer little guidance on how to proceed.}. 

Additionally, the \emph{Authority rules} of previous regimes are both formalized and enacted in new ways in the third regime. This includes formally dividing responsibility for the future provisioning of ICOADS resources, and the creation of a set of core institutions - beyond the original USA partners of NOAA, NCAR, and CIRES - that formally acknowledge their intention to sustain ICOADS over a five year period (beginning, December 2013).\\ 

\textbf{Summary of Rules and State Variables}

The systematic approach to characterizing state variables shows a number of important relationships between these variables and their impact on governance. 

\begin{itemize}
\item Governance scale, scale match, and the sub-variable articulation all correlated strongly with governance triggers. These are not causes for, but contexts of state changes such as a regime shift. 

\item Selective pressures, and the sub-variables articulation, capital, and coordination are illustrative of both the ways in which governance states experience, and react to disturbances. In ICOADS case, these variables are strongly related to the rules-in-use, and policy instruments used by each regime. 

\item Diagnostic capabilities help explain the effectiveness of a governance model in understanding the underlying causes of governance triggers, and in turn, provide helpful context for understanding the impact of selective pressures.  

\item The analysis of rules-in-use shows how rule types accumulate over time, and that as a governance system is formalized, more bureaucracy is added to specify not just who holds power or authority, but how that power is transfered (boundary rules), and how responsibilities assigned by an authority are directly mapped to outcomes (aggregation rules).

\item I caution against judging the addition of new rules, and the expansion of existing rule types ( e.g. authority rules in Regime 3) as either positive or negative evolution of the governance system. The more important variable related to sustainability is the match between governance, stakeholders, and resource sets.

\end{itemize}

Drawing on the data analysis above, I now answer the formally stated case study research questions. In each section, I re-state the research question, offer a summary of the findings, and then explain this answer with a supporting narrative. \\

\subsection{Case Study Research Question 1}

\emph{What are the different governance models that ICOADS has effectively used to manage shared resources over its thirty year existence ?}\\

\textbf{Summary}\\ 
ICOADS has moved from a monocentric governance model, with a single center of authority, to a polycentric model, which nests rule making at various institutional levels and scales. The polycentric model has a comparable effect as the ``comedy of the commons'' where public goods are efficiently provisioned by encouraging actors to sustain a resource set through co-production of its resources. However, polycentric models are not a panacea. The nesting of authority at different levels may create interdependencies among institutional networks that prove costly in the long-term.}\\

\textbf{Supporting narrative}\\
ICOADS governance evolved steadily over its thirty year period, with only three definitive shifts in models. Each shift is marked, as a I described in Chapter 4, by the need to produce new releases, but also by unique forms of selective pressures, disturbances, and requisite responses. 

ICOADS first regime used what appears to be a mono-centric system of governance. This model assigned rules and coordinated actions between only the partners that \emph{produced} COADS release 1; namely NOAA, NCAR, and CIRES. A monocentric system can be defined as a ``... single decision structure that has an ultimate monopoly over the legitimate exercise of coercive capabilities.'' (Ostrom 1972 as quoted in McGinnis 1999, p. 55–56) This is evidenced by the types of rules that are established; Position and authority rules are each aimed at establishing a basis of power: Who is in charge? What responsibilities do those positions then hold for the governance of the resource set? In regime 1 the power of authority was not distributed evenly, the lack of governance and resource scale match, and the policy instruments (which were incentive-based) each led to the monocentric governance model failing. 
											
ICOADS second regime was marked by the introduction of a parent organization, JCoMM, and a set of formal workshops, MARCDAT and CLIMAR. In effect, these semi-formalisms shifted the model from monocentric to self-governance, which is defined through the  ``capacity of communities to organize themselves so they can actively participate in all (or at least the most important) decision processes relating to their own governance.'' (McGinnis, 2011) This is evidenced by the monocentric governance model being, at least symbolically, voted upon by stakeholders at CLIMAR 1 (1999). Introducing a new name and a new system of releasing data, communicating with end-users, and coordinating the tasks of improving ICOADS data for use in large-scale climate efforts was done, almost exclusively through informal participation of community members, but served nonetheless to shift the power structure from a centralized to a decentralized form.  

The shift between ICOADS second and third regime includes a nesting of its self-organization into a polycentric model. Although ICOADS governance model formalized it did not move towards a command control, or centralized model; Instead, ICOADS nested itself within a number of national, international, organizational, and disciplinary jurisdictions of authority. 

Polycentricity is defined by Aligica and Tarko as ``structural feature of social systems of many decision centers having limited and autonomous prerogatives and operating under an overarching set of rules.'' (2011). The idea of a polycentric order was first offered by Michael Polanyi in the ``The Logic of Liberty'' (1951) and later used by Vincent Ostrom in the 1970's during debates about the efficiency of administrative structures in American municipalities (Aligica and Boettke, 2009). Critical to the study at hand, is the fact that Polyani originally developed the idea of polycentricy to describe the governance of science - which pursued knowledge in a variety of different disciplines and was hence subject to a variety of different value systems. But, what Polanyi described was the oddity that such a large enterprise could effectively function with no appeal to an ultimate power authority. Instead, informal rules and norms are learned through initiation, apprenticeship, and the evolution of social orderings. He was, in some sense, identifying this mode of knowledge production as a form of commons. 

A passage from Aligica and Tarko clarifies an important point about Polanyi's writing of polycentric systems in science,  ``...an abstract and underoperationalized ideal cannot be imposed on the participants by an overarching authority. Thus, the authority structure has to allow a multitude of opinions to exist, and to allow them not just as hypotheticals but as ideas actually implemented into practice. The attempt to impose progress toward an abstract ideal is doomed to failure, as progress is the outcome of a trial-and-error evolutionary process of many agents interacting freely.'' (2011) 

Polycentricity then is the absence of a centralized power, combined with a nested structure which allows the ``many centers of authority'' to cooperate effectively through trial and error. The emergence of cooperation through interactions is the crux of both Ostrom's work, and that of David Axelrod in his seminal work on game theory (1982). 

In contemporary science settings, such as ICOADS, the division of labor is becoming increasingly depends upon collective action, and thus the emergence of peering, including peer production, is itself a stylized form of a polycentric model of governance - a ``trial and erorr process, with many agents interacting freely'' but organizing themselves at different levels of authority and rule-making, to produce complex goods. In ICOADS, this nesting of authority takes the following shape: 

\emph{At the uppermost level is the World Meteorological Organization\footnote{The WMO is, itself, nested within the United Nations} which promulgates resolutions and standards for all of meteorology to follow. Nested within the WMO is the Joint WMO-IOC Technical Commission for Oceanography and Marine Meteorology (JCOMM), an intergovernmental body that promotes standards, and coordinated marine observation systems, as well as data management. Nested within JCoMM are a series of workshops, CLIMAR and MARCDAT, which coordinate the research activities and agendas of the field of marine climatology. As Kent et al. put it, the ``backbone of this work'' is ICOADS (2001), a knowledge commons whose governance is now shaped by a steering committee with representatives from seven different international research centers and universities. Each representative is further nested within laboratories, research groups, and the various sub-fields that their research contributes to.}

ICOADS polycentric system of governance operates efficiently because there is no one authority that transcends any one level of governance. Each level is able to appeal to a different level for directions on standards, operating procedures, or guidance.  There is no polycentric police, judge, or jury. There is no legal or financial obligation, for example, if the organizations that signed a ``letter of intent to cooperate''  choose to leave ICOADS.  But there are rules, and there are social repercussions for violating norms related to the historical background from which a commons emerges. In the case of ICOADS, the polycentric governance model creates a freedom to operate within one's own nation, discipline, or technical competence, but at the same time provides a rule set, and and enforcement strategies at various levels, types, and sectors of governance. 

The polycentric model of governance has two important implications for sustainability within knowledge commons:

\begin{enumerate}
\item As Carole Rose explained in the ``comedy of the commons'' (1986), public goods are successfully provisioned, in part, because the users of these resource system are made complicit in their maintenance, and upkeep. Such a complex system of duty and responsibility is impossible to design without some amount of rules, authority, and norms being nested within a community structure. Similarly, the polycentric model of governance in ICOADS has become successful because it enables loosely connected stakeholders, even those that are not formally part of the governance body to contribute through letters of support, attendance and engagement in CLIMAR and MARCDAT, and through contributions to improving biases in the data -either formally by making submissions to IVAD, or informally through filing bug reports. Like the comedy of the commons description, the efficiency of such a complex institution would be difficult to predict, and even harder to purposefully design. By creating a symmetry to operate within the commons, the polycentric model is able to achieve a high level of efficiency across scales and levels of a knowledge commons.  

\item Nesting governance at different levels creates a certain amount of protection from disruptive disturbances by the very fact that the commons is coordinated across different levels and scales of governance. But, these interdependencies are also dangerous for the very fact that failure at a high level has the potential to cascade negative effects down to each subsequent level. For instance, federal budget disagreements about funding for basic science research cascaded down to the ICOADS project level in a rather dramatic fashion. 
\end{enumerate}

\subsection{Case Study Research Question 2}

\emph{What causes a governance system to shift from one regime to another?}\\

\emph{Summary: Regime shifts are described through the ability of a knowledge commons to coordinate responses to selection pressures. A regime shift is related to selection pressures, diagnostics, and a number of other state variables - but ultimately, a regime will shift either out of opportunity (to improve upon the current model) or our of necessity (to avoid collapse). The ability of any system to respond to pressures and return to a functioning state can be described as resilience. ICOADS, in responding to selective pressures has demonstrated an ability to reconfigure its governance model in order to adapt to changing selection pressures. This is due, in part, to its flexibility in adopting new rules as well as its ability to formalize its governance when faced with pressures related to financial support. Transitions between governance regimes, and adaptivity are suggested as a future direction for sociotechnical systems research.}\\

\subsubsection{Supporting Narrative}

Following Smith, Stirling, and Berkhout (2005), I will explain a regime shift as a combination of:

\begin{enumerate}

\item Selection pressures bearing on the regime, and

\item The coordination of the current governance model to adapt to these pressures. 
\end{enumerate}

In the following sections, I characterize the regime shift in ICOADS governance between the first and second, and the second and third regimes. I then describe the implications of these processes on ICOADS sustainability.\

\textbf{Regime Shift 1 to 2}

In Regime 1, selection pressures where characterized as ``competition'' and ``resource availability.'' 

Niche competition within the climate science community for an ocean-atmosphere dataset resulted in significant conflict between the UK Meteorological Office, and the USA partners that were producing COADS. While both offered high quality data products, COADS could not make up for the temporal and spatial coverage of records that MDB held, and MDB conversely did not have the data processing capabilities, nor a sufficient cost-recovery model to significantly expand MDB. 

COADS partners originally drew upon a principle of ``free and unrestricted exchange'' from the WMO Resolution 35, as well as the legacy of Lt. Matthew Maury in producing an open access resource. The cooperating COADS institutions forefitted their intellectual property claim to these early resource sets, and over time model of distribution helped COADS establish itself as a high quality data source within the climate science community. However, it was not a principle of openness, a resolution from a governing body, nor a naval legacy that would cause COADS to merge with its main rival, the MDB; but, instead, as my participant put it, ``So, why did they merge? They had to. The science wasn't getting better without it'' [ISC - 02].

The shift from a monocentric governance model to a self-governance model was a result of a collective desire to improve upon the current state of knowledge. This manifested itself in selective pressures like competition and resource availability, and the development of rules regarding what information must be shared amongst partners (information rules), and payoff rules, which specified who was to receive what benefits, and when. For example, the UK MET office would receive an advance copy of ICOADS processed data, with enhanced statistical subsets related to SST. This gave the UK MET the ability to capitalize on their shared data, and begin working towards valuable secondary ICOADS data, such as the HadSST products.\\ 

\textbf{Regime Shift 2 to 3}

A shift between regime 2 and 3 was precipitated exclusively by the loss of financial support to NOAA partners. This sudden disturbance created a threat that ICOADS would no longer be a reliable source of marine surface data. The major selection pressure in this regime is ``obsolesence'', as ICOADS is forced to coordinate a broad network of actors in response to mounting pressure that it will not be able to serve end users, and produce a valued third release. 

In coordinating a response, ICOADS project partners in the USA formalized agreements with international partners through a letter of intent to cooperate, which had the effect of further decentralizing the governance, and adding a new set of boundary, authority, and scope rules that would assign decision making power, establish voting mechanisms, and introduce new policy instruments for use by specific actors. This regime shift, like the one previous, had a nesting effect and contributed largely to ICOADS shifting from self-governance to a polycentric model. 

\textbf{Causation}

The explanations above make use of descriptive state variables, which I argue can help reduce the complexity of understanding a sociotechnical system's response to pressures. However, these descriptions are not pointing out direct cause and effect mechanisms, even where relationships between variables are observed. With this in mind, two broad conclusions can be drawn about the process of regime shifts in ICOADS:

\begin{enumerate}
\item As a sociotechnical system, ICOADS was able to absorb the effects of these pressures, and to coordinate responses which allowed for successful regime transitions. In this sense, ICOADS is a robust sociotechnical system by the very fact that it was able to effectively respond to these pressures. However, not all selective pressures and effects of a shift are equal. In the shift from regime 2 to 3 it is evident that there continue to be difficulties in producing a third release, and in fully realizing the potential of planned enhancements, such as the ICOADS value-added database (IVAD). This may mean that ``obsolescence'' is a more disruptive pressure than others, such as competition or resource availability, which likely effect all sociotechnical systems at some point. 

\item Cycles of adaptivity, and system's level resilience are an important future direction for knowledge commons scholars interested in sustainability. In Chapter 1, I offered a definition of resilience as the ``measure of the persistence of systems and of their ability to absorb change and disturbance and still maintain the same relationships between populations or state variables.'' (Holling, 1974) But, the linear description of moving from one state to another is likely to have intermediate steps glossed over in this initial analysis. A move towards an adaptive cycle model, as pictured below, may be more helpful in understanding the ways in which sociotechnical systems, and knowledge commons in particular, respond to selective pressures; and more broadly, how sociotechnical systems successfully transition between governance regimes. I return to this proposal in Chapter 6.
\end{enumerate}

\includegraphics[width=4in, height=2.5in]{ad-cycle}\\
\captionbelow{ \textbf{Caption:} A system - whether biological, ecological, organizational, etc. - is said to go through an adaptive cycle, where the ability to move between a back loop (e.g. \alpha to \Omega ) to a fore loop (e.g. r to K) is a function of a system's its resilience.}


\subsection{Case Study Research Question 3}

\emph{Which types of disturbances are ICOADS resilient or vulnerable to?}\\

\emph{Summary: ICOADS current vulnerabilities are due to a lack of institutional diversity, and metric diversity. Over time, ICOADS has proven an ability to increase its resiliency through a governance model that accommodates regenerative disturbances. Future work is suggested for the identification and further clarification of disturbance types, and responses.}\\

\subsubsection{Supporting Narrative}

Disturbances are simply changes to a system state. Recall that a definition of a scientific cyberinfrastructure is that, in part, they allow for the ``smooth operation of scientific work at a distnace.'' (Edwards et al., 2006) Disturbances in a sociotechnical system like ICOADS can then be seen as phenomena that effect the ``smooth operation'' of their functioning. 

In this analysis, I'll focus on two types of disturbances: 

\begin{enumerate}
\item Regenerative disturbances build the capacity of a system to withstand future change.
\item Disruptive disturbances cause damage to a system, some of which may be irreparable.\\ 
\end{enumerate} 

\textbf{Regenerative Disturbances}

As ICOADS moved from a monocentric to a self-organized governance model it became less centralized, and matched its governance model with its resource scale. This in turn allowed ICOADS to develop a capacity to respond to selective pressures such as competition, resource availability and technological innovation. Controversies over name changes, infighting over future directions of the project, the introduction of new international governing bodies, and the discovery of known errors in the dataset (Woodruff, et al., 2006), caused initial changes to state variables, but ultimately helped ICOADS to refine the types of rules it was adopting to meet these needs. The responses and transitions between states due to regenerative distrubances should be seen as ICOADS building capacity to withstand change. 

\textbf{Disruptive Disturbances}

In shifting to a polycentric model, ICOADS faced a number of stresses, including a partial loss of funding, the use of servers hosting a valuable subset of ICOADS, and three dedicated full time employees to the development of software crucial to a third release.  An interesting correlation between variables when there is a sudden distrubances is that incentive-based policy instruments were replaced with input and output standards. This might imply that instead of simply trying to entice cooperation from partners, boundary and aggregation rules were put into place to assign responsibilities for producing a third release, restrict what institutional privileges came along with signing a letter of intent to cooperate, specify who was entitled to vote on future directions of the projects, and more generally, specify expectations for future contributions. 

One of ICOADS vulnerabilities across all three regimes is a lack of metric diversity. As Cox et al., note, this leaves sociecological commons particularly vulnerable to change because it ``optimizes system performance around criteria which may be indicators of short term success, but poor predictors of long term sustainability.''(2014) The same is likely true of sociotechnical systems, which may optimize around certain metrics of success (e.g. citations in academia) which indicate, but do not predict a system's robustness. 

When asked about the loss of funding, one participant explained:

``...whoever was making that decision did not understand how important ICOADS is to the research community. How widely used it is ... because there was no statistics, there is no metrics of how many people would all of the sudden have their data products disrupted if ICOADS goes away'' [ISC 06]

What bibliometric indicators that are available to quantify ICOADS impact are incomplete, and often misleading. For example, in a small sample of the documents that cite ICOADS\footnote{For clarification, to ``cite ICOADS'' simply means that a paper which announced a new release, such as ICOADS 2.5 is cited. This data comes from the citation content analysis described in Chapter 3}, we found that 24\% (n=79) had cited the wrong ICOADS release paper for the data that the study was purporting to use (Weber, Worley, and Mayernik, 2014). This gives the illusion that COADS release 1 remains widely used in contemporary research\footnote{We controlled for the fact that these studies may have indeed been doing historical studies which used release 1 data.c}, when in fact that particular data release hasn't been available to end users since the late 1990's. 
\begin{figure}
\includegraphics[width=4in, height=2.5in]{ICOADS-citations}\\
\captionbelow{\texbf{Caption:}Distribution of citations to four major releases of ICOADS (COADS Release 1: 1987; ICOADS Release 2 1998; ICOADS Release 2.1: 2005; ICOADS Release 2.5: 2011)}
\end{figure}

Additionally, when we look at the total number of citations to these data papers, they are disproportionate to the number of unique users that download ICOADS data. For instance, ICOADS data release papers received a total of 201 citations from 2010-2012. During this same time period, 12,519 unique users downloaded at least 10 mb of ICOADS data from NCAR's Research Data Archive. This gives ICOADS a unique use download to citation rate of 0.016. 
\begin{figure}
\includegraphics[width=4in, height=2.5in]{ICOADS-downloads}\\
\captionbelow{\textbf{Caption:} Number of Unique Users that Downloaded \> 10mb of ICOADS data, 2010-12}
\end{figure}

A final vulnerability that ICOADS faces, which fits neither the regenerative nor disruptive disturbance typology, is related to the selective pressure of obsolescence. In the sociology of science, Merton coined the phrase ``obliteration by incorporation'' (1948, p. 27–28) to refer to processes where an original idea gains rapidly in popularity and use, and quickly becomes incorporated into a common stock of domain knowledge. Occasionally, this results in a failure to recognize the idea producer, because the idea becomes so widely understood that acknowledging its creator would seem redundant. The most immediate example is ``Einstein's theory or relativity'', but less iconic scientists have created lasting ideas only to be forgotten by way ``obliteration'' of credit through rapid ``incorporation'' of their work in the stock of common scientific knowledge. Garfield refined this idea with citations in describing how a novel idea, technique, or concept is occasionally so broadly and quickly adopted that it is fails to be formally cited at a comparable rate (Garfield, 1975).  

ICOADS continues to be used as an input to large-scale reanalysis projects, and simultaneously is subset into innumerable smaller-scale products, many of which receive more citations and funding than ICOADS. It suffers from obliteration by incorporation into the climate record - failing to be recognized properly at either scale. Frischmann has similarly described the process of public infrastructures being subsumed into public good categorizations - they provide the valuable inputs into downstream innovation, but receive little of the upstream attention of investment firms or private sector support (2005). And indeed, we see ICOADS take on many infrastructure-like features, including the salient fact that it is rarely paid attention to until it breaks down (Star and Ruhleder, 1997).

\subsection{Limitations of Case Study}

The case study presented above has answered three case study specific research questions, largely by data collected during ethnographic and informetric studies of ICOADS community. There are a number of limitations to the results presented here, including the small sample size (n = 1), the scale of the informetric work, and the ability to demonstrate causative effects of changes in state variables. Establishing causation for regime shifts, or disturbance response should not be seen as the goal of the approach taken in this case study. Systematically coding state variables simply allows for the complexity of a system to be reduced, the different components of the system to be accurately described, and the findings of this work to be meaningfully compared to other cases. To overcome some of these limitations, I next compare the findings of this case study with those from three additional knowledge commons. 
 
\section{Cross-case Comparison}

In this section I synthesize findings from across four different case studies of the knowledge commons, their evolution, and the background environments out of which they emerged.  I begin first with an overview of three previously completed studies, and then describe the ways in which these different domains produce knowledge. I use a set of state variables developed in the ICOADS case study to compare the current governance models of all four commons. Finally, I use these comparisons to answer the dissertation's research questions. 

\subsection{Knowledge-Commons Framework}

Madison, Frischmann, & Strandburg have over the last five years developed a research agenda, and framework around the idea of a knowledge commons. Their argument turns on the assumption that ``innovation and creativity are matters of governance of a highly social cultural environment.'' (2010 p. 669) That is to say that a state vs. market model choice creates a false binary for governance in the cultural sphere, and further, these two approaches are incompatible with innovation required of collective action institutions.

Their response is to adapt and extend the Institutional Analysis and Development (IAD) framework as it has been used in environmental studies. The result, a ``knowledge commons framework'', provides a method to empirically challenge the accepted wisdom that intellectual property rights and state governance provide the best options for sustaining a knowledge commons. The following case studies are focused exclusively on knowledge commons in scientific research and development settings.

% Please add the following required packages to your document preamble:
% \usepackage{graphicx}
\begin{table}[h]
\resizebox{\textwidth}{!}{%
\begin{tabular}{p{4cm}|p{4cm}|p{4cm}|p{4cm}|p{4cm}|}
\cline{2-5}
 & \textbf{Genome Commons} & \textbf{Urea Cycle Net.} & \textbf{Galaxy Zoo} & \textbf{ICOADS} \\ \hline
\multicolumn{1}{|l|}{Domain} & Biology, Life Sciences & BioMedical, Public Health & Astronomy & Climate Science \\ \hline
\multicolumn{1}{|l|}{Year Established} & 1996* & 2003 & 2007 & 1983 \\ \hline
\multicolumn{1}{|l|}{Funding Mechanism} & Mixed Portfolio & NIH & Sloan, Oxford University, Johns Hopkins University & NOAA, NSF, National OcenaographiCenter (NOC - UK) \\ \hline
\multicolumn{1}{|l|}{Method of Data Collection (for KCF study)} & Document, Policy Analysis & Document, Policy Analysis, Survey, Interview, Observation & Document, Policy Analysis, Interview & Document, Policy Analysis, Interview, Observation, Participation, Informetric \\ \hline
\multicolumn{1}{|l|}{Commons Resource Set(s)} & Data, Databases & Data, Patients, Reagents, Literature, Infrastructure & Data, Software Platforms & Data, Metadata, Software, Stat. Techniques \\ \hline
\end{tabular}
}
\caption{}
\label{my-label}
\end{table}

\subsection{The Genome Commons}

Contreras contribution to the knowledge commons literature is a rich and detailed history of what he calls the genome commons (2014); databases and digital archives of genomic data (DNA nucleotide base pairs) that have emerged thanks to a lowered price of DNA sequencing technologies, and large-scale international collaborative efforts to ``map'' entire species genomes. Through policy analysis, Contreras demonstrates, ``the evolution of the genome commons from what was initially a public domain vehicle established to deter the proprietary designs of emerging biotechnology companies, into a unique polycentric governance institution for the growth, management, and stewardship of a massively shared public resource.'' (2014, p. 102) 

Contreras draws our attention to early controversies over the patenting of databases, and the private sector's attempts to compete with large public sector projects like the Human-Genome Project (HGP) (Collins et al., 1998). This conflict, in part, leads to the generation of thousands of pages of policy written about the rules that data producers must follow, how archives are to guarantee access for potential users, and what scientific societies should do to create institutions for provisioning these resource sets over the long-term.  The major contribution of Contreras work is twofold:

\begin{enumerate}
\item The genome commons demonstrate how, in an international setting with fierce competition for semi-rivalrous resources, a polycentric model of governance can emerge to nest institutions within a highly functional framework of cooperation. This is due largely to what Contreras identifies as the Mertonian norms upon which science, its actors, and the design of its institutions are based. 
\item In practice, the genome commons are effective because of policy instruments like ``embargo periods'' for archived data which grant ``right of first access'' to genomic data contributors. In previous works, Contreras describes the need for a ``latency analysis'' to better understand how the rapid dissemination of research results are balanced with limited exclusive rights to access (Contreras 2010a, 2010b). These dimensions, he suggests, are critical to the design of future policy governing knowledge commons. 
\end{enumerate}

\subsection{Galaxy Zoo, and the Astronomy Commons}

Madisson uses the knowledge commons framework to explore the intersection intellectual property rights, crowdsourcing, and ``big data.'' (2014) He pursues a comparative project asking ``How do the outcomes produced by commons governance differ from outcomes that might have been available if alternative governance had been employed?'' (2014 p. 231) To do so, Madisson's main contribution is a case study of Galaxy Zoo and its model of sharing credit amongst a large number of crowd workers, and astronomers. He uses findings from this work to compare the institutional arrangements of Galaxy Zoo with the Nearby Supernova Factory project, whose organizational structure resembles that of a traditional firm. 

Like Contreras, Madisson suggests that the Galaxy Zoo project succeeds in part because of a broad appeal to the Mertonian norms of science. Madisson also claims that this knowledge commons exemplifies Hagstrom's claim that science is a ``gift culture''  (1968) trading in ideas instead of material gains. This is a claim supported by an observation that the project's designers, exemplified by Chris Lintott, forefitted personal claims to data transcribed by crowdworkers, as well as the Galaxy Zoo's commitment to the credit and acknowledgment of these contributions in resulting journal publications (2014, p. 235). 

Madisson makes an important point that the openness of a resource set, both those produced and provisioned by an entity like Galaxy Zoo, is not a binary distinction; it is instead a continuum. The data, and underlying software that produces Galaxy Zoo is free, and openly accessible, but in the case of the data, it was only released to the public after an initial round of project specific publications were in press\footnote{ Describing how these kinds of distinctions in ``open'' and ``free'' access effect long-term sustainability is described a major open question for knowledge commons scholars.} (2014, p. 215).

Another major conclusion from this work is that ``...institutional order and knowledge governance, such as commons, are mutually constitutive. One largely exists because of the other.'' (Madisson, 2014 p. 211) Madisson argues that the emergence of crowdsourcing, peer production, and big data management in science are creating new orders of power, and democratizing access for many different (and new) knowledge commons stakeholders. In turn, the social dilemmas arising from broad collective action are poorly understood and deserve continued attention of those concerned with knowledge commons in science and technology settings. 

\subsection{Urea Cycle Disorder Research Network Commons}

Strandbrug, Cui and Frischmann apply the knowledge commons framework to a rare disease research context, which consists of patients, caretakers, biomedical researchers and life scientists, as well as public policy experts across a number of cooperating institutions in the USA (2014). As they note, the need to share resources such as patient data, outreach materials, study protocols, reagents, diagnosis, and even patients themselves creates the need for a commons approach to open exchange and resource pooling. This however, does not mean the research coordinating networks that participate in a commons are absent of rivalry; there is intense competition for funding from the NIH, for locating and managing patients to participate in a cohort study, and for longitudinal data collection that one institution may bear the majority of costs to produce and provision. 

Although embedded in the a biomedical field which often has strong private sector interests, the authors note that there is a strong incentive to openly share resources in this domain,``'Open' approaches are particularly attractive in the rare disease context, given the small numbers and geographical dispersion of potential research subjects and the inapplicability of the “blockbuster drug” business model.'' (2014, p. 155) Pharmaceutical companies may have little financial incentive to produce drugs for rare diseases \emph{because} they are rare and represent a relatively small market. As a result, rare disease research and treatment comes with the need to conduct clinical trials, rapidly share results, and work closely with private industry to produce effective drugs. The rare disease research context therefore nests clinical, academic, pharmaceutical / private industry, as well as the NIH and a number of other non-profit organizations. 

\section{Research Questions}

This dissertation has pursued answers to two broad research questions:\\

\emph{What are the effective institutional arrangements (governance) for sustainable scientific knowledge commons? And, how do these arrangements differ between domains of knowledge production?}\\

In the following section, I compare and contrast the previously described case studies based on the contexts in which they produce and provision a knowledge commons. This analysis provides answers to how knowledge commons differ between domains of knowledge production. In the section following this one, I characterize the effective governance of sustainable commons across these different domains using coded state variables.\\  

The following table summarizes the characteristics of knowledge commons that will be compared.\\  

p{3cm}

% Please add the following required packages to your document preamble:
% \usepackage{graphicx}
\begin{table}[h]
\centering
\resizebox{\textwidth}{!}{%
\begin{tabular}{p{4cm}|p{4cm}|p{4cm}|p{4cm}|p{4cm}|}
\cline{2-5}
                                                                 & \textbf{Genome Commons} & \textbf{Urea Cycle Disorder Consortium} & \textbf{Galaxy Zoo} & \textbf{ICOADS} \\ \hline
\multicolumn{1}{|l|}{Potential for Commercialization}            & High                    & Med / High                              & Low                 & Low       \\ \hline
\multicolumn{1}{|l|}{Resource Type (s)}                          & Common Pool Resource    & Common-Pool Resource / Public Goods     & Public Good         & Public Good     \\ \hline
\multicolumn{1}{|l|}{Governance Regime}                          & Polycentric             & Self-organized                          & Self-organized      & Polycentric     \\ \hline
\multicolumn{1}{|l|}{Discount Rate}                              & High                    & Low                                     & High                 & Low             \\ \hline
\multicolumn{1}{|l|}{Model of Peer Production (Haythornthwaite)} & Heavyweight             & Heavyweight                             & Lightweight         & Heavyweight     \\ \hline
\multicolumn{1}{|l|}{Task Uncertainty (Whitley)}                 & Low                     & Low                                     & Low                 & Low             \\ \hline
\multicolumn{1}{|l|}{Mutual Dependence (Whitley)}                & High                    & High                                    & High                & High            \\ \hline
\multicolumn{1}{|l|}{Latency Requirements}                       & High                    & Medium                                  & Low              & Low             \\ \hline
\end{tabular}
}
\caption{Comparing knowledge production and provisioning in different knowledge commons}
\label{my-label}
\end{table}

\textbf{Mutual Dependence and Task Uncertainty}

One way to initially compare across these different knowledge commons is to ask what the background environment supplies in the way of a ``default'' assumption about the production and sharing of resources. Since each case study is embedded in a scientific domain of knowledge production, the discipline or field specific to this context is likely to have some influence on these practices. Whitley's work on the differences of knowledge production practices in a`` field'' of science provides a helpful framework for understanding these issues. In this work, Whitley has described ``the nature'' of intellectual fields through variations in two dimensions: mutual dependence and task uncertainty (2000).

\begin{enumerate}
\item \emph{Mutual Dependence} describes the degree to which one field relies upon another for validation of findings or for inputs to new work. Fry and Talja point out mutual dependence also accounts for  ``...the extent to which a field adopts evaluation criteria and standards from other fields for the assessment of work externally produced, rather than developing its own criteria.'' (2007, p. 118). 

\item \emph{Task Uncertainty} is the degree to which members of a field understand and articulate problems to be solved, and agree on methods or techniques needed to solve those problems. For instance, high task uncertainty means there is little coordination, results are based on loosely constructed criteria for acceptance or standards for significance.
\end{enumerate}

For a commons to form in the first place there should be a low degree of task uncertainty. Producing a shared resource, in some ways, is agreeing upon the validity or meaning of that resource. And, it should be expected that each domain invested in commons-based work has a high degree of mutual dependence, as commons are often formed for the express purpose of collective action. 

All four case studies exhibit high mutual dependence and low task uncertainty. This classification, which is the only agreement across all four cases, is helpful because it makes clear the initial conditions that a field should have in order to most effectively use a commons arrangement\footnote{A small qualification: Other combinations may result in an effective commons, but these are not as readily obvious as the high mutual dependence and low task uncertainty combination}.\\ 

\textbf{Peer Production}\\
Each of these case studies demonstrates that the division of labor needed to reliably sustain a knowledge commons depends, in part, on peer production models. Increasingly, peering occurs in not just the production of new resources, but also in the provisioning of shared infrastructures, and the preservation of existing resource sets. To differentiate between models of peer production, I'll use Haythorthwaite's notion of ``heavyweight'' and ``lightweight'' peer production described in Chapter 2: The Genome Commons, The Urea Cycle Disorder Network, and ICOADS all use a form of heavyweight peer production, where strongly connected and highly committed experts make contributions to a common pool of resources which require peer review before acceptance. Galaxy Zoo on the other hand uses a crowdsourced form of data annotation which resembles the lightweight peer production model; a large number of weakly connected novices perform simple tasks that are aggregated and quality controlled in producing complex resources. 

These characteristics of peer production effect a number of other ways in which the resources and governance of a knowledge commons can be characterized; including, the resource type, the governance model that results, the latency requirements for protecting producers / innovators, and the discount rate that users of a commons adopt.\\ 

\textbf{Resource Types, Commercialization, and Latency Analysis}\\
Resources produced by these knowledge commons are largely the results of scientific research processes, and as such they have some degree of rivalry. This means that categorizing the type of good being pooled will be inexact; each of these commons has a shared resource system made of goods that fit somewhere between common-pool resources and public goods. Only the Genome and Urea Cycle disorder commons have the realistic capability of being commercially exploited, and this leads to their having a need for latency in the release of these data resources to a broader community of users. In Galaxy Zoo, and ICOADS\footnote{This was observed in major data contributors receiving early access to new releases} limited monopolies are not likely to be institutionalized, although both have at times informally delayed immediate data release for this purpose. In the future, this may require Galaxy Zoo and ICOADS to adopt new rules, or design new approaches to data access.\\     

\textbf{Discount Rate}\\
In Chapter 2, I described characteristics of successful commons in the socioecological realm. Ostrom (1999, 2000) argued that one of the most important factors in sustaining a commons is those who are most dependent on the resource have a ``low discount rate'' in approaching the common resource; in other words, they have a ``willingness to sacrifice current payoffs for higher payoffs in the future.'' (Acheson, 2011) In the knowledge commons, discount rates can be observed in the willingness of individuals to cooperate, contribute, or responsibly consume resources so that they exist for future generations to use.

Contreras discussion of the Genome commons shows that a high potential for short-term profit - either through commercialization or through credit of discovery - creates the need for formal strict regulatory policies and normative enforcement of rules in the genome commons. Another dimension of discounting important to the genome commons is related to the rapid technological advancement of the field; differences between first, second, and third generation sequencing data quality means that these datasets decay in value rapidly. This is the direct opposite of ICOADS, where historical weather records mature in value over time. Actors within these two domains have different approaches to valuing the future of a resource set: Genome commons actors will have a high discount rate, while ICOADS actors will have a low discount rate.

The Urea Cycle Disorder network curiously mixes high and low discount rates; the limited nature of its funding (5-year NIH renewable grants) requires the rapid production of research results, but also requires a sustained effort to keeping the network effectively functioning for future rounds of funding. This tension leaves actors sometimes discounting a future which might not exist, and others carefully preserving resources for long-term use (see Strandburg, Cui, and Frischmann 2014, p. 165 for further discussion).  

The Galaxy Zoo, and crowdsourcing in general, coordinate crowd workers with high discount rates - not only will they likely not be involved in the future of the research project, but the strength of ties between actors are very low. This is one of the steep costs of using a crowdsourcing approach to data transcription, and is likely to be important in address in sustainability planning. Astronomers involved in the project often have a low discount rate, as the data sets resulting from this work are highly valued by the research community and likely to be inputs to many future research projects.\\

\textbf{Governance Model}\\
Given the characterizations described above, which models of governance do these knowledge commons actually use? The Genome Commons and ICOADS are both nested within a number of different institutions of decision making power and authority. This is partially due to the fact that they are both international in scope, have existed for a long period of time, and receive funding or infrastructure support from a number of different institutions. They can both be seen as adopting a polycentric governance model, with nested levels of authority and rule enforcement.  

The Urea Cycle Disorder network most closely resembles a self-organized governance model; it receives only partial direction from the NIH for governing its shared resources, and all institutions directly involved in the network have equal authority in establishing and enforcing rules. Similarly, Galaxy Zoo receives partial direction from its project partners at Oxford University and the Sloan foundation as a funding agent, but by and large the project remains self-organized; reporting informally to a board of directors assembled to vote and establish by-laws. All project partners  having a relatively equal role in shaping future directions of the project, but the overwhelming power of authority is held by a few individuals.

\section{Governance Variables}

In this section I further characterize the effective governance of sustainable knowledge commons using a set of state variables developed in the case study of ICOADS. These states are summarized in the table below, but my analysis focuses on only a subset of these state variables.\

% Please add the following required packages to your document preamble:
% \usepackage{graphicx}
\begin{table}[h]
\centering
\resizebox{\textwidth}{!}{%
\begin{tabular}{p{4cm}|p{4cm}|p{4cm}|p{4cm}|p{4cm}|}
\cline{2-5}
 & Genome Commons & Urea Cycle Dis. & Galaxy Zoo & ICOADS \\ \hline
\multicolumn{1}{|l|}{Multiple levels} & Coordination among multiple levels & Coordination among multiplelevels & Single-level governance & Coordination among multiple levels \\ \hline
\multicolumn{1}{|l|}{Institutional diversity} & High & High & Low & Medium \\ \hline
\multicolumn{1}{|l|}{Centralization} & 2- Somewhat Decentralized & 3-somewhat centralized & 4- Highly Centralized & 2-somewhat decentralized \\ \hline
\multicolumn{1}{|l|}{Governance scale} & International & Sub-national & State & International \\ \hline
\multicolumn{1}{|l|}{Scale match} & Match & No-match & No Match & Match \\ \hline
\multicolumn{1}{|l|}{Metric diversity} & Medium & Medium & High & Low \\ \hline
\multicolumn{1}{|l|}{Rules-In-Use} & Information, Aggregation, Scope, Payoff & Boundary, Information, Authority, Aggregation, Scope, Payoff & Information, Payoffs & Boundary, Position, Information, Authority, Aggregation, Scope, Payoff \\ \hline
\end{tabular}
}
\caption{}
\label{my-label}
\end{table}

\subsection{Centralization}

The degree of centralization in any governance system impacts how its decisions are made, how outcomes are achieved, and how the commons is managed. Both ICOADS and the genome commons are decentralized (e.g. their administration and points of service are distributed across numerous institutions) including rule-making structures which extend to a number of national and international scientific agencies. The Urea Cycle Disorder network is somewhat centralized, as its decision making focuses largely on PI's named in its founding NIH grant. Similarly the Galaxy Zoo is highly centralized, with governance located almost exclusively within its founding institution, and small board of directors. 

\subsection{Governance Scale and Scale Match}
ICOADS and the genome commons are in a state of governance match, where the institutions coordinated, and the resources produced or provisioned are international in scope as is the governance scale. The Urea Cycle Disorder network governance is exclusively national in its governance model, but the patients they manage and the resources they produce are global in scope. Similarly, Galaxy Zoo has a state-level governance model, but coordinates the work of a large international crowd, whose interests and use of data includes experts and non-experts. Both Galaxy Zoo, and the Urea Cycle Disorder network governance do not have a scale match, which may impact their growth and stability. However, it is worth noting, that early in ICOADS tenure, it too lacked a scale match. The Urea Cycle Disorder network and the Galaxy Zoo are the two youngest knowledge commons being analyzed; a mismatch in scale and governance may simply be an artifact of the evolution of governance rather than a serious fault of either commons. 

\subsection{Metric Diversity}
Galaxy Zoo is the only knowledge commons with a high diversity of metrics. This is partially a result of the broad success of the project, which has been able to redefine it's goals and mission a number of times as the platform for crowdsourcing rapidly evolved. As a knowledge commons, Galaxy Zoo has also proved to be capable of hosting and designing novel transcription projects outside of astronomy. Thus, the resource set expanded from data, to software, to even the community of crowd-workers themselves that migrate from project to project. Each of these shifts has allowed Galaxy Zoo to set new goals, and consequently describe the success of those project's on terms that are appealing to potential funding agencies. A non-trivial spillover effect, is that this diversity of metrics allows Galaxy Zoo's spokesperson to quote convincing statistics of success throughout his many speaking engagements around the world - adding substantively to the reputation of the project (Madisson, 2014 p. 255)

ICOADS, the genome commons, and the Urea Cycle Disorder network all have medium to low diversity of metrics. A strong tie to funding agencies at national scales, and the academic nature of these research communities locks them into bibliographic and informetric (usage-based analytics) evaluation tools which are, as described throughout the ICOADS case study, limited indicators of a commons' success. 

\subsection{Rules-in-Use}

All four knowledge commons share information and payoff rules in common. At a broad level, these are likely the most essential rules for a successful knowledge commons to adopt, as they specify not only what information must be shared, and with whom, but also how resources are to be distributed amongst stakeholders. In this sense, entitlements (either through production or provisioning) are a prerequisite for engaging in a commons, both at an institutional and individual level.

The Urea Cycle Disorder network, the genome commons, and ICOADS also share aggregation and scope rules in common. These may be rule types of a mature commons, and may be an artifact of those commons having faced controversies in the past, such as the genome commons enclosure, or the defunding of ICOADS, which exert pressures related to governing outcomes. 

ICOADS is the oldest institution, and is also the only knowledge commons that, in its current state, has some version of all seven rules as a part of its governance structure. As the case study demonstrated, these rules have accreted over time; each new regime added a new set of rules, while retaining some or all of the functions of the previous rule set.

\subsection{Summary of Answer to Research Questions}

Based on a cross-case comparison, effective governance models for sustainable scientific knowledge commons include polycentric and self-governance models. These two methods of governance allow for what commons scholars have long recognized as a key to effectively functioning shared resources systems; they allow ``...a group of principals to organize themselves voluntarily to retain the residuals of their own efforts.'' (Ostrom, 1990) In ICOADS and the genome commons, we see that as the projects matured they became embedded in networks of rule making authorities, effectively nesting institutions in overarching policies that provide directions for particular actions, while still allowing individual actors and organizations to ``retain the residuals of their own efforts.''

The Urea Cycle Disorder network is a complex and valuable study of the limit of commons governance at a national scale. Quite obviously, the disease is international, so how this group effectively cooperates in a local context, and maximizes their ability to self-govern is an important case study for the NIH, as it approaches the design of a ``big data'' commons. And yet, the centralization of its rule-making, and the lack of scale match presents potential dilemmas for future sustainability efforts. It should be noted that this particular network had its NIH funding renewed in late 2014, and will effectively be operating until 2020, providing the opportunity for a longitudinal case study of self-governing, centralized commons governance. 

The Galaxy Zoo case study is unique to science commons, but provides a potentially valuable link to other lightweight peer production platforms governed as commons. The relatively centralized governance, which also lacks a scale match between its resources and rule-making, is nevertheless the most visible and diverse of the four case studies. Its has effectively ported its transcription model to annotation in a number of different fields, including the climate data from old log-books which are feeding into ICOADS Release 3.

What the diversity of case studies here indicate, is that commons governance is mutli-faceted. It requires a match between the resources being shared, the management structure of the provisioning institutions, and the production model of the resource contributors. And yet, as the Urea Cycle Disorder network and Galaxy Zoo point out, the effective combination of rules in use, and a diversity of metrics communicating success can overcome these initial criteria. This is especially important to Galaxy Zoo, as the discount rate of its potential contributors are quite high. 

Traditional notions of ``field'' specific knowledge production, although not exhaustive, remain helpful to the analysis of knowledge production in commons-based institutional settings. Whitley's mutual dependence and task uncertainty works as kind precursor to success - without which it is unlikely that a commons could easily succeed. But, the case studies described above also demonstrate that there are a number of important variables to consider besides simply the default background assumptions of an intellectual field. The use of networked information communication technologies increase the ease at which collaboration can take place across space and time; this, coupled with a the decrease in available funding for basic science research creates an environment of necessary innovation in science and technology research. The knowledge commons frameworks described here points to important state variables, but there are a number of other dimensions, each of which may be equally as important as governance, that have not been addressed in this study. In the following chapter, I summarize these limitations and point to future directions in this research. 

\section{Summary}
In this chapter I have summarized the research findings from a case study of ICOADS governance, and answered three research questions related to the effectiveness of ICOADS' governance. I then compared these findings with three previously completed case studies, and answered the overall research questions of this dissertation. In doing so, I demonstrated how a systematic approach to coding state variables could be adapted from socioecological to sociotechnical settings. In the final chapter, I describe limitations of this work, and the implications of this research for policy, practice, and theory of the commons. I conclude with future directions for this work. 