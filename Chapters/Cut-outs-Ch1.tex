In science and technology policy the two approaches are often referred to as a tension between the commercialization of science on the one hand, and the so called ``norms''of science on the other (Merton, 1963; Eisenberg 1989; Rai 1999; Reichman & Uhlir 2003; Mirowski, 2012; Madison, 2014). 

Both the the market and the state model are costly in a contemporary research and development environment that requires broad collaboration: Markets tend to create wealth by providing limited protection (e.g. patents) to innovators, but these barriers to access often impede the types of transformative breakthroughs that are necessary for combating global scale issues, such as the sharing of data around an Ebola outbreak in Western Africa (Lofgren et al., 2014); Governments tend to create equitable access and opportunity for innovation, but rarely do scientific funding agnecies have the capacity to sustain investments in research projects longer than a five year grant cycle; leaving innovation unrealized (Mirowksi, 2011). 

\subsection{The Science Commons}

Various forms of commons are in use or currently being explored within organizations invested in scientific knowledge production. These include small community resources such as the Weather Research and Forecasting model (WRF) model in meteorology, all the way up to national level funding strategies found in the European Commission's 8th Framework Program for Research and Technological Development. The National Institutes of Health (NIH) Big Data 2 Knowledge (BD2K) Program has initially described its mission as commons-based:

\begin{displayquote}
``The Commons is above all else a conceptual framework for a digital environment to allow efficient storage, manipulation, and sharing of research objects...The Commons belongs to and affects the whole research community... the concept relates to the entire global biomedical research enterprise, the NIH does not own it, so it is not the NIH Commons; similarly it is not just for data and hence is not the Data Commons. Rather it is the concept of sharing digital research objects from any domain, where sharing implies finding, using, reusing and attributing. The Commons could be considered analogous to the Internet – each user has his/her own definition of exactly what the Internet is, but all are able to use it every day for their own purposes. No one seems to own it yet it works because each participant abides by a simple set of agreed-upon rules.'' (NIH, 2014)
\end{displayquote}

While commons governance \emph{can} be just a simple set of rules they can also be complex, multi-scale and multi-level institutional arrangements. Just as in socioecological systems, sociotechnical systems that enable scientific knowledge production are likely require different kinds of governance systems in different contexts. Sustaining the different sets of resources systems needed to address grand-challenge science issues like climate change, the spread of infectious diseases, or biodiversity loss will require diverse institutions cooperating effectively in different arrangements. In short, an understanding of the types institutions that most effectively enable cooperative scientific work will require a systematic, and sustained research agenda as much as an individual study.

--

Successful examples of these cooperative institutions can be seen in the open-source software that powers high-performance computing facilities (),open access libraries of journal publications in the life sciences (), and archives of freely available genomic data (). 


Increases in collaborative aspects of science can be seen in growing number of authorship lists in peer-reviewed articles (Cronin, 2005), organizational diversity in research funding applications (Geuna and Martin, 2003), and the general organization of research institutes that combine expertise from two or more disciplines (Etzkowitz & Leydesdorff, 2010). Cooperative aspects of contemporary research and development can be seen in the collective action required for developing shared infrastructures (Olson, Zimmerman, and Bos, 2007; Edwards et al, 2007), and the sharing (e.g. Contreras, 2010; 2014; Strandburg, Cui and Frischmann, 2014).
