\chapter{} 

\section{Introduction}

\emph{In this chapter I restate the limitations of this work, and describe the implications of this dissertation's findings for policy, practice, and theory of the commons. I conclude with future directions for studies of sociotechnical systems sustainability.}\\

\section{Conclusions}\\

\subsection{Effective Governance for Knowledge Commons}

This dissertation began by noting the differences between scientific knowledge, and the objects that result from knowledge production: On the one hand, scientific knowledge is assumed to be a public good, with low rivalry and costly excludability; On the other hand, the products of scientific research (data, software, journal publications, etc.) are assumed to have rivalry for access, which in turn requires state intervention, or privatization for managing sustainable access and use.   

The analysis found in Chapters 4 and 5 has demonstrated the efficacy of a hybrid model, the commons, used by four domains of knowledge production in research and development settings. \emph{Under the right circumstances}, the commons can be an effective model for balancing interests between private and public sectors (e.g. genome commons, and Urea Cycle Disorder network), creating long-enduring institutions for producing and provisioning valued databases (e.g. ICOADS, and genome commons), and sustaining order and cooperation between large numbers of individuals competing for access to resources (e.g. observations made across all four cases).

The case studies reviewed in Chapter 5 are an initial step in providing empirical findings about those \emph{right circumstances}; including, how knowledge commons efficiently operate, how they evolve and change over time, and the ways that governance of sociotechnical systems differ in sustaining shared resource sets.

This research indicates sustainable knowledge commons in science settings have the ability to: 
\begin{itemize}
\item Match scales between governance regimes, and resource sets. 
\item Find ways to diversify institutional arrangements as resource sets and stakeholders grow in scale.
\item Articulate selection pressures, and diagnose underlying drivers and symptoms of those pressures.
\item Draw upon social and technical capital, and coordinate relevant stakeholders in response to selection pressures.
\item Define, and optimize activities for a diversity of metrics. 
\item Balance policy instruments that are regulatory, and incentive-based.
\item Adapt quickly to sudden disturbances, and draw on social and technical capital when disturbances are slow and continuous.
\item Seek out, and absorb ``regnerative disturbances'', and consequently build a capacity to withstand ``disruptive disturbances.''
\item Accrue and revise rules-in-use. Information and Payoff rules were recognized as an essential baseline for succesful knowledge commons across all four cases.
\end{itemize}

Some of these results are self-evident, but without empirical studies to clarify precursory assumptions, sociotechnical systems will, like socioecological systems, fall prey to oversimplifications of what works when, where, and for whom. This work also helps to underst understand which dimmmensions of sustainability should be explored in more detail, such as how the design of effective policy can encourage a match in governance scales;  or, how institutions might be encouraged to diversify institutional arrangements and metrics through incenstives. This is to say that, the type of research presented here may be preliminary, but it is also foundational: it picks out systems features, names them, and puts standardized categories on them so that they can be compared with other similar systems in order to build reliable knowledge.

In the next section, I extend some of these observations to the polycentric system of governance.  

\subsection{Polycentric Models of Governance}
Research and development activities in the sciences can be divided into many axis of categorization: basic vs. applied, theoretical vs. practical, inductive vs. deductive, etc. In contemporary research settings, a distinction that cuts across many of categorizations is that of collective action; research and researchers increasingly collaborate, share resource sets, and divide up labor in pursuit of grand challenge science problems. (Atkins, 2003) Even mathematics has turned to peer production to solve difficult theorems. (Ball, 2014) 

Governing these new arrangements for sustainability will be an important, but difficult challenge for their stakeholders. The case study of ICOADS, the genome commons, and the Urea Cycle Disorder network has demonstrated that as a project matures, it comes with a need to formalize certain governance issues through rule making procedures of self-organized groups. In the case of the Urea Cycle Disorder network, matching the scale of governance with the resource sets provisioned has proven difficult for an international phenomena that is funded exclusively as a USA based research project. The genome commons and ICOADS both present cases where international cooperation is required for both the provisioning, and the production of resource sets. Over time these two knowledge commons have found ways to match those scales, although each initially struggled to do so.  

The polycentric governance model observed in ICOADS and the genome commons developed as a result of growth and evolution, rather than purposeful design. The regime shifts that mark ICOADS changing from a self-governance model to a polycentric model was in fact response to selective pressures - such as a loss of funding from a major sponsor and the need to formally embed itself in a number of different bureaucratic structures. 

As these polycentric models themselves mature, an important next step is further investigating their features; What types of nested levels can be observed across polycentric models? When do transitions between self-governance and polycentric models occur? What are the negative externalities (the unintended consequences) of having an interdependent governance system like a polycentric system? How, and in what ways do the the negative (or positive) externalities of a polycentric system differ from that of Federalism? \\

\textbf{Limitations of the Polycentric Model}\\

While I have focused largely on the positive aspects of polycentric systems having multiple levels of jurisdiction, this also creates multiple sources of friction, and redundancy. This is especially evident in science and technology where a mix of self-governance and state-based hierarchical systems create policies which are too broad to be useful, or too specific to be followed with intended fidelity. This is the current case with data management planning, and data sharing initiatives promulgated by federal research funding agencies. Borgman characterizes the ``conundrum of data sharing'' this way:
\begin{quote}
``An investigator may be part of multiple, overlapping communities of interest, each of which may have different notions of what are data and different data practices. The boundaries of communities of interest are neither clear nor stable. In the case of data management plans, an investigator is asked to identify the appropriate community for the purposes of a specific grant proposal and for the proposed duration of that award.'' (2011)
\end{quote}

This characterization sounds much like a polycentric system of governance, but one in which there is a mismatch between resource sets, actors, and rule making procedures. Friction and confusion result from funding agency mandates aimed at governing data management at a project level, and practices around sharing data for reuse at a field or discipline level. The success of broad programs like a ``data management'' initiative will require a match between scales, rules in form and rules in use, as well as the exploration of polycentric models of governance. Otherwise the provisioning of important resource sets will continue to be ``conundrums'' rather than effective public policies for increasing the impact of science and technology on society. 

\section{Limitations}

Research in a sociotechnical setting, such as a knowledge commons, will inevitably suffer from a limited view of what are complex systems, and this research project is no exception. This work has been limited by a number of factors, including:
\begin{itemize}
\item The small sample size (n=4) of this work limits its generalizabilty. Additionally, the partial description of what are large and complex international institutions is limited by my own knowledge of these systems. 
\item The frameworks applied here are relatively new, and require a broader set of case studies to test their efficacy for generating reliable, comparable results. 
\item A majority of the analysis in this document has focused on the governance of knowledge commons. Other components briefly described here, such as actor groups or the characteristics of a resource set, may be equally important to sustainability. 
\item The approach used to systematically coding state variables is also relatively untested in sociotechnical systems research. With a small sample size, the limits of this approach are not well understood. 
\item Sociotechnical systems include many more context than research and development in science, and this may well effect the applicability of this approach to new contexts.
\end{itemize}

\section{Future Work}

Future work in the study of sociotechnical systems requires a number of important modifications to this intial study, which I describe in detail below. I focus in particular on two open research questions from the comparative case study found in Chapter 5. 

\begin{itemize}
\item Both Contreras and Madison relied on an account of Mertonian norms, otherwise known as CUDOS - Communalism, Universalism, Disinterestedness, and Organize Skepticism (1973) - to explain their case study subjects being committed to openness and sharing. Although the concept of ``communalism'' seems a good fit for this context, Merton's norms in general are an awkward framework for a research agenda like the knowledge commons. Merton's normative theory is grounded in functionalism - he sees the actions, behaviors and adopted rules of a culture through latent and manifest functions that are, above all else, constructed for the sake of preserving a culture (1963). Functions explain how cultures organize and act, in order to remain the same (how they continue to \emph{function}), not how and why cultures evolve. This seems to conflict with the major goal of the knowledge commons research agenda, which is to provide an alternative to the functionalist accounts of institutions, especially those that rely on intellectual property rights protections. (see, Madisson, Frischmann and Strandburg, 2010 p. 665-78) 

And yet, in the ICOADS case study I too described a ``default'' moral framework that my participants referenced in describing the historical precedence of their work, especially as it related to Lt. Matthew Maury. My findings also seem to conflict with the fact that, in practice, these systems resembled flexible evolution, and adaptation, rather than fierce guarding against change. An open research question then is which theories (political, sociological, or otherwise) can help explain the continuity of this normative dimension of communalism given the dynamic evolutionary account that occurs throughout this dissertation. To put it more simply, collective action requires dynamic change and readjustment for sustainability, so how is it that there remains a normative center upon which geneticists, rare disease researchers, astronomers, and climatologists all seem to adhere? This cannot be the case. Careful attention to future differences between domains of knowledge production, their norms, and their practices is needed within the knowledge commons framework. 

\item The institutional level case studies presented here, with the exception of the Galaxy Zoo, runs somewhat counter-intuitive to previously published literature about the culture of science being ``gift giving ''(Hagstrom, 1965) and collaboration having a propensity that is subject to bounded-rationally (Birnholtz, 2005), or a matter of individuals using a limited monopoly for rent extraction (Birnholtz and Bietz, 2003; Harvey, 2009). Traditional notions of rent are effective only when the costs and benefits of collaboration are symmetrically distributed amongst stakeholders. Collective action dilemmas, which emerge when resources are pooled to accomplish basic tasks, such as calculating historical weather trends or sequencing a species genome, pose a different set of problems than property-based concepts like ``rent'' or bounded-rationality can provide explanations for. And this is becuase institutions for collective action suffer from social dilemmas where costs are \emph{not} symmetrically distributed. This is not to say that behavioral theories are inapplicable to the commons context; quite the opposite - there is a need to understand how those traditional notions of limited monopoly (rent) effect commons when nested in polycentric governance structures. Contreras offers ``latency analysis'' but this describes only one temporal aspect of the social dilemmas of sharing resources and receiving credit.

\item The work presented here also suggests that beyond science commons, the sustainability of sociotechnical systems should be treated as a collective action dilemma - the groundwork for this approach in design based fields was laid by Paul Dourish (2010) in HCI. As he points out, design interventions for environmental sustainability are often focused on individual choice, which has (intentionally or not) ``transformed the problems of sustainability into the cost benefit trade-offs of rational actor economics'' (2010, p. 2). Again, abstracting from the environmental domain to the sociotechnical requires a number of modifications to work in HCI that focuses on ICTs for (environmental) sustainability. Developing new commons theory to explain institutional successes and failures, including case studies using the Knowledge Commons Framework, is an important first step.

\end{itemize}

To continue advancing a robust empirical account of sustainability in sociotechnical systems, I suggest that further work is needed on resilience and adaptive cycles. 

\subsection{Resilience, Vulnerability \& Adaptive Cycles}

The work presented here describes how a knowledge commons effectively transitions between different governance regimes over time. It suggests a need to turn away from functional integrity explanations of sustainability, and instead adopt a framework of analysis that is more congruent with studying knowledge production, and the governance of knowledge products through systems concepts like adaptivity and resilience. Future work can and should go in a number of directions. I will summarize two below:

\begin{itemize}
\item Adaptive cycles focus on two phases or transitions between system states: Phase one, the foreloop, is characterized by growth and accumulation, and phase two, the backloop, is a rapid response for reorganization and renewal (Holling, 1973). 

\begin{figure} [h]
\includegraphics[width=4in, height=2.5in]{ad-cycle}\\
\captionbelow{Holling's adaptive cycle (1973), with four state changes and tranistions (loops) between different states}
\end{figure}

If a system doesn't develop a capacity to complete a backloop (from Omega to Alpha), it fails to renew and will collapse, hence it is not resilient to change and is unsustainable.  The sociotechnical systems described here all had the ability to complete a backloop, although with varying degrees of success, and over different (longer and shorter) periods of reorganization and renewal. As Holling pointed out in socioecological systems, a path towards ecosystem collapse begins by extending foreloops (2007). Often times this means that capital is spent treating the symptoms of a problem, delaying the initial regeneration and increasing the  likelihood that a system won't continue to respond quickly or efficiently to external pressures. 

\item To move towards a model of sociotechnical systems resilience and adaptivity, future work might begin by  characterizing components of change, such as those from Holling's model, and the transitions between those components. In the state variable coding developed in this dissertation, I partially addressed adaptivity through the ``diagnostic'' variable, where either the symptoms or the underlying drivers of a disturbance are treated. Future work on resiliency and adaptive cycles for sociotechnical systems should also investigate this closer - trying to both characterize and better understand responses to selective pressures, disturbances, etc.  

Geels and Schot have developed a concept they call ``pathways to transition between sociotechnical regimes.'' (2007) Like foreloops and backloops in the above socioecological model, the transitions approach conceives of regime change through four sustainability pathways called transformation, reconfiguration, technological substitution, and de-alignment & re-alignment (Geels and Schot, 2007). These do not conceive of state changes as cyclical, and so are not meant to connect any one state to any other. Future work in this area should try to characterize how transitions, whether on Geels and Schot's terms or othwerise, connect to identifiable, general states of a sociotechnical system. It as at that point that collapse, or resilience can be better conceived in the sociotechnical realm.\\ 
\end{itemize}

Finally, much of this work has focused on differentiating between sociotechnical and socioecological systems, but as the work of Dourish (2010), Blevis (2007), and Tomlinson et al (2013) have shown the two are important compliments in a broader sustainability context. My work here demonstrates that commons as well as systems theory can aim to generalize to both sociotechnical and socioecological settings. 
