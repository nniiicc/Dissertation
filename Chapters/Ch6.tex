\chapter{6} 

\section{Introduction}

\emph{In this chapter I summarize the research results of this project and describe their limitations. I conlcude with the implications of this research for policy, practice, and theory of the commons.}

\section{Implications of Findings}

The implications of these research results are discussed with respect to policy (at both local and international scale), commons theory, and the practice of doing commons-based research

\subsection{Policy}

The implications of this work on policy are described through three state variables

\begin{itemize}
\item \textbf{Metric Diversity}: The case studies reviewed in this dissertation, with the exception of Galaxy Zoo, all suffered from a lack of metrics that help in evaluating the success, and sustainability of a project. It is clear that the context in which a commons operates, and the practices of its community will make promulgating metrics difficult, but a focus on formative evaluation techniques is an important and easy first step for policymakers to address. 
\item \textbf{Latency} Contrerars work on the importance of latency in depositing, and releasing data remains an underexpolred area for developing the commons in science settings. The NIH's 6 month embargo has created 

There is an important opportunity for federal agencies in the US, which are now revising 


\item \textbf{Governance Scale \& Scale Match}

\end{itemize}

\subsection{Theory}

\begin{itemize}
\item Rules-in-use
\item Polycentric 
\end{itemize}

\subsection{Practice}

\begin{itemize}
\item Knowledge Commons Framework
\item State Variable Coding
\end{itemize} 


\section{Conclusions}

This dissertation began by noting the differences between scientific knowledge, and the objects that result from knowledge production: on the one hand, scientific knowledge is assumed to be a public good, with low rivalry and costly excludability; on the other hand, the products of scientific research (data, software, journal publications, etc.) are assumed to have rivalry for access, which in turn requires state intervention, or privatiztion for managing access and use.   

The work here has demonstrated the efficacy of a hybrid model, the commons, used by four domains of knowledge production in research and development settings. In the right circumstances, the commons can be an effective model for -  balancing interests of the private and public sectors (e.g. genome commons, and urea cycle disorder network), creating long-enduring institutions for producing and provisioning valued databases (e.g. ICOADS, genome commons), and sustaining order and cooperation between large numbers of individuals competing for access to resources (e.g. observations made across all four cases). 

These case studies reviewed in Chapter 5 are an intial step in providing empirical findings about how knowledge commons work, how they evolve and change over time, and the ways that institutions for collective action differ in sustaining shared resource sets. Two important conclusions can be drawn from this work, related to polycentric governance models, and peer production.   

\subsection{Polycentric Models of Governance}
Research and development activities in the sciences can be divded up among many axis of categorization: basic vs. applied, theoretical vs. practical, inductive vs. deductive, etc. In contemporary research settings, a distinciton that cuts across many of categorizations is that of collective action; research and researchers increasingly collaborate, share resource sets, and divide up labor in purusit of grand challenge science problems (Atkins, 2003). Even mathmatics has turned to peer production to solve difficult theorems (Ball, 2014). 

Governing these new arrangements will be an important future challenge for their sustainability. The case study of ICOADS, the genome commons, and the Urea Cycle Disorder network has demonstrated that as a project matures, it comes with a need to formalize certain governance issues through rule making of self-organized groups. In the case of the Urea Cycle Disorder network, matching the scale of governance with the resource sets provisioned will prove difficult for a nationally funded, and exclusively USA based research project. But, the genome commons and ICOADS both present cases where international cooperation is required for both the provisioning, and the production of resource sets. 

The polycentric governance model observed in both commons developed less through purpseful choice, and more as a result of growth and evolution. For instance, the regime shifts that mark ICOADS changing from a self-governance model to a polycentric model was in response to selective pressures - such as a loss of funding from a major sponsor and the need to formally embed itself in a number of different bueracratic structures through the signing of agreements, the application for recognition as a critical infrastructure, etc. 

As these polycentric models themselves mature, an important next step is defining their particular features; What types of nested levels can be observed across polycentric models? When does the transition between self-governance and polycentric models occur? What are the negative externalities (the unintended consequences) of having an interdependent governance system like a polycentric system? How, and in what ways do the the negative (or positive) externalities of a polycentric system differ from that of Federalism? 

\subsection{Peer Production}

Peer production was described Chapter 1 as, 

\begin{quote}
``...radically decentralized, collaborative, and non-proprietary; based on sharing resources and outputs among widely distributed, loosely connected individuals who cooperate with each other without relying on either market signals or managerial commands.'' (2006, p. 60)
\end{quote}\\

.I have relied on Haythorthwaite's lightweight and heavyweight qualifications of peer production, but another qualification seems required of peer production given the case studies in this dissertation; ``relying'' on market signals or managerial commands does not mean that a commons is 1. egalitarian, or 2. absent marketplace . Rights and responsibilities assigned to actors through both formal and informal rules was one of the critical components of ICOADS sustaining itself as a commons. Similarly, the backg


\section{Limitations and Furture Work}

Reasearch in a sociotechnical setting, such as a knowledge commons, will inevitably suffer from a limited view of what are complex systems, and this research project is no exception. The small sample size (n=4) of this work limits its generalizabilty, and the partial view of what are large and complex international institutions is limited by my own knowledge of these systems.  Additionaly, the frameworks applied here are relatively new, and require a broader set of cases to test their efficacy for generating reliable, comparable case study results. The approach used to systematically coding state variables is relatively untested in sociotechnical systems research, and requires a number of important modifications, which I describe in detail below.

From the comparative case study work, two major questions remain open to future exploration: 

\begin{itemize}
\item Both Contreras and Madison relied on an accounth Mertonian norms, otherwise known as CUDOS - Communalism, Universalism, Disinterstedness, and Organize Skepticism (1973) - to explain their case study subjects being committed to openness and sharing. Although the concept of ``communalism'' seems a good fit for this context, Merton's norms in general are an awkward framework for a research agenda like the knowledge commons. Merton's normative theory is grounded in functionalism - he sees the actions, behaviors and adopted rules of a culture through latent and manifest functions that are, above all else, constructed for the sake of preserving a culture (1973). Functions explain how cultures organize and act, in order to remain the same (how they continue to \emph{function}), not how and why cultures evolove.   This seems to conflict with the major goal of the knowledge commons research agenda, which is to provide an alternative to the functionalist accounts of institutions, especially those that rely on intellectual property rights protections (see, Madisson, Frischmann and Strandburg, 2010 p. 17). 

And yet, in the ICOADS case study I too described a ``default'' moral framwork that my participants refferenced in describing the historical precedence of their work, especially as it related to Lt. Matthew Maury. Which also seems to conflict with the fact that, in practice, these systems resembled flexible evolution, and adaptation, rather than fierce guarding against change. An open research question then is which theories (political, sociological, or otherwise) can help explain the continuity of this normative dimension of communalism given the dynamic evolutionary account that occurs throughout this dissertation. To put it more simply, collective action requires dynamic change and readjustment for sustainability, so how is it that there remains a normative center upon which genetcists, rara disease researchers, astronomers, and climatologists all seem to agree?  

\item The instituional level case studies presented here, with the exception of the Galaxy Zoo, runs somewhat counter-intuitive to previously published literature about the culture of science being ``gift giving ''(Hagstrom, 1965) and collaboration having a propensity that is subject to bounded-rationally (Birnholtz, 2007), or a matter of individuals using a limited monopoly for rent extraction (Birnholtz and Bietz, 2003; Harvey, 2009). Traditional notions of rent are effective only when the costs and benefits of collaboration are symmetrically distributed amongst stakeholders. Collective action dilemmas, which emerge when resources are pooled to accomplish basic tasks, such as calculating historical weather trends or sequencing a species genome, pose a different set of problems than property-based concepts like ``rent'' or bounded-rationality can provide explanations for - by the very fact that they suffer from social dilemmas where costs are not symmetrically distributed. This is not to say that behavioral theories are innapplicable to the commons context; quite the opposite - there is a need to understand how those traditional notions of limited monopoly (rent) effect commons when nested in polycentric governance structures. Contreras offers ``latency analysis'' but this describes only one temporal aspect of the social dilemmas of sharing resources and recieving credit. An set of open questions about behavioral aspects of the knowledge commons are as follows: What 

\subsection{Resilience}

The work presented here describes how a knowledge commons effectively transitions between different governance regimes over time. It suggests a need to turn away from functional integrity explanations of sustainability, and instead adopt a framework of analysis that is more congruent with studying knowledge production, and the governance of knowledge products through systems concepts like adaptivity and resilience. Future work can and should go in a number of directions. I will summarize one below:

Adaptive cycles focus on two phases or transitions between system states: Phase one, the foreloop, is characterized by growth and accumulation, and phase two, the backloop, is a rapid response for reorganization and renewal (Holling, 1973). 

\begin{figure}
\includegraphics[width=4in, height=2.5in]{ad-cycle}\\
\captionbelow{Holling's adaptive cycle (1973), with four state changes and tranistions (loops) between different states}
\end{figure}

If a system doesn't develop a capacity to complete a backloop (from Omega to Alpha), it fails to renew and will collapse, hence it is not resilient to change and is unsustainable.  The sociotechnical systems described here all had the ability to complete a backloop, although with varying degrees of success, and over different (longer and shorter) periods of reorganization and renewal. As, In a path towards ecosystem collapse ``often the remedial responses simply continued and extended the process, protected by gradually increasing investments of money to monitor, subsidize and control.'' (Holling, 2007) It remains unclear from this work whether the knowledge commons case studies represented remedial responses or robust ones.   In the state variable coding developed in this dissertation, this was addressed partially thorugh the ``diagnostic'' variable, where either the symptoms or the underlying drivers of a selective pressure are treated. Future work on resciliency and adaptive cycles for sociotechnical systems should investigate this closer - trying to both characterize and better understand responses to selective pressures. More broadly, future work in this area needs to characterize components of change, such as those from Holling's model, and the transitions between those components. 

Geels and Schot have already developed a concept they call ``pathways to transition between sociotechnical regimes'' Like foreloops and backloops in the above socioecological model, the transitions approach conceives of regime change through four sustainability pathways called transformation, reconfiguration, technological substitution, and de-alignment & re-alignment (Geels and Schot, 2007). These do not conceive of state changes as cyclical, and so are not meant to connect any one state to any other. Future work in this area should try to characterize how transitions, whether on Geels and Schot's terms or othwerise, connect to identifiable, general states of a sociotechnical system. It as at that point that collapse, or resilience can be better conceived in the sociotechnical realm. 



\subsection{Limitations }


I indicated that the analysis included diverse resources sets, but spent a majority of my time focusing on data. Strangely, software remains undervalued - future work turns in this direction ...

Case studies while covering big data, were still relatively new and relatively small - the genome commons being the lone exception. Fields like chemistry, geology, or seismology, which have complex private sector relationships are an important target for future knowledge commons work. Trying to understand 

My n was small, and my knowledge of what is a complex system was even smaller

Much of the previous literature on data sharing, resource pooling, or software sustainability in science and technology studies focuses on individual actors, and their preferences for risk, altruism, reward, etc. 

Moving from behavioral level of ``motivations'' and ``practices'' of sharing, producing, and provisioning research objects, including data (Faniels and Jacobson, 2011), software (Howison and Herbsleb, 2012), or computing (Ribes and Finholt, 2009; Kee and Brown, 2012) to institutional level analysis requires a number of new  approaches, only one of which is introduced here. I show how a systematic approach to coding variables related to a particular aspect of this resource system can help reduce the complexity of understanding how and why governance models differ between and within a commons.

I have not however dealt at all with the ways that, at the individual level of practice, these types of motivations and behaviors map onto institutional theories. The two undoutedly exist at different levels of analysis, but finding ways to move between institutional levels and individual levels is critical for understanding resource sharing, reuse, and remixing in a contemporary science and technology setting.  


Polycentric systems are not a panacea. When there is not a match between governance levels, either in terms of scale or authority, conflict will often result. As polycentric systems have multiple levels of jurisdisction, there is multiple means of friction, disagreement, and redundancy. This is especially evident in science and technology where the state-based hierarchical system creates policy which ... as Borgman explains

``An investigator may be part of multiple, overlapping communities of interest, each of which may have different notions of what are data and different data practices. The boundaries of communities of interest are neither clear nor stable. In the case of data management plans, an investigator is asked to identify the appropriate community for the purposes of a specific grant proposal and for the proposed duration of that award.'' (Borgman, 2011)


I have emphasized the need to turn away from markets and states as the binary choice for managing shared resources, but in doing so I've largely left out the many positive aspects of both models. While I argue that capitalistic explanations of sustainability tend toward neo-liberal naivete of rational actors, so too might sweeping claims of the efficacy of commons for solving all that ails science research. 
It would be naive to thing that capitalism isn't now, and will continue to be a major... 
finding ways to integrate the work of Fuchs, and the critique of crowds 




The work presented here also suggests that beyond science commons, the sustainability of sociotechnical systems should be treated as a collective action dilemma - the groundwork for this approach in design based fields was laid by Paul Dourish (2010) in HCI. As he points out, design interventions for environmental sustainability are often focused on individual choice, which has (intentionally or not) ``transformed the problems of sustainability into the cost benefit trade-offs of rational actor economics'' (2010, p. 2). Again, abstracting from the environmental domain to the sociotechnical requires a number of modifications to work in HCI that focuss on ICTs for (environmental) sustainability. Developing new commons theory to explain institutional successes and failures, including case studies using the Knowledge Commons Framework, is an important first step. To continue advancing a robust empirical account of sustainability in sociotechnical systems, I suggest that further work is needed in the following areas: 

- Adapting resilience models
- Extending the approach developed here for a  


Although much of this work has focused on differentiating between and adapting, sociotechnical systems are importantly embedded in socioecological systems, and have profound implications for responsible environmental stewardship.


- coupling sociotechnical and socioecological systems analsyis... 
- some precedence as is found Geels et al... 

 

####%%%^^^&&& Polycentricity nests governance models, and this provides a 



- Metric diversity, in this case, is strongly tied to diagnostics; an indicator of healthy or robust knowledge commons may indeed be a diversity of metrics for evaluating success, attainment of goals, or more broadly conveying the importance of a resource set.

\subsubsection{Path Dependence and Lock-in}
A missing component of analysis in the current literature 

\subsubsection{'The Business of Public Goods'}

Public Entrepreneurship ... 





For insance, Velden unpacks theory selection with Gläsersian terms, noting that ``Researchers orient their actions towards creating knowledge that they can offer as novel contributions to the shared knowledge base. Hence the shared knowledge base ensures coordination of a collective of autonomous producers.  This coordination is decentralized, and not enforced by institutions or direct coordinating actions. The social order of a scientific production community is an emergent property'' (p. 262)


Velden's model concieves of openness (and cooperation) and secrecy (and competition) in scientific fields as a systemic tension inherent to the collective production of scientific knowledge.


\section{Theoretical Contributions}

\subsubsection{Normative dimmesion}

Previous studies of sustainability in sociotechnical systems related to knowledge production have 

Fry et al began the process of tying individual actions studied at the behavioral level to policy and organizational structure at the institutional level, allowing for a normative account of sustainability to emerge... 

This dissertation presents a way to further extend this process... 

The overall conclusion of this work, is that a normative approach to sustainability is necessary for knowledge production systems that impact societal well-being. Just as policy, such as the Information Quality Act () establishes a set of ... so too should. 


Sheila Jasanoff's work on the the co-production of knowledge in science and society argues that ``technoscience ''different than that of previous works on nature and society. In the domain of the social and the technical, sustainability is less about balancing stocks and flows, or identifying targeted populations whom we \emph{should} conserve resources for, as it is about maintaining order along along moral, political and symbolic dimmensions. The co-production account of sustainability, in some sense, provides the   

The original defitnion that I offered was:

a sociotechnical system that over the long run, enhances both the quality and the resource base on which science depends, provides for the continued support of resources (such as data, or software, or instrumentation), is economically viable, and enhances the quality of science being conducted.

To add to this account we can say a sustainable sociotechnical systems must be able to maintain order among different users, politics, provisioners, producers; it must balance resources including the 



Increasingly important for technological solutionism, which takes the logic of factor-four analysis 

Three pillars of sustainability, otherwise known as tripple-bottom line accounting 


\section{Implications for Policy \& Practice}


\section{Future Work}


\textbf{Resilience processes}

Two themes underly resilience thinking - Thresholds and Adaptive Cycles:

Thresholds: level where steady state is changed

Adaptive cycles - how social ecological systems change over time - system dynamics. Usually in a sequence of rapid growth, conservation, release, and reorganization (usually in that sequence). These cycles operate over many different scales of time and space. Manner in which they are linked across scales is crucial to the dynamics of the systems.

tripple bottom line - environment, economic and societal. 


Finds ways to type resilience and vulnerability... 

Which types of disturbances are ICOADS resilient or vulnerable to?

\section{Chapter Summary}