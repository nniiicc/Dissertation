\chapter{6} 

\emph{In this chapter I summarize the research results of this project, describe the theoretical contributons of this work, and offer directions for future work. I conclude with the implications for science and technology policy aimed at sustaining sociotechnical systems}

\section{Conclusions}

\subsubsection{Limitations}

\subsubsection{Polycentric Governance: Principle of Complimentarity}

Complex systems, whether observed through direct methods like ethnography, indirect methods like informetrics, or through explanations of participants within the system, such as the interpretive itneractionist approach, will suffer from the principle of complimentarity - this holds that there are generally four ways of describing a system at multiple levels :

Redundant - a redundnant system description is when one actors description can be fully incorporated by another

Equivalent - an equivalent system description takes place when two actors give the same, description of a system

Independent - Includes two system level desctiptions with no overlap 

Complementary - When a set of descriptions lacks any of the above. Complimentarity usually results from partial descriptions, which can be (typically) eliminiated if we pursue more details, and more (Easterbrook, 2004)

However, the principle of complimentarity holds that complex systems can only ever be partially described, and as such they can 






\section{Theoretical Contributions}

\subsubsection{Normative dimmesion}

Previous studies of sustainability in sociotechnical systems related to knowledge production have 

Fry et al began the process of tying individual actions studied at the behavioral level to policy and organizational structure at the institutional level, allowing for a normative account of sustainability to emerge... 

This dissertation presents a way to further extend this process... 

The overall conclusion of this work, is that a normative approach to sustainability is necessary for knowledge production systems that impact societal well-being. Just as policy, such as the Information Quality Act () establishes a set of ... so too should. 


Sustainability, as it relates to sociotechnical systems is as much about preservation and conservation, as it is about maintaining order along three dimmensions:

1. Morally and metaphysically sustaining: divide the world of hybrids and cyborgs into less ambiguous categories to be dealt with by law or custom. (also - self and other; structure and agency; state and citizen.) 

2. Politically sustaining: help societies to accommodate new knowledge and technological capabilities without tearing apart (often by reaffirming the legitimacy of existing social arrangements

3. Symbolically sustaining: provide surrogate markers for the continued validity of certain familiar dispensations when uncertainties threaten to overwhelm or disrupt.\\


Increasingly important for technological solutionism, which takes the logic of factor-four analysis 

\section{Future Work}


\textbf{Resilience processes}

Two themes underly resilience thinking - Thresholds and Adaptive Cycles:

Thresholds: level where steady state is changed

Adaptive cycles - how social ecological systems change over time - system dynamics. Usually in a sequence of rapid growth, conservation, release, and reorganization (usually in that sequence). These cycles operate over many different scales of time and space. Manner in which they are linked across scales is crucial to the dynamics of the systems.

tripple bottom line - environment, economic and societal. 



\section{Implications for Policy \& Practice}

\section{Chapter Summary}