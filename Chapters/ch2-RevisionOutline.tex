Chapter 2. 



1. Studies of "sharing" in science... 

- typically focus on a single resource - data
- at the individual, behavioral level
- there is limited work looking at the context in which "sharing" 
- - relates to a diverse set of resources
- - happens at an institutional level


But increasingly, broad institutional cooperation is needed: 
- to tackle grand challenge science problems
- to overcome limited research funding for sustaining these arrangements

This is achieved, in part, through peer production: 
- define peer prodcution
- describe peer production in contemporary science settings

Contribtion: This dissertation will add to our understanding of resource sharing at institutional level, through peer production. 

2. Sustaining sharing...

"Sharing" at this level creates different kinds of social dilemmas than at the individual level - such as how to approach common property, as opposed to individual property. 

Much work exists, for instance, on commons in environmental domains ... 

What this work teaches us is that.. differences in types of goods is important to context of how they are sustained... 

Common resources are often sustained through informal institutions... but depend on rules. 

Sustaining shared resource systems reqiures - governance... 

Contribution: This dissertation will add to our understanding of effective governance in shared resource systems... 

3. Differences between environmental and knowledge commons

- Engineered
- User / Producer roles
- Complexity 

Contribution: 