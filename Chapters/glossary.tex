\chapter{Glossary}

 This appendix includes a glossary of concepts used in this document. Where appropriate, these terms are described in the document. 

\textbf{Commons}\\ 
The commons is a generic term that can refer to a resource management approach,  as well as the shared resource system being managed. In the broadest sense, the commons are marked by ``privileges and immunities for an undefined public, rather than rights and powers for a defined person or persons...The main function of commons is to institutionalize freedom to operate, free of the particular risk that any other can deny us use of that resource set, subject to symmetric known constraints and the risk of congestion applicable to that resource set, under those rules, within the expected population of users.'' (Benkler, 2014).  The two classes of commons can inlcude, on the one hand pastures, forests, and irrigation districts of the natural world, and on the other, high-performance computing, software libraries, and data archives of the digital realm. This dissertation deals exclusively with the latter type of commons. 

\textbf{Sustainability}\\ 
Sustainability can be a relative term. In the context of this project, a sustainable knowledge commons is a sociotechnical system that over the long run, enhances both the quality and the resource base on which science depends, provides for the continued support of resources (such as data, or software, or instrumentation), is economically viable, and enhances the quality of science being conducted. In this sense, a sustainable knowledge commons doesn't just persist over time, but evolves given different external and internal pressures. This idea is a central theme of Chapter 5. 

\textbf{Sociotechnical Systems}\\ 
This project conceives of a sociotechnical system as the mutual constitution of people and technologies in social, political and economic settings that require collective action in order to effectively function over time (Sawyer and Jahairi, 2013). The contextual and embedded nature of sociotechnical systems makes governance, institutional arrangements, and symmetric information exchange paramount to their success. 

\textbf{Infrastructure}\\ 
Infrastructure, like sustainability, is an inherently relational concept. In a science and technology setting, infrastructures can be defined in relation to organized practices of communities, disciplines, or fields of study (Star and Ruhleder, 199). The term of art for scientific infrastructures has recently become cyberinfrastructure (Atkins, 2000), which I take to mean ``...the set of organizational practices, technical infrastructure and social norms that collectively provide for the smooth operation of scientific work at a distnace. All three sets are objects of design and engineering; a cyberinfrastructure will fail if any one is ignored''(Edwards et al., 2007, p. 6)

Given these explanations, a natural question is what are the blurred lines between infrastructures, cyberinfratructures, sociotechnical systems, and commons? 

Commons focus on the institutional arrangements, with an emphasis on the rules and governance of people, resources, and the bundle of property rights that are negotiated for their long-term sustainability. Infrastructure, and cyberinfrastructures, are a congruent, and complimentary view of these interconnected elements. Infrastructures can (and often are) managed as a common property (Frischmann, 2005), for which collective action is required to keep the ``smooth operation of scientific work at a distance'' occurring. Although two literatures - commons and infrastructure studies -  make reference to one another, analysis of their relationships are rarely combined (a notable exception, Frischmann, 2005). For instance, commons scholars increasingly acknowledge the importance of developing an account of the infrastructural resources that shape, and are shaped by long-term interactions within shared resource systems (Dietz et al., 2010). Part of the goal of this research project is to better align the findings from a long line of cyberinfrastructure studies, and the emerging knowledge commons frameworks described in Chapter 2. 

\textbf{Governance}\\ 
A helpful definition of governance is that it includes,  ``a complex of public and/or private coordinating, steering and regulatory processes established and conducted for social (or collective) purposes where powers are distributed among multiple agents, according to formal and informal rules'' (Burns and Stöhr 2011: 234). As stated above, a sociotechnical perspective  recognizes that collective action is needed to sustain the social relations, orderings, and enforcement of cultural norms, as well as the technical components that allow a commons to effectively function over time. A governance model, a term also used throughout the dissertation, implies the sets of``institutional arrangements (such as rules, policies, and governance activities) that are used by one or more actor groups to interact with and govern'' shared resources (Cox et al., 2015). 

Governance models typically differ in their centralization (or decentralization) of decision making power - such as self-governing or monocentric governance. This dissertation explores polycentric governance models that nest authority at multiple levels, types, sectors, or jurisdictions. A preliminary defintion of polycentric is it is a ``structural feature of social systems of many decision centers having limited and autonomous prerogatives and operating under an overarching set of rules.'' (Aligica and Tarko, 2011).  The overlap of these different levels create collective action dilemmas, which require multiple rule types to function efficiently. Polycentric models, as I discuss in the following chapter, are increasingly effective for helping sociotechnical systems cope with social dilemmas related to sustainability.

\textbf{Resilience}\\ 
The notion of resilience in socioecological systems is partially evolutionary, and partially ecological. Holling formulated this idea in the early 1970s, defining resilience as a ``measure of the persistence of systems and of their ability to absorb change and disturbance and still maintain the same relationships between populations or state variables.'' (Holling, 1974) In Chapter 5 I focus on transitions of ICOADS between different governance regimes; the ability of a commons to routinely make these governance transitions is described through \emph{resiliency processes}. The NSF's Critical Resilient Interdependent Infrastructure Systems and Processes (CRISP) program offers a helpful definition of a resilient process for infrastructures; these are ``the features of an infrastructure’s inherent capacity to resist disturbances, initial loss of service quality, and trajectory of service restoration. Conceived as a process, infrastructure resiliency can be achieved by a myriad strategies in addition to simple repair and replacement.'' (2013). The ``myriad strategies'' of ICOADS resiliency are a subject explored throughout this project.